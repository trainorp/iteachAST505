\documentclass{article}
\textwidth=7in
\textheight=9.5in
\topmargin=-1in
\headheight=0in
\headsep=.5in
\hoffset  -1in
\pagestyle{empty}
\pretolerance=10000
\tolerance=2000 
\emergencystretch=10pt
\setlength{\unitlength}{1in}

\usepackage{hyperref}

\begin{document}

\begin{center}
{\bf AST 505-M03: Statistical Inference I}
\begin{picture}(6,.1) 
\put(0,0) {\line(1,0){6.25}}         
\end{picture}\\
\end{center}

\noindent\textbf{Course description:} This course provides an introduction to the concepts and methods of statistical inference. Topics include the binomial, chi-square, normal, and Student’s t-distributions, one-way analysis of variance, and simple linear regression. \\

\noindent\textbf{Prerequisite:} consent of the instructor \\

\noindent\textbf{Instructor:} Patrick Trainor, PhD, MS, MA \\
\noindent\textbf{Email:} ptrainor@nmsu.edu \\
\noindent\textbf{Office:} Guthrie 412 \\
\noindent\textbf{Office phone:} (575) 646-2822 \\

\noindent\textbf{Course dates:} 8/22/2019 - 12/13/2019 \\


\noindent\textbf{Weekly schedule:}
\begin{center}
	\begin{tabular}{p{4cm}p{4.5cm}p{3.5cm}p{4cm}}
		\underline{Type of meeting} & \underline{Day} & \underline{Time} & \underline{Location} \\
		Lecture & Tuesday, Thursday & 9:00-10:15 AM & Domenici Hall 018 \\
		Lab & Tuesday & 3:00-4:50 PM &  Guthrie 303 \\
		Office Hours* & Tuesday & 10:30 - 11:30 AM & Guthrie 412 \\
		Office Hours* & Thursday & 12:00-2:00 PM & Guthrie 412
	\end{tabular}
\end{center}
\noindent *Other times by appointment (please email)\\

\noindent\textbf{Textbook:}  \emph{An Introduction to Statistical Methods and Data Analysis}, 7th edition, by Ott and Longnecker\\
The e-book version may be cheaper than the hardcover (as may be rental copies) \\

\noindent\textbf{Grading policy:} 
Grades will be based on collected homeworks (14), lab assignments (13), quizzes (6), exams (2), and a final exam. Each student's lowest homework, lab assignment, and quiz grade will be dropped. A breakdown of points that can be attained in this course is included below:

\begin{center}
	\begin{tabular}{cc}
		\textbf{Item} & \textbf{Points possible} \\
		\hline
		14 homeworks & 15 each; 195 Total\\
		13 lab assignments & 8 each; 96 Total \\
		6 quizzes & 40 each; 200 Total \\
		2 exams & 150 each; 300 Total \\
		1 final exam & 209 Total \\
		\hline
		\textbf{Total} & \textbf{1000}\\
	\end{tabular}
\end{center}

Final grades will be assigned as:
\begin{center}
	\begin{tabular}{p{2cm}p{2.5cm}}
		\textbf{Points} & \textbf{Final Grade} \\ \hline
		970 - 1000 & A+ \\
		930 - 969 & A \\
		900 - 929 & A- \\
		870 - 900 & B+ \\
		830 - 869 & B \\
		800 - 829 & B- \\
		770 - 800 & C+ \\
		730 - 769 & C \\
		700 - 729 & C- \\
		600 - 699 & D \\
		0 - 599 & F \\ \hline
	\end{tabular}
\end{center}

\noindent\textbf{Homework:} Homework assignments will be assigned on Tuesday (or Thursday in the case that there is no lecture on Tuesday). They will be due the following Tuesday at the beginning of the lecture or exam. Student's may type or handwrite the assignment. \\

\noindent\textbf{Labs:} The purpose of the course labs is: (a) to reinforce topics covered in lecture by solving practical problems, and (b) to provide the student a guided introduction to using a software tool for applying statistical methods. The students will not need to download or install any software but will use an interactive online notebook. These notebooks are also intended to provide the student with ``recipes'' that they may customize for their own data analyses outside of the course. Students will be required to submit their solutions to assigned projects as a ``lab assignment''. Students should be able to finish these assignments during the lab period; however, the lab assignments will not be due until the following lab period.  \\

\noindent\textbf{Exams:} Exam dates are provided in the course schedule (below). Exams will be closed book; however, a sheet with relevant formulas will be provided by the instructor. The problems included in the exams will be very similar to homework problems that have been assigned previously. \\

\noindent\textbf{Quizzes:} Six short ($\approx$ 10 minutes) quizzes will occur during scheduled lectures. The problems included in the quizzes will be very similar to homework problems that have been assigned. The dates of these quizzes will not be announced in advance. \\

\noindent\textbf{Absence policy:} 
The University's policy on absences is included below:

\noindent{\em Absence from class will be excused for the reasons listed below, with the expectation that students will be permitted to, and will be responsible for, making up course work missed due to an excused absence.
\begin{enumerate}
	\item The student is representing the university at a function or event, and is making satisfactory progress in the class (e.g. ASNMSU officials representing the university at a legislative session, student-athletes competing in university scheduled athletic events, students traveling to a university-approved educational field trip or conference).
	\item The student is legally obligated to participate in training or has received a military deployment as an active member of the military or Reserve/National Guard. Students will provide the Military and Veterans Programs Office with official military documentation (paper, electronic orders or a Unit’s memorandum) with as much advance notification as possible.
	\item The student is obligated to attend or participate in a court or legal proceeding by summons or subpoena (e.g. to serve as a juror, to testify as a witness).
	\item The student has requested a reasonable accommodation based on a conflict between an academic requirement and a religious practice or belief.
	\item The student has been granted a reasonable accommodation consistent with the Americans with Disabilities Act of 1990 as amended and/or Section 504 of the Rehabilitation Act of 1973, amended as documented by either the Student Accessibility Office or the Office of Institutional Equity.
	\item The student is subject to an interim measure in accordance with Title IX of the Education Act of 1972, as amended, as documented by the Office of Institutional Equity.
\end{enumerate}
Absences based on extenuating circumstances outside the control of the student other than those listed above may be excused at the discretion of the faculty member.
}\\

Makeup quizzes or exams will only be permitted for extreme extenuating circumstances (e.g. death of an immediate family member, or the student is hospitalized / needs medical attention). For less than extreme extenuating circumstances, please remember that your lowest quiz and exam grades will be dropped. Late submission of homework or lab reports will not be accepted barring extreme extenuating circumstances. If you believe that you have an extenuating circumstance that is outside of your control, please contact the instructor as soon as possible. If you miss a lecture or lab, you are responsible for obtaining notes, assignments, and any handouts from other students. \\

\noindent\textbf{Final and incomplete grades:} Final grades are normally available to students on the web within a day or two of the deadline for submitting grades. Under FERPA regulations, public posting of grades is not permitted even with a confidential PIN number identifier. A minimum grade of B- must be earned to receive a grade of S. Under university policy, incompletes may be given only if a student has a passing grade at mid-semester (the last day to withdraw from a class) and is precluded from successful completion of the second half of the course by a documented illness, documented death, family crisis or other similar circumstances beyond the student's control. An incomplete should not be given to avoid assigning a grade for marginal or failing work. \\

\noindent\textbf{Academic and non-academic misconduct:}
The Student Code of Conduct defines academic misconduct, non-academic misconduct and the consequences or penalties for each.  The Student Code of Conduct is available in the NMSU Student Handbook online:  \url{http://studenthandbook.nmsu.edu/}. Academic misconduct is explained here: \\ \url{http://studenthandbook.nmsu.edu/student-code-of-conduct/academic-misconduct/}. \\

\noindent\textbf{Plagiarism:} Plagiarism is using another person's work without acknowledgment, making it appear to be one's own. Intentional and unintentional instances of plagiarism are considered instances of academic misconduct and are subject to disciplinary action such as failure on the assignment, failure of the course or dismissal from the university. The NMSU Library has more information and help on how to avoid plagiarism at \url{http://lib.nmsu.edu/plagiarism/}. \\

\noindent\textbf{Assistance for members of the military and veterans:} The Office of Military and Veterans Programs (\url{https://mvp.nmsu.edu/}) provides services to current and former service members.  The office is located in room 244 of the Corbett Center Student Union and can be reached at (575) 646-4524 or mvp@nmsu.edu. \\

\noindent\textbf{Disability accommodation:} Section 504 of the Rehabilitation Act of 1973 and the Americans with Disabilities Act Amendments Act (ADAAA) covers issues relating to disability and accommodations. If a student has questions or needs an accommodation in the classroom (all medical information is treated confidentially), contact:\\

\noindent Trudy Luken, Director \\
Student Accessibility Services (SAS) \\
Corbett Center Student Union, Rm. 208 \\
Phone: (575) 646-6840 \\
E-mail: sas@nmsu.edu \\
Website: \url{http://sas.nmsu.edu/} \\
 
\noindent\textbf{Prohibition of discrimination:} New Mexico State University, in compliance with applicable laws and in furtherance of its commitment to fostering an environment that welcomes and embraces diversity, does not discriminate on the basis of age, ancestry, color, disability, gender identity, genetic information, national origin, race, religion, retaliation, serious medical condition, sex (including pregnancy), sexual orientation, spousal affiliation, or protected veteran status in its programs and activities, including employment, admissions, and educational programs and activities. Inquiries may be directed to the Laura Castille, Executive Director, Title IX and Section 504 Coordinator, Office of Institutional Equity, P.O. Box 30001, E. 1130 University Avenue, Las Cruces, NM 88003; 575.646.3635; 575-646-7802 (TTY); equity@nmsu.edu. Title IX prohibits sex harassment, sexual assault, intimate partner violence, stalking and retaliation. For more information on discrimination or Title IX, or to file a complaint contact: \\

\noindent Laura Castille, Executive Director and Title IX Coordinator \\
Office of Institutional Equity (OIE) \\
O'Loughlin House, 1130 University Avenue \\
Phone: (575) 646-3635 \\
E-mail: equity@nmsu.edu \\
Website: \url{http://eeo.nmsu.edu/} \\

\noindent\textbf{Other NMSU Resources:}

\begin{tabular}{p{7cm}p{8cm}}
NMSU Police Department & (575) 646-3311 \& www.nmsupolice.com \\
NMSU Police Victim Services & (575) 646-3424 \\
NMSU Counseling Center & (575) 646-2731 \\
NMSU Dean of Students & (575) 646-1722 \\
For any on-campus emergencies & 911
\end{tabular}
\newpage
\noindent\textbf{Tentative course schedule (subject to change):}

\begin{center}
	\begin{tabular}{|p{2.5cm}|p{1.5cm}|p{9cm}|p{2.8cm}|}
		\hline
		Date & Type & Topic & Readings \\ \hline \hline
		8/22 (Th) & Lecture & 1) Course Introduction \& What is statistical inference? & Ch. 1 \\ \hline
		8/27 (Tu 9AM) & Lecture & 2) Design of experiments / studies & Ch. 2 \\ \hline
		8/27 (Tu 3PM) & Lecture & 3) Descriptive statistics 1 & Ch. 3 \\ \hline
		8/29 (Th) & Lecture & 4) Descriptive statistics 2 & Ch. 3 \\ \hline
		9/3 (Tu 9AM) & Lecture & 4) Descriptive statistics 2 (cont.) & Ch. 3 \\ \hline
		9/3 (Tu 3PM) & Lab & Lab 1) Descriptive statistics lab & \\ \hline
		9/5 (Th) & Lecture & 5) Events \& probability laws & Ch. 4.2 - 4.5 \\ \hline
		9/10 (Tu 9AM) & Lecture & 6) Random variables \& probability distributions & Ch. 4.6 - 4.10 \\ \hline
		9/10 (Tu 3PM) & Lab & Lab 2) Probability lab & \\ \hline
		9/12 (Th) & Lecture &  7) Sampling distributions \& normal approximations & Ch. 4.11 - 4.14 \\ \hline
		9/17 (Tu 9AM) & Lecture & 8) Inferences about the mean & Ch. 5.2 - 5.6 \\ \hline
		9/17 (Tu 3PM) & Lab & Lab 3) Normal approximation lab \& Exam review & \\ \hline
		9/19 (Th) & Exam & Exam 1: Lectures 1 - 8 & \\ \hline
		9/24 (Tu 9AM) & Lecture & 9) Inferences about the median & Ch. 5.9 \\ \hline
		9/24 (Tu 3PM) & Lab & Lab 4) Measures of center \& spread lab  & \\ \hline
		9/26 (Th) & Lecture & 10) Inferences about differences in means and paired samples & Ch. 6.2, 6.4 \\ \hline
		10/1 (Tu 9AM) & Lecture &  11) Nonparametric inference & Ch. 6.3, 6.5 \\ \hline
		10/1 (Tu 3PM) & Lab & Lab 5) Difference in means lab & \\ \hline 
		10/3 (Th) & Lecture & 12) Inference and sample size & Ch. 5.3, 6.6 \\ \hline
		10/8 (Tu 9AM) & Lecture & 13) Inferences about variances & Ch. 7.1 - 7.4 \\ \hline
		10/8 (Tu 3PM) & Lab & Lab 6) Sample size lab & \\ \hline 
		10/10 (Th) & Lecture & 14) Analysis of Variance (ANOVA) 1 & Ch. 8.2 - 8.4 \\ \hline
		10/15 (Tu 9AM) & Lecture & 15) Analysis of Variance (ANOVA) 2 & Ch. 8.2 - 8.4 \\ \hline
		10/15 (Tu 3PM) & Lab & Lab 7) ANOVA Lab & \\ \hline
		10/17 (Th) & Lecture & 16) Non-parametric alternative to ANOVA & Ch. 8.5 \\ \hline
		10/22 (Tu 9AM) & Lecture & 17) Data transformations & Ch. 8.6 \\ \hline
		10/22 (Tu 3PM) & Lab & Lab 8) ANOVA / Non-parametric lab  & \\ \hline
		10/24 (Th) & Lecture & 18) Multiple comparisons & Ch. 9.2 - 9.7 \\ \hline
		10/29 (Tu 9AM) & Lecture & 19) Inferences about a proportion & Ch. 10.2 \\ \hline
		10/29 (Tu 3PM) & Lab & Lab 9) Multiple comparisons lab \& Exam review & \\ \hline
		10/31 (Th) & Exam & Exam 2: Lectures 10 - 18 & \\ \hline
		11/5 (Tu 9AM) & Lecture & 20) Inferences about a difference in proportions & Ch. 10.3 \\ \hline
		11/5 (Tu 3PM) & Lab & Lab 10) Proportions lab & \\ \hline
		11/7 (Th) & Lecture & 21) Chi-squared tests \& contingency tables & Ch. 10.4 - 10.7 \\ \hline
		11/12 (Tu 9AM) & Lecture & 22) Odds and odds ratios & Ch. 10.7 \\ \hline
		11/12 (Tu 3PM) & Lab & Lab 11) Chi-square test lab & \\ \hline
		11/14 (Th) & Lecture & 23) Linear regression \& parameter estimation & Ch. 11.2 \\ \hline
		11/19 (Tu 9AM) & Lecture & 24) Inferences about regression parameters \& correlation  & Ch. 11.3, 11.6 \\ \hline
		11/19 (Tu 3PM) & Lab & Lab 12) Linear regression lab & \\ \hline
		11/21 (Th) & Lecture & 25) Linear regression: prediction \& fit diagnostics & Ch. 11.4 - 11.5 \\ \hline
		11/26 (Tu 9AM) & & Thanksgiving Holiday (No class) & \\ \hline
		11/26 (Tu 3PM) & & Thanksgiving Holiday (No class) & \\ \hline
		11/28 (Th) & & Thanksgiving Holiday (No class) & \\ \hline
		12/3 (Tu 9AM) & Lecture & 26) Multiple regression* & Ch. 12.3 - 12.5\\ \hline
		12/3 (Tu 3PM) & Lab & Lab 13) Linear regression lab 2 & \\ \hline
		12/5 (Th) & Lecture & 27) Logistic regression* \& Final Exam Review  & Ch. 12.8 \\ \hline
		12/12 (Th) & Exam &  Final Exam (8 AM - 10AM) & \\ \hline
	\end{tabular}
\end{center}
*Denotes topics that may be removed if more time is needed for covering the other topics


\end{document}
