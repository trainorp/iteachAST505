\documentclass[xcolor=dvipsnames]{beamer} 
\usetheme{AnnArbor}
\usecolortheme{beaver}

\usepackage{amsmath,graphicx,booktabs,tikz,subfig,color,lmodern}
\definecolor{mycol}{rgb}{.4,.85,1}
\setbeamercolor{title}{bg=mycol,fg=black} 
\setbeamercolor{palette primary}{use=structure,fg=white,bg=red}
\setbeamercolor{block title}{fg=white,bg=red!50!black}
% \setbeamercolor{block title}{fg=white,bg=blue!75!black}

\title[Lecture 11]{Lecture 11: Nonparametric inference}
\author[Patrick Trainor]{Patrick Trainor, PhD, MS, MA}
\institute[NMSU]{New Mexico State University}
\date{September 30, 2019}

\newcommand\myeq{\mathrel{\overset{\makebox[0pt]{\mbox{\normalfont\tiny\sffamily D}}}{=}}}

\begin{document}

\begin{frame}
\maketitle
\end{frame}

\begin{frame}{Outline}
\tableofcontents[hideallsubsections]
\end{frame}

\section{Introduction}
\begin{frame}{Outline}
\tableofcontents[currentsection,subsectionstyle=show/shaded/hide]
\end{frame}

\begin{frame}{Wilcoxon Rank-Sum Tests}
	\begin{itemize}
		\item One of the assumptions of t-tests is that the data be normally distributed. If the sample size is very large, this assumption is less important
		\item[]
		\item However, if our data distribution is far from normal, we want to use ``nonparametric tests'' rather than t-tests
		\item[]
		\item \textbf{\emph{Wilcoxon rank-sum test:}} The Wilcoxon rank-sum test is a nonparametric test that assumes the following:
		
		\begin{enumerate}
			\item We have two independent random samples with sample sizes $n_1$ and $n_2$: $\{x_1, x_2, \hdots, x_{n_1}\}$ and $\{y_1, y_2, \hdots, y_{n_2}\}$
			\item[]
			\item The population distributions of the $x$'s and $y$'s are identical but one may be shifted to the left or right of the other
		\end{enumerate}
	\end{itemize}
\end{frame}

\begin{frame}{Location shift}
	\begin{center}
		\includegraphics[width=.9 \linewidth]{locationShift}
	\end{center}
\end{frame}

\begin{frame}{Location shift}
	\begin{itemize}
		\item Notation for location shift: $y \myeq x + \Delta$
	\end{itemize}
\begin{center}
	\includegraphics[width=.85 \linewidth]{locationShift2}
\end{center}
\end{frame}

\begin{frame}{Location shift}
	\begin{itemize}
		\item Notation for location shift: $y \myeq x + \Delta$
		\item[]
		\item If $\Delta > 0$, then the $y$'s are typically larger and the distribution of $y$'s is to the right of $x$'s
		\item[]
		\item If $\Delta > 0$, then the $y$'s are typically smaller and the distribution of $y$'s is to the left of $x$'s
	\end{itemize}
\end{frame}

\begin{frame}{Wilcoxon Rank-Sum test}
	\begin{itemize}
		\item The Wilcoxon rank-sum test is based on ``ranks'' rather than measurement values
		\begin{itemize}
			\item The smallest value in the combined sample has rank 1
			\item[]
			\item The largest value in the combinded sample has rank $N = n_1 + n_2$
			\item[]
		\end{itemize}
	\item Process for determining Rank-Sum:
		\begin{enumerate}
		\item List the data values in the combined data set from smallest to largest
		\item In the next column assign numbers $1$ to $n$ to the data values by their order
		\item If there are ties in the combined data set, the ranks for the observations in the tie are taken to be the average of those ranks
		\item Let $T$ denote the sum of ranks for the observation from the $y$'s
	\end{enumerate}
	\end{itemize}
\end{frame}

\begin{frame}{Example Data}
\begin{itemize}
	\item Research question: Are survival times worse (lower) for Stage 4 tumors than Stage 3 tumors? 
\end{itemize}
\begin{center}
	\begin{tabular}{|c|c|}
		\hline
		\textbf{Survival Time (months)} &  \textbf{Stage} \\ \hline \hline
		52.9  &Stage 4 \\ \hline
		12.9 &Stage 4 \\ \hline
		34.9& Stage 4 \\ \hline
		38.7 &Stage 4 \\ \hline
		22.9& Stage 4 \\ \hline
		22.8 &Stage 4 \\ \hline
		60.0 &Stage 3 \\ \hline
		95.8& Stage 3 \\ \hline
		33.2 &Stage 3 \\ \hline
		54.5 &Stage 3 \\ \hline
		119.3 &Stage 3 \\ \hline
		47.0 &Stage 3 \\ \hline
	\end{tabular}
\end{center}
\end{frame}

\begin{frame}{Example Data}
\begin{itemize}
	\item Assume Stage 4 tumors are the $y$'s and Stage 3 tumors are the $x$'s
	\item Then we want to know if: $y \myeq x + \Delta$, where $\Delta < 0$
\end{itemize}
\begin{center}
	\begin{tabular}{|c|c|}
		\hline
		\textbf{Survival Time (months)} &  \textbf{Stage} \\ \hline \hline
		52.9  &Stage 4 \\ \hline
		12.9 &Stage 4 \\ \hline
		34.9& Stage 4 \\ \hline
		38.7 &Stage 4 \\ \hline
		22.9& Stage 4 \\ \hline
		22.8 &Stage 4 \\ \hline
		60.0 &Stage 3 \\ \hline
		95.8& Stage 3 \\ \hline
		33.2 &Stage 3 \\ \hline
		54.5 &Stage 3 \\ \hline
		119.3 &Stage 3 \\ \hline
		47.0 &Stage 3 \\ \hline
	\end{tabular}
\end{center}
\end{frame}

\begin{frame}{Calculating the rank-sum}{Example Data}
\begin{itemize}
	\item Our first step is to order the data from smallest to largest
\end{itemize}
\begin{center}
	\begin{tabular}{|c|c|}
		\hline
		\textbf{Survival Time (months)} &  \textbf{Stage} \\ \hline \hline
		12.9 &Stage 4 \\ \hline
		22.8 &Stage 4 \\ \hline
		22.9& Stage 4 \\ \hline
		33.2 &Stage 3 \\ \hline
		34.9& Stage 4 \\ \hline
		38.7 &Stage 4 \\ \hline
		47.0 &Stage 3 \\ \hline		
		52.9  &Stage 4 \\ \hline
		54.5 &Stage 3 \\ \hline				
		60.0 &Stage 3 \\ \hline
		95.8& Stage 3 \\ \hline
		119.3 &Stage 3 \\ \hline
	\end{tabular}
\end{center}
\end{frame}

\begin{frame}{Calculating the rank-sum}{Example Data}
\begin{itemize}
	\item Now we assign the ranks
\end{itemize}
\begin{center}
	\begin{tabular}{|c|c|c|}
		\hline
		\textbf{Survival Time (months)} &  \textbf{Stage} & \textbf{Rank}\\ \hline \hline
		12.9 &Stage 4 & 1 \\ \hline
		22.8 &Stage 4 & 2\\ \hline
		22.9& Stage 4 & 3\\ \hline
		33.2 &Stage 3 & 4\\ \hline
		34.9& Stage 4 & 5\\ \hline
		38.7 &Stage 4 & 6\\ \hline
		47.0 &Stage 3 & 7\\ \hline		
		52.9  &Stage 4 & 8\\ \hline
		54.5 &Stage 3 & 9\\ \hline				
		60.0 &Stage 3 & 10\\ \hline
		95.8& Stage 3 & 11\\ \hline
		119.3 &Stage 3 & 12\\ \hline
	\end{tabular}
\end{center}
\end{frame}

\begin{frame}{Calculating the rank-sum}{Example Data}
\begin{itemize}
	\item Sum: $T = 1 + 2 + 3 + 5 + 6 + 8 = 25$
\end{itemize}
\begin{center}
	\begin{tabular}{|c|c|c|}
		\hline
		\textbf{Survival Time (monts)} &  \textbf{Stage} & \textbf{Rank}\\ \hline \hline
		12.9 &Stage 4 & 1 \\ \hline
		22.8 &Stage 4 & 2\\ \hline
		22.9& Stage 4 & 3\\ \hline
		33.2 &Stage 3 & 4\\ \hline
		34.9& Stage 4 & 5\\ \hline
		38.7 &Stage 4 & 6\\ \hline
		47.0 &Stage 3 & 7\\ \hline		
		52.9  &Stage 4 & 8\\ \hline
		54.5 &Stage 3 & 9\\ \hline				
		60.0 &Stage 3 & 10\\ \hline
		95.8& Stage 3 & 11\\ \hline
		119.3 &Stage 3 & 12\\ \hline
	\end{tabular}
\end{center}
\end{frame}

\begin{frame}{Sampling distribution of the rank-sum}
	\begin{itemize}
		\item If the null hypothesis is true, that is the population distributions are identical between two groups, then:
		\begin{itemize}
			\item The individual ranks are a random sample from $N$ integers: $\{1, 2, \hdots, N\}$
			\item The sum of ranks, $T$, is a random variable that only depends on the sample sizes $n_1$ and $n_2$
			\item The sum of ranks, $T$, is a random variable that does not depend on the population distributions
			\item And we have mathematical formulas for the mean and variance of the distribution for the sum of ranks:
			\begin{gather*}
			\mu_T = \frac{n_1 (n_1 + n_2 + 1)}{2} \quad \quad \text{and} \quad \quad \sigma_T^{2} = \frac{n_1 n_2}{12}(n_1 +n_2 +1)
			\end{gather*}
		\end{itemize}
	\item Since $\mu_T$ is the expected value of $T$ when the null hypothesis is true (same population distribution), values of $T$ far from $\mu_T$ would be evidence they are not the same distribution
	\end{itemize}
\end{frame}

\begin{frame}{Calculating the rank-sum}{Example Data}
\begin{center}{\tiny
	\begin{tabular}{|c|c|c|}
		\hline
		\textbf{Survival Time (months)} &  \textbf{Stage} & \textbf{Rank}\\ \hline \hline
		12.9 &Stage 4 & 1 \\ \hline
		22.8 &Stage 4 & 2\\ \hline
		22.9& Stage 4 & 3\\ \hline
		33.2 &Stage 3 & 4\\ \hline
		34.9& Stage 4 & 5\\ \hline
		38.7 &Stage 4 & 6\\ \hline
		47.0 &Stage 3 & 7\\ \hline		
		52.9  &Stage 4 & 8\\ \hline
		54.5 &Stage 3 & 9\\ \hline				
		60.0 &Stage 3 & 10\\ \hline
		95.8& Stage 3 & 11\\ \hline
		119.3 &Stage 3 & 12\\ \hline
	\end{tabular}}
\end{center}
\begin{itemize}
	\item Sum: $T = 1 + 2 + 3 + 5 + 6 + 8 = 25$
	\item[]
	\item $\mu_T = \frac{n_1 (n_1 + n_2 + 1)}{2} = 6(6+6+1)/2 = 39$
	\item[]
	\item Is $T$ sufficiently different from $\mu_T$ to believe that the population distributions are different?
\end{itemize}
\end{frame}

\section{Exact test}
\begin{frame}{Outline}
\tableofcontents[currentsection,subsectionstyle=show/shaded/hide]
\end{frame}

\begin{frame}{Wilcoxon Rank-Sum Test}{Process}
	\begin{itemize}
		\item Null Hypothesis. $H_0$: The population distributions are identical $\Delta = 0$
		\item[]
		\item Alternative hypotheses (choose one). $H_a$:
			\begin{enumerate}
				\item Population 1 is shifted to the right of popluation 2 ($\Delta >0$)
				\item Population 1 is shifted to the left of population 2 ($\Delta < 0$)
				\item Population 1 and 2 are shifted from each other ($\Delta \neq 0$)
				\item[]
			\end{enumerate}
		\item Rejection regions / rejection rules:
		\begin{enumerate}
			\item Reject $H_0$ if $T > T_U$ (one-tailed)
			\item Reject $H_0$ if $T < T_L$ (one-tailed)
			\item Reject $H_0$ if $T > T_U$ or $T < T_L$ (two-tailed)
		\end{enumerate}
		\end{itemize}
\end{frame}

\begin{frame}{Wilcoxon Rank-Sum Test}{Critical values}
	\begin{center}
		\includegraphics{table}
	\end{center}
\end{frame}

\begin{frame}{Example Data}{Test setup}
\begin{center}{\tiny
		\begin{tabular}{|c|c|c|}
			\hline
			\textbf{Survival Time (months)} &  \textbf{Stage} & \textbf{Rank}\\ \hline \hline
			12.9 &Stage 4 & 1 \\ \hline
			22.8 &Stage 4 & 2\\ \hline
			22.9& Stage 4 & 3\\ \hline
			33.2 &Stage 3 & 4\\ \hline
			34.9& Stage 4 & 5\\ \hline
			38.7 &Stage 4 & 6\\ \hline
			47.0 &Stage 3 & 7\\ \hline		
			52.9  &Stage 4 & 8\\ \hline
			54.5 &Stage 3 & 9\\ \hline				
			60.0 &Stage 3 & 10\\ \hline
			95.8& Stage 3 & 11\\ \hline
			119.3 &Stage 3 & 12\\ \hline
	\end{tabular}}
\end{center}
\begin{itemize}
	\item $H_0$: The population distributions are identical $\Delta = 0$
	\item $H_a$: Population 1 (Stage 4 tumors) is shifted to the left of population 2 (Stage 3 tumors) ($\Delta < 0$)
	\item RR: reject $H_0$ if $T < T_L$ (one-tailed)
	\begin{itemize}
		\item From the table $T_L = 26$
		\item We reject the null hypothesis
	\end{itemize}
\end{itemize}
\end{frame}

\subsection{Point estimate and confidence intervals for $\Delta$}
\begin{frame}{Outline}
\tableofcontents[currentsection,subsectionstyle=show/shaded/hide]
\end{frame}

\begin{frame}{Point estimate for $\Delta$}
	\begin{itemize}
		\item Now we want to find an estimate and confidence interval for the population $\Delta$ using our sample
		\item[]
		\item Process:
		\begin{itemize}
			\item Compute $M = n_1 \times n_2$ possible differences: $x_i - y_j$ for $i = 1, 2, \hdots, n_1$ and $j = 1, 2, \hdots, n_2$
			\item[]
			\item We then order the set of differences: $D_{(1)} \leq D_{(2)} \leq \hdots \leq D_{(M)}$
			\item[]
			\item We then compute our estimate for $\Delta$ as the median of this set of $M$ differences
			\item[]
			\item Confidence intervals will also be based on this set of ordered differences
 		\end{itemize}
	\end{itemize}
\end{frame}

\begin{frame}{Point estimate for $\Delta$}{Set of Differences}
	\begin{center}
		\begin{tabular}{|c|c|c|c|c|c|}
			\hline
	\textbf{Pair \#} & $i$ & $j$ & \textbf{Pop 1 value} & \textbf{Pop 2 value} & \textbf{Difference} \\ \hline \hline
			1 &    1  &  1 &   52.9  &  60.0  &  -7.1\\ \hline
			2 &   2  &  1  &  12.9  &  60.0  & -47.1\\ \hline
			3  &  3  &  1  &  34.9 &   60.0  & -25.1\\ \hline
			4 &   4  &  1  &  38.7  &  60.0  & -21.3\\ \hline
			5 &   5  &  1  &  22.9  &  60.0  & -37.1\\ \hline
			6 &   6  &  1  &  22.8  &  60.0  & -37.2\\ \hline
			7 &   1  &  2  &  52.9  &  95.8  & -42.9\\ \hline
			8 &   2  &  2  &  12.9  &  95.8  & -82.9\\ \hline
			9 &   3  &  2  &  34.9  &  95.8  & -60.9\\ \hline
			10 &  4  &  2  &  38.7  &  95.8  & -57.1\\ \hline
			\vdots &\vdots &\vdots &\vdots & \vdots & \vdots \\ \hline
			35  &    5   &   6   &   22.9   &     47  &   -24.1 \\ \hline
			36   &   6   &   6   &   22.8   &     47   &  -24.2 \\ \hline
		\end{tabular}
	\end{center}
\end{frame}

\begin{frame}{Point estimate for $\Delta$}{Set of Differences}
	\begin{itemize}
		\item Set of ordered differences: 
		\begin{gather*}
			\{-106.4,  -96.5,  -96.4,  -84.4,  -82.9,  -80.6,  -73.0,  -72.9,  -66.4,\\ -60.9,  -57.1,  -47.1,  -42.9, -41.6,  -37.2,  -37.1,  -34.1, 
			-31.7, \\ -31.6,  -25.1,  -24.2,  -24.1,  -21.3,  -20.3,  -19.6,  -15.8,  -12.1, \\ -10.4   -10.3,   -8.3,   -7.1,   -1.6,    1.7,    5.5, 5.9,   19.7\}
		\end{gather*}
		\item Point estimate formula: $\hat{\Delta} = D_{((M+1)/2)}$ if $M$ is an odd number or $\hat{\Delta} = \frac{1}{2} (D_{(M/2)} + D_{(M/2+1)})$ if $M$ is an even number
		\item[]
		\item Point estimate: $\hat{\Delta} = \frac{1}{2} (D_{18} + D_{(19)}) = .5(-31.7 - 31.6) = -31.65$
		\begin{itemize}
			\item We estimate that survival times are 31.65 months less for Stage 4 tumors than stage 3 tumors
		\end{itemize}
	\end{itemize}
\end{frame}

\begin{frame}{Confidence intervals for $\Delta$}{Process}
\begin{itemize}
	\item To construct $100(1-\alpha) \%$ confidence intervals we will use:
	\begin{itemize}
		\item $\Delta_L = D_{(C_{\alpha / 2})}$
		\item $\Delta_U = D_{(M + 1 - C_{\alpha / 2})}$
		\item where:
		\begin{gather*}
		C_{\alpha / 2} = \frac{n_1 (2n_2 + n_1 + 1)}{2} + 1 - T_U
		\end{gather*}
				\item If $C_{\alpha / 2}$ is not an integer we round to the nearest integer
	\end{itemize}
	\item To construct a 95\% confidence interval ($\alpha = 0.05$) for $\Delta$, use $T_U$ corresponding to a $\alpha /2 = .025$ one-tailed test
\end{itemize}
\begin{center}
	\includegraphics[width=.5 \linewidth]{table}
\end{center}
\end{frame}

\begin{frame}{Confidence intervals for $\Delta$}{Example}
	\begin{itemize}
		\item First we find $C_{\alpha/2}$:
		\begin{align*}
		C_{0.025} &= \frac{n_1 (2n_2 + n_1 + 1)}{2} + 1 - T_U \\
		&= \frac{6 (2(6) + 6 + 1)}{2} + 1 - 52 \\
		&= 6
		\end{align*}
		\item Then we have:
		\begin{itemize}
			\item $\Delta_L = D_{(C_{\alpha / 2})} = D_{(6)} = -80.6$
			\item $\Delta_U = D_{(M + 1 - C_{\alpha / 2})} = D_{(36 +1 - 6)} =D_{(31)} = -7.1$
			\item[]
		\end{itemize}
	\item So a 95\% confidence interval for $\Delta$ is: $(-80.6, -7.1)$
	\end{itemize}
\end{frame}

\section{Normal approximation}

\begin{frame}{Outline}
\tableofcontents[currentsection,subsectionstyle=show/shaded/hide]
\end{frame}

\end{document}