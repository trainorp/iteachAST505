\documentclass[xcolor=dvipsnames]{beamer} 
\usetheme{AnnArbor}
\usecolortheme{beaver}

\usepackage{amsmath,graphicx,booktabs,tikz,subfig,color,lmodern}
\definecolor{mycol}{rgb}{.4,.85,1}
\setbeamercolor{title}{bg=mycol,fg=black} 
\setbeamercolor{palette primary}{use=structure,fg=white,bg=red}
\setbeamercolor{block title}{fg=white,bg=red!50!black}
% \setbeamercolor{block title}{fg=white,bg=blue!75!black}

\title[Lecture 11]{Lecture 11: Nonparametric inference}
\author[Patrick Trainor]{Patrick Trainor, PhD, MS, MA}
\institute[NMSU]{New Mexico State University}
\date{September 30, 2019}

\newcommand\myeq{\mathrel{\overset{\makebox[0pt]{\mbox{\normalfont\tiny\sffamily D}}}{=}}}

\begin{document}

\begin{frame}
\maketitle
\end{frame}

\begin{frame}{Outline}
\tableofcontents[hideallsubsections]
\end{frame}

\begin{frame}{Wilcoxon Rank-Sum Tests}
	\begin{itemize}
		\item One of the assumptions of t-tests is that the data be normally distributed. If the sample size is very large, this assumption is less important
		\item[]
		\item However, if our data distribution is far from normal, we want to use ``nonparametric tests'' rather than t-tests
		\item[]
		\item \textbf{\emph{Wilcoxon rank-sum test:}} The Wilcoxon rank-sum test is a nonparametric test that assumes the following:
		
		\begin{enumerate}
			\item We have two independent random samples with sample sizes $n_1$ and $n_2$: $\{x_1, x_2, \hdots, x_{n_1}\}$ and $\{y_1, y_2, \hdots, y_{n_2}\}$
			\item[]
			\item The population distributions of the $x$'s and $y$'s are identical but one may be shifted to the left or right of the other
		\end{enumerate}
	\end{itemize}
\end{frame}

\begin{frame}{Location shift}
	\begin{center}
		\includegraphics[width=.9 \linewidth]{locationShift}
	\end{center}
\end{frame}

\begin{frame}{Location shift}
	\begin{itemize}
		\item Notation for location shift: $y \myeq x + \Delta$
	\end{itemize}
\begin{center}
	\includegraphics[width=.85 \linewidth]{locationShift2}
\end{center}
\end{frame}

\begin{frame}{Location shift}
	\begin{itemize}
		\item Notation for location shift: $y \myeq x + \Delta$
		\item[]
		\item If $\Delta > 0$, then the $y$'s are typically larger and the distribution of $y$'s is to the right of $x$'s
		\item[]
		\item If $\Delta > 0$, then the $y$'s are typically smaller and the distribution of $y$'s is to the left of $x$'s
	\end{itemize}
\end{frame}

\begin{frame}{Wilcoxon Rank-Sum test}
	\begin{itemize}
		\item The Wilcoxon rank-sum test is based on ``ranks'' rather than measurement values
		\begin{itemize}
			\item The smallest value in the combined sample has rank 1
			\item[]
			\item The largest value in the combinded sample has rank $N = n_1 + n_2$
			\item[]
		\end{itemize}
	\item Process for determining Rank-Sum:
		\begin{enumerate}
		\item List the data values in the combined data set from smallest to largest
		\item In the next column assign numbers $1$ to $n$ to the data values by their order
		\item If there are ties in the combined data set, the ranks for the observations in the tie are taken to be the average of those ranks
		\item Let $T$ denote the sum of ranks for the observation from the $y$'s
	\end{enumerate}
	\end{itemize}
\end{frame}

\begin{frame}{Calculating the rank-sum}{Example Data}
\begin{itemize}
	\item Assume Stage 4 tumors are the $y$'s and Stage 3 tumors are the $x$'s
\end{itemize}
	\begin{center}
		\begin{tabular}{|c|c|}
			\hline
			    \textbf{Survival Time} &  \textbf{Stage} \\ \hline \hline
			52.9  &Stage 4 \\ \hline
			12.9 &Stage 4 \\ \hline
		    34.9& Stage 4 \\ \hline
		    38.7 &Stage 4 \\ \hline
			  22.9& Stage 4 \\ \hline
		   22.8 &Stage 4 \\ \hline
		   60.0 &Stage 3 \\ \hline
			  95.8& Stage 3 \\ \hline
		   33.2 &Stage 3 \\ \hline
			  54.5 &Stage 3 \\ \hline
			 119.3 &Stage 3 \\ \hline
			  47.0 &Stage 3 \\ \hline
		\end{tabular}
	\end{center}
\end{frame}

\begin{frame}{Calculating the rank-sum}{Example Data}
\begin{itemize}
	\item Our first step is to order the data from smallest to largest
\end{itemize}
\begin{center}
	\begin{tabular}{|c|c|}
		\hline
		\textbf{Survival Time} &  \textbf{Stage} \\ \hline \hline
		12.9 &Stage 4 \\ \hline
		22.8 &Stage 4 \\ \hline
		22.9& Stage 4 \\ \hline
		33.2 &Stage 3 \\ \hline
		34.9& Stage 4 \\ \hline
		38.7 &Stage 4 \\ \hline
		47.0 &Stage 3 \\ \hline		
		52.9  &Stage 4 \\ \hline
		54.5 &Stage 3 \\ \hline				
		60.0 &Stage 3 \\ \hline
		95.8& Stage 3 \\ \hline
		119.3 &Stage 3 \\ \hline
	\end{tabular}
\end{center}
\end{frame}

\begin{frame}{Calculating the rank-sum}{Example Data}
\begin{itemize}
	\item Now we assign the ranks
\end{itemize}
\begin{center}
	\begin{tabular}{|c|c|c|}
		\hline
		\textbf{Survival Time} &  \textbf{Stage} & \textbf{Rank}\\ \hline \hline
		12.9 &Stage 4 & 1 \\ \hline
		22.8 &Stage 4 & 2\\ \hline
		22.9& Stage 4 & 3\\ \hline
		33.2 &Stage 3 & 4\\ \hline
		34.9& Stage 4 & 5\\ \hline
		38.7 &Stage 4 & 6\\ \hline
		47.0 &Stage 3 & 7\\ \hline		
		52.9  &Stage 4 & 8\\ \hline
		54.5 &Stage 3 & 9\\ \hline				
		60.0 &Stage 3 & 10\\ \hline
		95.8& Stage 3 & 11\\ \hline
		119.3 &Stage 3 & 12\\ \hline
	\end{tabular}
\end{center}
\end{frame}

\begin{frame}{Calculating the rank-sum}{Example Data}
\begin{itemize}
	\item Sum: $T = 1 + 2 + 3 + 5 + 6 + 8 = 25$
\end{itemize}
\begin{center}
	\begin{tabular}{|c|c|c|}
		\hline
		\textbf{Survival Time} &  \textbf{Stage} & \textbf{Rank}\\ \hline \hline
		12.9 &Stage 4 & 1 \\ \hline
		22.8 &Stage 4 & 2\\ \hline
		22.9& Stage 4 & 3\\ \hline
		33.2 &Stage 3 & 4\\ \hline
		34.9& Stage 4 & 5\\ \hline
		38.7 &Stage 4 & 6\\ \hline
		47.0 &Stage 3 & 7\\ \hline		
		52.9  &Stage 4 & 8\\ \hline
		54.5 &Stage 3 & 9\\ \hline				
		60.0 &Stage 3 & 10\\ \hline
		95.8& Stage 3 & 11\\ \hline
		119.3 &Stage 3 & 12\\ \hline
	\end{tabular}
\end{center}
\end{frame}

\end{document}