\documentclass[xcolor=dvipsnames]{beamer} 
\usetheme{AnnArbor}
\usecolortheme{beaver}

\usepackage{amsmath,graphicx,booktabs,tikz,subfig,color,lmodern}
\definecolor{mycol}{rgb}{.4,.85,1}
\setbeamercolor{title}{bg=mycol,fg=black} 
\setbeamercolor{palette primary}{use=structure,fg=white,bg=red}
\setbeamercolor{block title}{fg=white,bg=red!50!black}
% \setbeamercolor{block title}{fg=white,bg=blue!75!black}

\title[Lecture 11]{Lecture 11: Nonparametric inference}
\author[Patrick Trainor]{Patrick Trainor, PhD, MS, MA}
\institute[NMSU]{New Mexico State University}
\date{October 7, 2019}

\newcommand\myeq{\mathrel{\overset{\makebox[0pt]{\mbox{\normalfont\tiny\sffamily D}}}{=}}}

\begin{document}

\begin{frame}
\maketitle
\end{frame}

\begin{frame}{Outline}
\tableofcontents[hideallsubsections]
\end{frame}

\section{Introduction}
\begin{frame}{Outline}
\tableofcontents[currentsection,subsectionstyle=show/shaded/hide]
\end{frame}

\begin{frame}{Nonparametric tests}
	\begin{itemize}
		\item One of the assumptions of t-tests (for comparing population means) is that the data be normally distributed. If the sample size is very large, this assumption is less important \pause
		\item[]
		\begin{itemize}
			\item Even if the sample size is large so that a t-test is appropriate for comparing population means, the mean may not be the best measure of central tendency \pause
			\item[]
		\end{itemize}
		\item If our data distribution is far from normal, we want to use ``nonparametric tests'' rather than t-tests 
	\end{itemize}
\end{frame}

\begin{frame}{Wilcoxon Rank-Sum Tests}
	\begin{itemize}
		\item \textbf{\emph{Wilcoxon rank-sum test:}} The Wilcoxon rank-sum test is a nonparametric test that assumes the following:  \pause
		
		\begin{enumerate}
			\item We have two independent random samples with sample sizes $n_1$ and $n_2$: $\{x_1, x_2, \hdots, x_{n_1}\}$ and $\{y_1, y_2, \hdots, y_{n_2}\}$ \pause
			\item[]
			\item The population distributions of the $x$'s and $y$'s are identical but one may be shifted to the left or right of the other
		\end{enumerate}
	\end{itemize}
\end{frame}

\begin{frame}{Location shift}
	\begin{center}
		\includegraphics[width=.9 \linewidth]{locationShift}
	\end{center}
\end{frame}

\begin{frame}{Location shift}
	\begin{itemize}
		\item Notation for location shift: $y \myeq x + \Delta$
	\end{itemize}
\begin{center}
	\includegraphics[width=.85 \linewidth]{locationShift2}
\end{center}
\end{frame}

\begin{frame}{Location shift}
	\begin{itemize}
		\item Notation for location shift: $y \myeq x + \Delta$ \pause
		\item[]
		\item If $\Delta > 0$, then the $y$'s are typically larger and the distribution of $y$'s is to the right of $x$'s \pause
		\item[]
		\item If $\Delta > 0$, then the $y$'s are typically smaller and the distribution of $y$'s is to the left of $x$'s
	\end{itemize}
\end{frame}

\begin{frame}{Wilcoxon Rank-Sum test}
	\begin{itemize}
		\item The Wilcoxon rank-sum test is based on ``ranks'' rather than measurement values \pause
		\begin{itemize}
			\item The smallest value in the combined sample has rank 1 \pause
			\item[]
			\item The largest value in the combinded sample has rank $N = n_1 + n_2$
			\item[]
		\end{itemize}
	\item Process for determining Rank-Sum:
		\begin{enumerate}
		\item List the data values in the combined data set from smallest to largest \pause
		\item In the next column assign numbers $1$ to $N$ to the data values by their order \pause
		\item If there are ties in the combined data set, the ranks for the observations in the tie are taken to be the average of those ranks \pause
		\item Let $T$ denote the sum of ranks for the observation from the $y$'s
	\end{enumerate}
	\end{itemize}
\end{frame}

\begin{frame}{Example Data}
\begin{itemize}
	\item Research question: Are survival times worse (lower) for Stage 4 tumors than Stage 3 tumors? 
\end{itemize}
\begin{center}
	\begin{tabular}{|c|c|}
		\hline
		\textbf{Survival Time (months)} &  \textbf{Stage} \\ \hline \hline
		52.9  &Stage 4 \\ \hline
		12.9 &Stage 4 \\ \hline
		34.9& Stage 4 \\ \hline
		38.7 &Stage 4 \\ \hline
		22.9& Stage 4 \\ \hline
		22.8 &Stage 4 \\ \hline
		60.0 &Stage 3 \\ \hline
		95.8& Stage 3 \\ \hline
		33.2 &Stage 3 \\ \hline
		54.5 &Stage 3 \\ \hline
		119.3 &Stage 3 \\ \hline
		47.0 &Stage 3 \\ \hline
	\end{tabular}
\end{center}
\end{frame}

\begin{frame}{Example Data}
	\begin{center}
		\includegraphics[width=.9 \linewidth]{boxplot}
	\end{center}
\end{frame}

\begin{frame}{Example Data}
\begin{itemize}
	\item Assume Stage 4 tumors are the $y$'s and Stage 3 tumors are the $x$'s \pause
	\item Then we want to know if: $y \myeq x + \Delta$, where $\Delta < 0$ \pause
\end{itemize}
\begin{center}
	\begin{tabular}{|c|c|}
		\hline
		\textbf{Survival Time (months)} &  \textbf{Stage} \\ \hline \hline
		52.9  &Stage 4 \\ \hline
		12.9 &Stage 4 \\ \hline
		34.9& Stage 4 \\ \hline
		38.7 &Stage 4 \\ \hline
		22.9& Stage 4 \\ \hline
		22.8 &Stage 4 \\ \hline
		60.0 &Stage 3 \\ \hline
		95.8& Stage 3 \\ \hline
		33.2 &Stage 3 \\ \hline
		54.5 &Stage 3 \\ \hline
		119.3 &Stage 3 \\ \hline
		47.0 &Stage 3 \\ \hline
	\end{tabular}
\end{center}
\end{frame}

\begin{frame}{Calculating the rank-sum}{Example Data}
\begin{itemize}
	\item Our first step is to order the data from smallest to largest
\end{itemize}
\begin{center}
	\begin{tabular}{|c|c|}
		\hline
		\textbf{Survival Time (months)} &  \textbf{Stage} \\ \hline \hline
		12.9 &Stage 4 \\ \hline
		22.8 &Stage 4 \\ \hline
		22.9& Stage 4 \\ \hline
		33.2 &Stage 3 \\ \hline
		34.9& Stage 4 \\ \hline
		38.7 &Stage 4 \\ \hline
		47.0 &Stage 3 \\ \hline		
		52.9  &Stage 4 \\ \hline
		54.5 &Stage 3 \\ \hline				
		60.0 &Stage 3 \\ \hline
		95.8& Stage 3 \\ \hline
		119.3 &Stage 3 \\ \hline
	\end{tabular}
\end{center}
\end{frame}

\begin{frame}{Calculating the rank-sum}{Example Data}
\begin{itemize}
	\item Now we assign the ranks
\end{itemize}
\begin{center}
	\begin{tabular}{|c|c|c|}
		\hline
		\textbf{Survival Time (months)} &  \textbf{Stage} & \textbf{Rank}\\ \hline \hline
		12.9 &Stage 4 & 1 \\ \hline
		22.8 &Stage 4 & 2\\ \hline
		22.9& Stage 4 & 3\\ \hline
		33.2 &Stage 3 & 4\\ \hline
		34.9& Stage 4 & 5\\ \hline
		38.7 &Stage 4 & 6\\ \hline
		47.0 &Stage 3 & 7\\ \hline		
		52.9  &Stage 4 & 8\\ \hline
		54.5 &Stage 3 & 9\\ \hline				
		60.0 &Stage 3 & 10\\ \hline
		95.8& Stage 3 & 11\\ \hline
		119.3 &Stage 3 & 12\\ \hline
	\end{tabular}
\end{center}
\end{frame}

\begin{frame}{Calculating the rank-sum}{Example Data}
\begin{itemize}
	\item Sum: $T = 1 + 2 + 3 + 5 + 6 + 8 = 25$
\end{itemize}
\begin{center}
	\begin{tabular}{|c|c|c|}
		\hline
		\textbf{Survival Time (monts)} &  \textbf{Stage} & \textbf{Rank}\\ \hline \hline
		12.9 &Stage 4 & 1 \\ \hline
		22.8 &Stage 4 & 2\\ \hline
		22.9& Stage 4 & 3\\ \hline
		33.2 &Stage 3 & 4\\ \hline
		34.9& Stage 4 & 5\\ \hline
		38.7 &Stage 4 & 6\\ \hline
		47.0 &Stage 3 & 7\\ \hline		
		52.9  &Stage 4 & 8\\ \hline
		54.5 &Stage 3 & 9\\ \hline				
		60.0 &Stage 3 & 10\\ \hline
		95.8& Stage 3 & 11\\ \hline
		119.3 &Stage 3 & 12\\ \hline
	\end{tabular}
\end{center}
\end{frame}

\begin{frame}{Sampling distribution of the rank-sum}
	\begin{itemize}
		\item If the null hypothesis is true, that is the population distributions are identical between two groups, then: \pause
		\begin{itemize}
			\item The individual ranks are a random sample from $N$ integers: $\{1, 2, \hdots, N\}$ \pause
			\item The sum of ranks, $T$, is a random variable that only depends on the sample sizes $n_1$ and $n_2$ \pause
			\item The sum of ranks, $T$, is a random variable that does not depend on the population distributions \pause
			\item And we have mathematical formulas for the mean and variance of the distribution for the sum of ranks: \pause
			\begin{gather*}
			\mu_T = \frac{n_1 (n_1 + n_2 + 1)}{2} \quad \quad \text{and} \quad \quad \sigma_T^{2} = \frac{n_1 n_2}{12}(n_1 +n_2 +1)
			\end{gather*}\pause
		\end{itemize}
	\item Since $\mu_T$ is the expected value of $T$ when the null hypothesis is true (same population distribution), values of $T$ far from $\mu_T$ would be evidence they are not the same distribution
	\end{itemize}
\end{frame}

\begin{frame}{Calculating the rank-sum}{Example Data}
\begin{center}{\tiny
	\begin{tabular}{|c|c|c|}
		\hline
		\textbf{Survival Time (months)} &  \textbf{Stage} & \textbf{Rank}\\ \hline \hline
		12.9 &Stage 4 & 1 \\ \hline
		22.8 &Stage 4 & 2\\ \hline
		22.9& Stage 4 & 3\\ \hline
		33.2 &Stage 3 & 4\\ \hline
		34.9& Stage 4 & 5\\ \hline
		38.7 &Stage 4 & 6\\ \hline
		47.0 &Stage 3 & 7\\ \hline		
		52.9  &Stage 4 & 8\\ \hline
		54.5 &Stage 3 & 9\\ \hline				
		60.0 &Stage 3 & 10\\ \hline
		95.8& Stage 3 & 11\\ \hline
		119.3 &Stage 3 & 12\\ \hline
	\end{tabular}}
\end{center}
\begin{itemize}
	\item Sum: $T = 1 + 2 + 3 + 5 + 6 + 8 = 25$ \pause
	\item[]
	\item $\mu_T = \frac{n_1 (n_1 + n_2 + 1)}{2} = 6(6+6+1)/2 = 39$ \pause
	\item[]
	\item Is $T$ sufficiently different from $\mu_T$ to believe that the population distributions are different?
\end{itemize}
\end{frame}

\section{Wilcoxon Rank-Sum test (small $n$)}

\subsection{Hypothesis Tests}
\begin{frame}{Outline}
\tableofcontents[currentsection,subsectionstyle=show/shaded/hide]
\end{frame}

\begin{frame}{Wilcoxon Rank-Sum Test}{Process}
	\begin{itemize}
		\item Null Hypothesis. $H_0$: The population distributions are identical $\Delta = 0$ \pause
		\item[]
		\item Alternative hypotheses (choose one). Recall $y \myeq x + \Delta$. $H_a$:\pause
			\begin{enumerate}
				\item Population 1 is shifted to the right of population 2 ($\Delta >0$)\pause
				\item Population 1 is shifted to the left of population 2 ($\Delta < 0$)\pause
				\item Population 1 and 2 are shifted from each other ($\Delta \neq 0$)\pause
				\item[]
			\end{enumerate}
		\item Rejection regions / rejection rules:
		\begin{enumerate}
			\item Reject $H_0$ if $T > T_U$ (one-tailed) \pause
			\item Reject $H_0$ if $T < T_L$ (one-tailed) \pause
			\item Reject $H_0$ if $T > T_U$ or $T < T_L$ (two-tailed)
		\end{enumerate}
		\end{itemize}
\end{frame}

\begin{frame}{Wilcoxon Rank-Sum Test}{Critical values}
	\begin{center}
		\includegraphics{table}
	\end{center}
\end{frame}

\begin{frame}{Example Data}{Test setup}
\begin{center}{\tiny
		\begin{tabular}{|c|c|c|}
			\hline
			\textbf{Survival Time (months)} &  \textbf{Stage} & \textbf{Rank}\\ \hline \hline
			12.9 &Stage 4 & 1 \\ \hline
			22.8 &Stage 4 & 2\\ \hline
			22.9& Stage 4 & 3\\ \hline
			33.2 &Stage 3 & 4\\ \hline
			34.9& Stage 4 & 5\\ \hline
			38.7 &Stage 4 & 6\\ \hline
			47.0 &Stage 3 & 7\\ \hline		
			52.9  &Stage 4 & 8\\ \hline
			54.5 &Stage 3 & 9\\ \hline				
			60.0 &Stage 3 & 10\\ \hline
			95.8& Stage 3 & 11\\ \hline
			119.3 &Stage 3 & 12\\ \hline
	\end{tabular}}
\end{center}
\begin{itemize}
	\item $H_0$: The population distributions are identical $\Delta = 0$ \pause
	\item $H_a$: Population 1 (Stage 4 tumors) is shifted to the left of population 2 (Stage 3 tumors) ($\Delta < 0$) \pause
	\item RR: reject $H_0$ if $T < T_L$ (one-tailed) \pause
	\begin{itemize}
		\item From the table $T_L = 26$ \pause
		\item We reject the null hypothesis
	\end{itemize}
\end{itemize}
\end{frame}

\subsection{Point estimate and confidence intervals for $\Delta$}
\begin{frame}{Outline}
\tableofcontents[currentsection,subsectionstyle=show/shaded/hide]
\end{frame}

\begin{frame}{Point estimate for $\Delta$}
	\begin{itemize}
		\item Now we want to find an estimate and confidence interval for the population $\Delta$ using our sample \pause
		\item[]
		\item Process:
		\begin{itemize}
			\item Compute $M = n_1 \times n_2$ possible differences: $x_i - y_j$ for $i = 1, 2, \hdots, n_1$ and $j = 1, 2, \hdots, n_2$ \pause
			\item[]
			\item We then order the set of differences: $D_{(1)} \leq D_{(2)} \leq \hdots \leq D_{(M)}$ \pause
			\item[]
			\item We then compute our estimate for $\Delta$ as the median of this set of $M$ differences \pause
			\item[]
			\item Confidence intervals will also be based on this set of ordered differences
 		\end{itemize}
	\end{itemize}
\end{frame}

\begin{frame}{Point estimate for $\Delta$}{Set of Differences}
	\begin{center}
		\begin{tabular}{|c|c|c|c|c|c|}
			\hline
	\textbf{Pair \#} & $i$ & $j$ & \textbf{Pop 1 value} & \textbf{Pop 2 value} & \textbf{Difference} \\ \hline \hline
			1 &    1  &  1 &   52.9  &  60.0  &  -7.1\\ \hline
			2 &   2  &  1  &  12.9  &  60.0  & -47.1\\ \hline
			3  &  3  &  1  &  34.9 &   60.0  & -25.1\\ \hline
			4 &   4  &  1  &  38.7  &  60.0  & -21.3\\ \hline
			5 &   5  &  1  &  22.9  &  60.0  & -37.1\\ \hline
			6 &   6  &  1  &  22.8  &  60.0  & -37.2\\ \hline
			7 &   1  &  2  &  52.9  &  95.8  & -42.9\\ \hline
			8 &   2  &  2  &  12.9  &  95.8  & -82.9\\ \hline
			9 &   3  &  2  &  34.9  &  95.8  & -60.9\\ \hline
			10 &  4  &  2  &  38.7  &  95.8  & -57.1\\ \hline
			\vdots &\vdots &\vdots &\vdots & \vdots & \vdots \\ \hline
			35  &    5   &   6   &   22.9   &     47  &   -24.1 \\ \hline
			36   &   6   &   6   &   22.8   &     47   &  -24.2 \\ \hline
		\end{tabular}
	\end{center}
\end{frame}

\begin{frame}{Point estimate for $\Delta$}{Set of Differences}
	\begin{itemize}
		\item Set of ordered differences: \pause
		\begin{gather*}
			\{-106.4,  -96.5,  -96.4,  -84.4,  -82.9,  -80.6,  -73.0,  -72.9,  -66.4,\\ -60.9,  -57.1,  -47.1,  -42.9, -41.6,  -37.2,  -37.1,  -34.1, 
			-31.7, \\ -31.6,  -25.1,  -24.2,  -24.1,  -21.3,  -20.3,  -19.6,  -15.8,  -12.1, \\ -10.4   -10.3,   -8.3,   -7.1,   -1.6,    1.7,    5.5, 5.9,   19.7\}
		\end{gather*} \pause
		\item Point estimate formula: $\hat{\Delta} = D_{((M+1)/2)}$ if $M$ is an odd number or $\hat{\Delta} = \frac{1}{2} (D_{(M/2)} + D_{(M/2+1)})$ if $M$ is an even number \pause
		\item[]
		\item Point estimate: $\hat{\Delta} = \frac{1}{2} (D_{18} + D_{(19)}) = .5(-31.7 - 31.6) = -31.65$
		\begin{itemize}
			\item We estimate that survival times are 31.65 months less for Stage 4 tumors than stage 3 tumors
		\end{itemize}
	\end{itemize}
\end{frame}

\begin{frame}{Confidence intervals for $\Delta$}{Process}
\begin{itemize}
	\item To construct $100(1-\alpha) \%$ confidence intervals we will use: \pause
	\begin{itemize}
		\item $\Delta_L = D_{(C_{\alpha / 2})}$ \pause
		\item $\Delta_U = D_{(M + 1 - C_{\alpha / 2})}$ \pause
		where:
		\begin{gather*}
		C_{\alpha / 2} = \frac{n_1 (2n_2 + n_1 + 1)}{2} + 1 - T_U
		\end{gather*}
				\item If $C_{\alpha / 2}$ is not an integer we round to the nearest integer \pause
	\end{itemize}
	\item To construct a 95\% confidence interval ($\alpha = 0.05$) for $\Delta$, use $T_U$ corresponding to a $\alpha /2 = .025$ one-tailed test
\end{itemize}
\begin{center}
	\includegraphics[width=.5 \linewidth]{table}
\end{center}
\end{frame}

\begin{frame}{Confidence intervals for $\Delta$}{Example}
	\begin{itemize}
		\item First we find $C_{\alpha/2}$: \pause
		\begin{align*}
		C_{0.025} &= \frac{n_1 (2n_2 + n_1 + 1)}{2} + 1 - T_U \\
		&= \frac{6 (2(6) + 6 + 1)}{2} + 1 - 52 \\
		&= 6
		\end{align*}
		\item Then we have: \pause
		\begin{itemize}
			\item $\Delta_L = D_{(C_{\alpha / 2})} = D_{(6)} = -80.6$ \pause
			\item $\Delta_U = D_{(M + 1 - C_{\alpha / 2})} = D_{(36 +1 - 6)} =D_{(31)} = -7.1$ \pause
			\item[]
		\end{itemize}
	\item So a 95\% confidence interval for $\Delta$ is: $(-80.6, -7.1)$
	\end{itemize}
\end{frame}

\section{Wilcoxon Rank-Sum test (Normal approximation)}
\subsection{Hypothesis Tests}
\begin{frame}{Outline}
\tableofcontents[currentsection,subsectionstyle=show/shaded/hide]
\end{frame}

\begin{frame}{Normal approximation}
	\begin{itemize}
		\item If both $n_1$ and $n_2$ are large (book says $>10$ each), then the sampling distribution of $T$ is approximately normal \pause
		\item[]
		\item So we can compute a $z$ statistic: \pause
		\begin{gather*}
		z = \frac{T- \mu_T}{\sigma_T} 
		\end{gather*}
		where:
		\begin{gather*}
		\mu_T = \frac{n_1 (n_1 + n_2 + 1)}{2} \quad \quad \text{and} \quad \quad \sigma_T^{2} = \frac{n_1 n_2}{12}(n_1 +n_2 +1)
		\end{gather*} \pause
		and we can use a normal approximation for hypothesis tests and confidence intervals
	\end{itemize}
\end{frame}

\begin{frame}{Normal approximation}{Presence of ties}
\begin{itemize}
	\item However, if ties are present in our data (and we have averaged ranks), we need to adjust our formula for the variance of $T$: \pause
	\begin{gather*}
	\sigma_T^{2} = \frac{n_1 n_2}{12}(n_1 +n_2 +1)
	\end{gather*}
	becomes: \pause
	\begin{gather*}
	\sigma_T^{2} = \frac{n_1 n_2}{12}\left[(n_1 +n_2 +1)- \frac{\sum_{j=1}^k t_j (t_j^2-1)}{(n_1+n_2)(n_1+n_2-1)}\right]
	\end{gather*}
	where $k$ is the number of tied groups and $t_j$ is the number of tied observation in each $j$th group
\end{itemize}
\end{frame}

\begin{frame}{Example Data}
	\begin{itemize}
		\item Phycocyanin, a pigment in cyanobacteria measured by fluorometry \pause
		\item The USGS measures this to study water quality and the presence of harmful blooms  \pause
		\item Rough River Lake, KY:
		\begin{center}
			\includegraphics[width = .7\linewidth]{RoughRiver} \\
			{\tiny Source: Patrick Jennings (shutterstock)}
		\end{center}
	\end{itemize}
\end{frame}

\begin{frame}{Example Data}
\begin{itemize}
	\item Assume the USGS takes 100 samples from different locations around the lake in June and August
\end{itemize}
	\begin{center}
		\includegraphics[width = .7 \linewidth]{RoughHist}
	\end{center}
\end{frame}

\begin{frame}{Example Data}
\begin{itemize}
	\item Research question: Does the level of phycocyanin differ between June and August?
\end{itemize}
\begin{center}
	\includegraphics[width = .7 \linewidth]{RoughHist}
\end{center}
\end{frame}

\begin{frame}{Normal approximation}{Process}
\begin{itemize}
	\item Null Hypothesis. $H_0$: The population distributions are identical $\Delta = 0$ \pause
	\item[]
	\item Alternative hypotheses (choose one). $H_a$: \pause
	\begin{enumerate}
		\item Population 1 is shifted to the right of population 2 ($\Delta >0$) 
		\item Population 1 is shifted to the left of population 2 ($\Delta < 0$) 
		\item Population 1 and 2 are shifted from each other ($\Delta \neq 0$)  \pause
		\item[]
	\end{enumerate}
	\item Test statistic: $z^* =\frac{T- \mu_T}{\sigma_T}$ where $T$ is the sum of ranks in sample 1 \pause
	\item[]
	\item Rejection regions / rejection rules: \pause
	\begin{enumerate}
		\item Reject $H_0$ if $z^* \geq z_{\alpha}$ with level of significance: $p\text{-value}=P(z \geq z^*)$
		\item Reject $H_0$ if $z^* \leq -z_{\alpha}$ with $p\text{-value}=P(z \leq z^*)$
		\item Reject $H_0$ if $|z^*| \geq z_{\alpha/2}$ with $p\text{-value}=2 \times P(z \geq |z^*|)$
	\end{enumerate}
\end{itemize}
\end{frame}

\begin{frame}{Example Problem}
\begin{itemize}
	\item Research question: Does the level of phycocyanin differ between June and August? \pause
	\item[]
	\item Null hypothesis. $H_0$: The two populations (June and August) are identical ($\Delta = 0$) \pause
	\item[]
	\item Alternative hypothesis. $H_a$: Population 1 (June) and Population 2 (August) are shifted from each other ($\Delta \neq 0$) \pause
	\item[]
	\item Let's test the alternative hypothesis against the null at $\alpha = 0.05$
\end{itemize}
\end{frame}

\begin{frame}{June Data}
	\begin{center}
		\includegraphics[width=.475\linewidth]{data1}
	\end{center}
\end{frame}

\begin{frame}{August Data}
	\begin{center}
		\includegraphics[width=.475\linewidth]{data2}
	\end{center}
\end{frame}

\begin{frame}{Normal approximation}{Example}
	\begin{itemize}
		\item From the phycocyanin data we have that $T = 8287$ and the following are ties: 
		\begin{center}
			\begin{tabular}{|c|c|c|}
					\hline
				  \textbf{value} & \textbf{month}&  \textbf{rank} \\ \hline \hline
				 145.7  & June &  30.5\\ \hline
				 145.7  & June &  30.5\\ \hline
				 196.3  & June &  63.5\\ \hline
				 196.3 &August &  63.5\\ \hline
				 354.5 &  June & 139.5\\ \hline
				 354.5& August & 139.5\\ \hline
			\end{tabular}
		\end{center} \pause
	\item Now to calculate our test statistic we need to find $\mu_T$ and $\sigma_T^2$:
	\begin{align*}
		\mu_T = \frac{n_1 (n_1 + n_2 + 1)}{2} = \frac{100(100+100+1)}{2} = 10050
	\end{align*}
	\end{itemize}
\end{frame}

\begin{frame}{Normal approximation}{Sampling distribution variance calculation example}
\begin{itemize}
	\item Formula: 	
	\begin{gather*}
	\sigma_T^{2} = \frac{n_1 n_2}{12}(n_1 +n_2 +1)- \frac{\sum_{j=1}^k t_j (t_j^2-1)}{(n_1+n_2)(n_1+n_2-1)}
	\end{gather*}
	where $k$ is the number of tied groups and $t_j$ is the number of tied observation in each $j$th group \pause
	\item[]
	\item We have three groups of ties with 2 observations in each group so:
	\begin{gather*}
		\sum_{j=1}^k t_j (t_j^2-1) =  [2(2^2 -1)] + [2(2^2 -1)] + [2(2^2 -1)] = 18
	\end{gather*}
\end{itemize}
\end{frame}

\begin{frame}{Normal approximation}{Sampling distribution variance calculation example}
\begin{itemize}
	\item Formula for $\sigma_T^{2}$: 	
	\begin{align*}
	\sigma_T^{2} &= \frac{n_1 n_2}{12}\left[(n_1 +n_2 +1)- \frac{\sum_{j=1}^k t_j (t_j^2-1)}{(n_1+n_2)(n_1+n_2-1)}\right] \\
	&= \frac{n_1 n_2}{12}\left[(n_1 +n_2 +1)- \frac{18}{(n_1+n_2)(n_1+n_2-1)}\right] \\
	&= \frac{(100)(100)}{12}\left[(100 + 100 +1)- \frac{18}{(100+100)(100+100-1)}\right] \\
	&= 167500 - 0.3768844 \\
	&= 167499.6
	\end{align*} \pause
	\item Also $\sigma_T = \sqrt{\sigma_T^2} = 409.267$
\end{itemize}
\end{frame}

\begin{frame}{Normal approximation}{Example}
	\begin{itemize}
		\item Recap: we have $T = 8287$, $\mu_T = 10050$ and $\sigma_T = 409.267$ \pause
		\item[]
		\item So our test statistic is:
		\begin{align*}
			z^* =\frac{T- \mu_T}{\sigma_T} = \frac{8287-10050}{409.267} = -4.31
		\end{align*} \pause
		\item Since $z_{\alpha/2} = z_{0.025} = 1.96$ and $|-4.31| = 4.31 > 1.96$ we reject the null hypothesis \pause
		\item[]
		\item The level of significance of this test is $p\text{-value}=2 \times P(z \geq |z^*|) = 2 \times P(z \geq 4.31) = 0.00001633$
	\end{itemize}
\end{frame}

\subsection{Confidence intervals for $\Delta$}
\begin{frame}{Outline}
\tableofcontents[currentsection,subsectionstyle=show/shaded/hide]
\end{frame}

\begin{frame}{Confidence intervals for $\Delta$}{Normal approximation}
	\begin{itemize}
		\item For large $n_1$ and $n_2$ we can approximate $C_{\alpha / 2}$ as:
		\begin{gather*}
		C_{\alpha / 2} = \frac{n_1 n_2}{2} - z_{\alpha / 2} \sqrt{\frac{n_1 n_2(n_1+n_2+1)}{12}}
		\end{gather*} \pause
		\item To construct $100(1-\alpha) \%$ confidence intervals we will use:
		\begin{itemize}
			\item $\Delta_L = D_{(C_{\alpha / 2})}$
			\item $\Delta_U = D_{(M + 1 - C_{\alpha / 2})}$
			\item[]
		\end{itemize} \pause
	\item Back to our phycocyanin data, let's calculate a 90\% confidence interval for $\Delta$
	\end{itemize}
\end{frame}

\begin{frame}{Confidence intervals for $\Delta$}{Normal approximation example}
\begin{itemize}
	\item For large $n_1$ and $n_2$ we can approximate $C_{\alpha / 2}$ as: \pause
	\begin{align*}
	C_{0.10 / 2} &= \frac{n_1 n_2}{2} - z_{0.05} \sqrt{\frac{n_1 n_2(n_1+n_2+1)}{12}} \\
	&=\frac{100*100}{2} - 1.65 \sqrt{\frac{100*100(100+100+1)}{12}} \\
	&= 4326.815 \\
	&\approx 4327 \pause
	\end{align*}
	\item $(\Delta_L, \Delta_U) = \left(D_{(C_{\alpha / 2})}, D_{(M + 1 - C_{\alpha / 2})} \right) = \left(D_{(4327)}, D_{(5674)}\right) = (-110.2, -51.4)$ \pause
	\item The point estimate will be the medians of the ordered set of differences: $-80.6$
\end{itemize}
\end{frame}

\section{Wilcoxon Signed-Rank test (Paired data)}
\begin{frame}{Outline}
\tableofcontents[currentsection,subsectionstyle=show/shaded/hide]
\end{frame}

\begin{frame}{Wilcoxon Signed-Rank Test}
	\begin{itemize}
		\item When the population distributions underlying some paired data are non-normal we use a Wilcoxon Signed-Rank test rather than the paired samples t-test \pause
		\item[]
		\item Assumptions: \pause
		\begin{itemize}
			\item We assume that the population distributions of differences (between pairs) is symmetric about some unknown median $M$ \pause
			\item[]
			\item We want to know if the distribution of differences is shifted left or right from some specific value $D_0$ (in most cases $D_0 = 0$) \pause
			\item[]
		\end{itemize}
	\end{itemize}
\end{frame}

\begin{frame}{Example data}
\begin{itemize}
			\item Example: An exercise scientist wants to know if a new swimming program improves athletes' time in swimming a 200M freestyle. The distribution of times is non-normal. He has 10 swimmers take a baseline test and a follow-up test after 4 weeks of training
\end{itemize}
	\begin{center}
		\begin{tabular}{|c|c|c|c|}
			\hline
			  \textbf{Athlete} &  \textbf{Baseline} &      \textbf{Post}    & \textbf{Difference} \\ \hline 
			  \hline
1  &     119.0 &  118.3 &  -0.7\\ \hline
2   &    145.6 &  139.6 &  -6.0\\ \hline
3    &    97.7 &   95.9 &  -1.8\\ \hline
4    &   123.4 &  116.0 &  -7.4\\ \hline
5    &   103.0 &  101.3 &  -1.7\\ \hline
6    &   116.8 &  110.1 &  -6.7\\ \hline
7     &  135.7 &  134.7 &  -1.0\\ \hline
8     &  127.8 &  128.1 &   0.2\\ \hline
9     &  124.3 &  125.4 &   1.1\\ \hline
10    &  113.5 &  112.0 &  -1.5\\ \hline
		\end{tabular}
	\end{center}
\end{frame}

\begin{frame}{Example data}
	\begin{center}
		\includegraphics[width=.8 \linewidth]{scatter}
	\end{center}
\end{frame}

\begin{frame}{Example data}
\begin{center}
	\includegraphics[width=.85 \linewidth]{timecourse}
\end{center}
\end{frame}

\begin{frame}{Deriving a test statistic}
\begin{enumerate}
	\item Calculate the differences in the $n$ pairs of observations \pause
	\item[]
	\item Subtract $D_0$ from all of the differences \pause
	\item[]
	\item Remove all zero values. Let $n$ be the number of non-zero values \pause
	\item[]
	\item List the absolute values of the differences in increasing orders and assign them the ranks $1, 2, \hdots, n$ (or the average of ranks in case of ties) \pause
	\item[]
	\item A few intermediate calculations: \pause
	\begin{itemize}
		\item $n =$ the number of pairs of observations with a nonzero difference \pause
		\item $T_+ = $  the sum of the positive ranks. If none then $T_+=0$ \pause
		\item $T_- = $ the sum of the negative ranks. If none then $T_- = 0$ \pause
		\item $T = $ the smaller of $T_+ = $ and $T_-$ 
	\end{itemize}
\end{enumerate}
\end{frame}

\begin{frame}{Example}{Towards a test statistic}
	\begin{itemize}
		\item Assume we are going to want to test against $D_0 = 0$ (no change in 200M times)
	\end{itemize}
	\begin{center}
	\begin{tabular}{|c|c|c|c|}
		\hline
		\textbf{Athlete} &  \textbf{Baseline} &      \textbf{Post}    & \textbf{Difference} \\ \hline 
		\hline
		1  &     119.0 &  118.3 &  -0.7\\ \hline
		2   &    145.6 &  139.6 &  -6.0\\ \hline
		3    &    97.7 &   95.9 &  -1.8\\ \hline
		4    &   123.4 &  116.0 &  -7.4\\ \hline
		5    &   103.0 &  101.3 &  -1.7\\ \hline
		6    &   116.8 &  110.1 &  -6.7\\ \hline
		7     &  135.7 &  134.7 &  -1.0\\ \hline
		8     &  127.8 &  128.1 &   0.2\\ \hline
		9     &  124.3 &  125.4 &   1.1\\ \hline
		10    &  113.5 &  112.0 &  -1.5\\ \hline
	\end{tabular}
\end{center}
\end{frame}

\begin{frame}{Towards a test statistic}{Step 1}
\begin{itemize}
	\item Calculate the differences in the $n$ pairs of observations
\end{itemize}
\begin{center}
	\begin{tabular}{|c|c|c|c|}
		\hline
		\textbf{Athlete} &  \textbf{Baseline} &      \textbf{Post}    & \textbf{Difference} \\ \hline 
		\hline
		1  &     119.0 &  118.3 &  -0.7\\ \hline
		2   &    145.6 &  139.6 &  -6.0\\ \hline
		3    &    97.7 &   95.9 &  -1.8\\ \hline
		4    &   123.4 &  116.0 &  -7.4\\ \hline
		5    &   103.0 &  101.3 &  -1.7\\ \hline
		6    &   116.8 &  110.1 &  -6.7\\ \hline
		7     &  135.7 &  134.7 &  -1.0\\ \hline
		8     &  127.8 &  128.1 &   0.2\\ \hline
		9     &  124.3 &  125.4 &   1.1\\ \hline
		10    &  113.5 &  112.0 &  -1.5\\ \hline
	\end{tabular}
\end{center}
\end{frame}

\begin{frame}{Towards a test statistic}{Step 2}
\begin{itemize}
	\item Subtract $D_0$ from all of the differences (in our case $D_0 = 0$)
\end{itemize}
\begin{center}
	\begin{tabular}{|c|c|c|c|c|}
		\hline
		\textbf{Athlete} &  \textbf{Baseline} &      \textbf{Post}    & \textbf{Difference} & $d_i -D_0$ \\ \hline 
		\hline
		1  &     119.0 &  118.3 &  -0.7 & -0.7\\ \hline
		2   &    145.6 &  139.6 &  -6.0 & -6.0\\ \hline
		3    &    97.7 &   95.9 &  -1.8 &  -1.8\\ \hline
		4    &   123.4 &  116.0 &  -7.4&  -7.4\\ \hline
		5    &   103.0 &  101.3 &  -1.7&  -1.7\\ \hline
		6    &   116.8 &  110.1 &  -6.7&  -6.7\\ \hline
		7     &  135.7 &  134.7 &  -1.0&  -1.0\\ \hline
		8     &  127.8 &  128.1 &   0.2&   0.2\\ \hline
		9     &  124.3 &  125.4 &   1.1&   1.1\\ \hline
		10    &  113.5 &  112.0 &  -1.5&  -1.5\\ \hline
	\end{tabular}
\end{center}
\end{frame}

\begin{frame}{Towards a test statistic}{Step 3}
\begin{itemize}
\item Remove all zero values. Let $n$ be the number of non-zero values
\end{itemize}
\begin{center}
	\begin{tabular}{|c|c|c|c|c|}
		\hline
		\textbf{Athlete} &  \textbf{Baseline} &      \textbf{Post}    & \textbf{Difference} & $d_i -D_0$ \\ \hline 
		\hline
		1  &     119.0 &  118.3 &  -0.7 & -0.7\\ \hline
		2   &    145.6 &  139.6 &  -6.0 & -6.0\\ \hline
		3    &    97.7 &   95.9 &  -1.8 &  -1.8\\ \hline
		4    &   123.4 &  116.0 &  -7.4&  -7.4\\ \hline
		5    &   103.0 &  101.3 &  -1.7&  -1.7\\ \hline
		6    &   116.8 &  110.1 &  -6.7&  -6.7\\ \hline
		7     &  135.7 &  134.7 &  -1.0&  -1.0\\ \hline
		8     &  127.8 &  128.1 &   0.2&   0.2\\ \hline
		9     &  124.3 &  125.4 &   1.1&   1.1\\ \hline
		10    &  113.5 &  112.0 &  -1.5&  -1.5\\ \hline
	\end{tabular}
\end{center}
\end{frame}

\begin{frame}{Towards a test statistic}{Step 4}
\begin{itemize}
	\item List the absolute values of the differences in increasing orders and assign them the ranks $1, 2, \hdots, n$ (or the average of ranks for ties)
\end{itemize}
\begin{center}
	\begin{tabular}{|c|c|c|}
		\hline
		$d_i -D_0$ & \textbf{$|d_i - D_0|$} &  \textbf{Rank}  \\ \hline \hline
   0.2 & 0.2 &     1\\ \hline
  -0.7  &0.7  &    2\\ \hline
  -1.0  &1.0  &    3\\ \hline
   1.1  & 1.1  &    4\\ \hline
  -1.5  & 1.5  &    5\\ \hline
  -1.7  &1.7  &    6\\ \hline
  -1.8  & 1.8  &    7\\ \hline
  -6.0  & 6.0  &    8\\ \hline
  -6.7  & 6.7  &    9\\ \hline
  -7.4  &7.4  &   10\\ \hline
	\end{tabular}
\end{center}
\end{frame}

\begin{frame}{Towards a test statistic}{Step 5}
\begin{itemize}
	\item $n = 10$ the number of pairs of observations with a nonzero difference \pause
	\item $T_+ = 5$  the sum of the positive ranks \pause
	\item $T_- = 50$ the sum of the negative ranks \pause
	\item $T^* = 5$ the smaller of $T_+ = $ and $T_-$  \pause
\end{itemize}
{\tiny
\begin{center}
	\begin{tabular}{|c|c|c|}
		\hline
		$d_i -D_0$ & \textbf{$|d_i - D_0|$} &  \textbf{Rank}  \\ \hline \hline
		0.2 & 0.2 &     1\\ \hline
		-0.7  &0.7  &    2\\ \hline
		-1.0  &1.0  &    3\\ \hline
		1.1  & 1.1  &    4\\ \hline
		-1.5  & 1.5  &    5\\ \hline
		-1.7  &1.7  &    6\\ \hline
		-1.8  & 1.8  &    7\\ \hline
		-6.0  & 6.0  &    8\\ \hline
		-6.7  & 6.7  &    9\\ \hline
		-7.4  &7.4  &   10\\ \hline
	\end{tabular}
\end{center}}
\end{frame}

\begin{frame}{The sampling distribution}
	\begin{itemize}
		\item The sampling distribution for the test statistic $T$ has the following parameters: \pause
		\begin{gather*}
			\mu_{T} = \frac{n (n+1)}{4} \quad \quad \text{and} \quad \quad \sigma_{T}^2 = \frac{n(n+1)(2n+1)}{24}
		\end{gather*} \pause
		\item As with the Wilcoxon Rank-Sum test statistic, when there are ties we adjust the variance: \pause
		\begin{gather*}
			\sigma_{T}^2 = \frac{n(n+1)(2n+1)-1/2 \sum_j t_j (t_j-1)(t_j+1)}{24}
		\end{gather*}
		where $j$ is the index for the groups of ties and $t_j$ is the number of tied ranks in group $j$
	\end{itemize}
\end{frame}

\begin{frame}{The Wilcoxon Signed-Rank test (small samples $n\leq 50$)}
		\begin{itemize}
		\item Null Hypothesis. $H_0$: $M=D_0$ where $M$ is the median of population differences \pause
		\item[]
		\item Alternative hypotheses (choose one). $H_a$: \pause
		\begin{enumerate}
			\item $M > D_0$
			\item $M < D_0$
			\item $M \neq D_0$
			\item[]
		\end{enumerate} \pause
		\item Test statistic: $T^*$ is the smaller of $T_-$ and $T_+$ \pause
		\item[]
		\item Rejection regions / rejection rules: \pause
		\begin{enumerate}
			\item Reject $H_0$ if $T^* \leq T$ for one-sided critical value (Appendix 6)
			\item Reject $H_0$ if $T^* \leq T$ for one-sided critical value (Appendix 6)
			\item Reject $H_0$ if $T^* \leq T$ for two-sided critical value (Appendix 6)
		\end{enumerate}
	\end{itemize}
\end{frame}

\begin{frame}{The Wilcoxon Signed-Rank test (small samples $n\leq 50$)}{Critical values}
	\begin{center}
		\includegraphics[width=.8 \linewidth]{critical}
	\end{center}
\end{frame}

\begin{frame}{Swim Program example}
	\begin{itemize}
		\item Assume the exercise scientist wants to test whether the program improves times (smaller times = negative differences) at $\alpha = .05$ \pause
		{\tiny
			\begin{center}
				\begin{tabular}{|c|c|c|}
					\hline
					$d_i -D_0$ & \textbf{$|d_i - D_0|$} &  \textbf{Rank}  \\ \hline \hline
					0.2 & 0.2 &     1\\ \hline
					-0.7  &0.7  &    2\\ \hline
					-1.0  &1.0  &    3\\ \hline
					1.1  & 1.1  &    4\\ \hline
					-1.5  & 1.5  &    5\\ \hline
					-1.7  &1.7  &    6\\ \hline
					-1.8  & 1.8  &    7\\ \hline
					-6.0  & 6.0  &    8\\ \hline
					-6.7  & 6.7  &    9\\ \hline
					-7.4  &7.4  &   10\\ \hline
				\end{tabular}
		\end{center}} \pause
	\item Hypothesis test: $H_0: M = 0$ versus  $H_a: M < 0$ (one-sided test) \pause
	\item[]
	\item From before: $n = 10$, $T^* = 5$ \pause
	\item[] 
	\item From the appendix table our critical value is $T = 10$, so since $T^* < T$ we reject the null hypothesis (we have evidence the program improves times!)
	\end{itemize}
\end{frame}

\begin{frame}{The Wilcoxon Signed-Rank test (large samples $n > 50$)}
	\begin{itemize}
		\item When $n >50$ a Normal approximation is appropriate \pause
		\item[]
		\item Test statistic:
		\begin{gather*}
			z = \frac{T-\frac{n(n+1)}{4}}{\sqrt{\frac{n(n+1)(2n+1)}{24}}}
		\end{gather*}
		\item[]
		\item Rejection rule: For cases 1 and 2, reject $H_0$ if $z<-z_{\alpha}$; For case 3, reject $H_0$ if $z < -z_{\alpha/2}$
	\end{itemize}
\end{frame}

\begin{frame}{The end of Lecture \#11}
	\begin{center}
		\includegraphics[width=.7 \linewidth]{Princess} \\
		{\tiny Princess Falls in McCreary County via @ExploreKentucky}
	\end{center}
\end{frame}

\end{document}