\documentclass[xcolor=dvipsnames]{beamer} 
\usetheme{AnnArbor}
\usecolortheme{beaver}

\usepackage{amsmath,graphicx,booktabs,tikz,subfig,color,lmodern}
\definecolor{mycol}{rgb}{.4,.85,1}
\setbeamercolor{title}{bg=mycol,fg=black} 
\setbeamercolor{palette primary}{use=structure,fg=white,bg=red}
\setbeamercolor{block title}{fg=white,bg=red!50!black}
% \setbeamercolor{block title}{fg=white,bg=blue!75!black}

\title[Lecture 12]{Lecture 12: Power \& Sample Size}
\author[Patrick Trainor]{Patrick Trainor, PhD, MS, MA}
\institute[NMSU]{New Mexico State University}
\date{October 2, 2019}

\begin{document}

\begin{frame}
\maketitle
\end{frame}

\begin{frame}{Outline}
\tableofcontents[hideallsubsections]
\end{frame}

\section{Sample size for estimating $\mu$ (CI width)}
\section{Introduction}
\begin{frame}{Outline}
\tableofcontents[currentsection,subsectionstyle=show/shaded/hide]
\end{frame}

\begin{frame}{Confidence intervals}{What is a 95\% CI?}
\begin{itemize}
	\item Back to confidence intervals for $\mu$, with $\sigma$ known:
	\begin{gather*}
		\bar{y}\pm z_{1-\alpha/2} \sigma \sqrt{n}
	\end{gather*}
	\item The idea was to fix $\alpha$ our probability of making a Type I error or equivalently to fix a level of confidence $100(1-\alpha)\%$
	\item[]
	\item For example, if we want a 95\% confidence interval for $\mu$ we have $\alpha = 0.05$
	\item[]
	\item We expect that \emph{on average} a 95\% CI will contain $\mu$,  95\% of the time
\end{itemize}
\end{frame}

\begin{frame}{Confidence intervals}{What is a 95\% CI?}
	\begin{center}
		\includegraphics[width=.9\linewidth]{coverage1}
	\end{center}
\end{frame}

\begin{frame}{Confidence intervals}{What is a 95\% CI?}
\begin{center}
	\includegraphics[width=.9\linewidth]{coverage2}
\end{center}
\end{frame}

\begin{frame}{Confidence intervals}{What is a 95\% CI?}
\begin{center}
	\includegraphics[width=.9\linewidth]{coverage3}
\end{center}
\end{frame}

\begin{frame}{Confidence intervals}{What do we control?}
\begin{itemize}
	\item Back to confidence intervals for $\mu$, with $\sigma$ known:
	\begin{gather*}
	\bar{y}\pm z_{1-\alpha/2} \sigma \sqrt{n}
	\end{gather*}
	\item There are two values that we exercise control over: $\alpha$ and $n$
	\item[]
	\item $\bar{y}$ is computed from the data and $\sigma$ is known (or estimated from the data). We cannot control these
\end{itemize}
\end{frame}

\begin{frame}{Changing $\alpha$}
	\begin{center}
			\includegraphics[width=.9\linewidth]{coverage1b}
	\end{center}
\end{frame}

\begin{frame}{Changing $\alpha$}
\begin{center}
	\includegraphics[width=.7\linewidth]{coverage1c}
\end{center}
\end{frame}

\begin{frame}{Changing $\alpha$}
\begin{itemize}
	\item You may notice that the smaller the interval the less likely the population parameter is to be in the interval 
	\begin{itemize}
		\item Small $\alpha$ $\rightarrow$ wide interval ($\mu$ is more likely to be in the CI)
		\item Large $\alpha$ $\rightarrow$ narrow interval ($\mu$ is less likely to be in the CI)
	\end{itemize}
	\item[]
	\item The level of $\alpha$ is ultimately \emph{your} choice
	\item[]
	\item Some conventions exist. ($\alpha = .05$) is pervasive
	\item[]
	\item Ultimately you decide the level of risk ($\alpha$) you can tolerate for committing a Type 1 error (having a population parameter $\mu$ not being in the CI)
\end{itemize}
\end{frame}

\begin{frame}{Changing $n$}
\begin{center}
	\includegraphics[width=.9\linewidth]{coverage1bb}
\end{center}
\end{frame}

\begin{frame}{Changing $n$}
\begin{center}
	\includegraphics[width=.9\linewidth]{coverage1bbb}
\end{center}
\end{frame}

\begin{frame}{Changing $n$}
\begin{center}
	\includegraphics[width=.7\linewidth]{coverage1cc}
\end{center}
\end{frame}

\begin{frame}{Changing $n$}
\begin{itemize}
		\item A greater sample size $n$ $\rightarrow$ smaller confidence intervals for a fixed $\alpha$ or confidence level
	\item[]
	\item Intuitive result: more data $=$ more certainty as to the location of a population parameter
\end{itemize}
\end{frame}

\begin{frame}{The width of a confidence interval}
	\begin{itemize}
		\item Much of the time we know what $\alpha$ we want (what type I error rate we can tolerate), but we need to know what $n$ should be to allow us to have a CI of a specific width
		\item[]
		\item Width of a CI can be identified:
		\begin{align*}
			\bar{y}\pm z_{1-\alpha/2} \sigma / \sqrt{n} = \bar{y} \pm E = (\bar{y}-E, \bar{y}+E)
		\end{align*}
		So width: $W = 2E$ where $E = z_{1-\alpha/2} \sigma  /\sqrt{n}$
	\end{itemize}
\end{frame}

\begin{frame}{Example: Municipal water testing}
	\begin{center}
		\includegraphics[width = 1\linewidth]{crescent}
	\end{center}
\end{frame}

\begin{frame}{Desired with of a confidence interval}
	\begin{itemize}
		\item Example: The city of Louisville, Kentucky needs to estimate the population mean $\mu$ of lead in tap water (measured in parts per billion). The city has a low tolerance for type I errors (population parameter in the CI), so will set $\alpha = .01$ and look for a 99\% CI. The city knows from past testing that the variance in lead measurements from tap water sources is 1 ppb. What sample size (of independent tap water measurements) would be required for the city to estimate the population mean lead level to have a CI with width 0.1 ppb?
	\end{itemize}
\end{frame}

\begin{frame}{The width of a confidence interval}
\begin{itemize}
	\item Above we found:
	\begin{align*}
	\bar{y}\pm z_{1-\alpha/2} \sigma \sqrt{n} = \bar{y} \pm E = (\bar{y}-E, \bar{y}+E)
	\end{align*}
	where $W = 2E$ and $E = z_{1-\alpha/2} \sigma \sqrt{n}$
	\item[]
	\item So to find $n$ we do some algebra:
	\begin{gather*}
		E = z_{1-\alpha/2} \sigma / \sqrt{n} \\
		\sqrt{n} E = z_{1-\alpha/2} \sigma  \\
		\sqrt{n} = \frac{z_{1-\alpha/2} \sigma}{E} \\
		n = \left(\frac{z_{1-\alpha/2} \sigma}{E}\right)^2 = \frac{(z_{1-\alpha/2})^2 \sigma^2}{E^2}
	\end{gather*}
\end{itemize}
\end{frame}

\begin{frame}{Desired with of a confidence interval}
		\begin{itemize}
		\item Example: The city of Louisville, Kentucky needs to estimate the population mean $\mu$ of lead in tap water (measured in parts per billion). The city has a low tolerance for type I errors (population parameter in the CI), so will set $\alpha = .01$ and look for a 99\% CI. The city knows from past testing that the variance in lead measurements from tap water sources is 1 ppb. What sample size (of independent tap water measurements) would be required for the city to estimate the population mean lead level to have a CI with width 0.1 ppb?
		\item[]
		\item So since we want $W = 0.1$, $E = 0.01/2 = 0.05$
		\item[]
		\item We also know $\alpha = .01$, $\sigma^2 = 1$
	\end{itemize}
\end{frame}

\begin{frame}{Desired with of a confidence interval}
\begin{itemize}
	\item Given $E = 0.05$, $\alpha = .01$, $\sigma^2 = 1$:
	\begin{align*}
		n &= \frac{(z_{1-\alpha/2})^2 \sigma^2}{E^2}\\
		n &=\frac{2.58^2 1^2}{0.05^2} \\
		n &= 2662.56
	\end{align*}
	\item So the city should plan a sample of size $n = 2663$ to achieve that expected precision
\end{itemize}
\end{frame}

\section{Power ($1-\beta$) of a test}

\begin{frame}{Outline}
	\tableofcontents[currentsection,subsectionstyle=show/shaded/hide]
\end{frame}

\begin{frame}{Continuing with the water example}
	\begin{itemize}
		\item Example \#2: The city of Louisville, Kentucky wants to test whether the population mean $\mu$ of lead in tap water (in PPB) is less than 5.
		\begin{itemize}
			\item $H_0: \mu_0 \geq 5$
			\item $H_a: \mu_0 < 5$
			\item[]
		\end{itemize}
		\item The city needs to be conservative for the safety of residents, so the water quality officer sets $\alpha = 0.01$. He knows that the variance in lead measurements from tap water sources is 1 ppb
		\item[]
		\item The water quality officer decides he will randomly select $n=10$ taps for testing the water for the presence of lead
		\item[]
		\item Measurements: \{4.42, 6.43, 2.45, 4.78, 2.97, 4.23, 5.73, 5.13, 4.85, 3.94\}
	\end{itemize}
\end{frame}

\begin{frame}{Water example \#2}
	\begin{itemize}
		\item So we have that $\bar{x} = 4.493$, $\sigma = 1$, $\alpha = .01$, and $n = 10$
		\item[]
		\item We have the rejection rule: Reject $H_0$ if $z^* \leq -2.33$ as  $z_{\alpha} = z_{.01}= -2.33$
		\item[]
		\item To calculate our test statistic:
		\begin{align*}
			z &= \frac{\bar{x}-\mu_0}{\sigma / \sqrt{n}} \\
			&=\frac{4.493-5}{1 / \sqrt{10}} \\
			&= -1.603
		\end{align*}
		\item So we fail to reject the null hypothesis
	\end{itemize}
\end{frame}

\begin{frame}{Water example \#2}
	\begin{itemize}
		\item Now purely for the sake of argument let's assume the same sample mean had come from 100 measurements, that is $n = 100$
		\item[]
		\item So we have that $\bar{x} = 4.493$, $\sigma = 1$, $\alpha = .01$, and $n = 10$
		\item[]
		\item To calculate our test statistic:
		\begin{align*}
		z &= \frac{\bar{x}-\mu_0}{\sigma / \sqrt{n}} \\
		&=\frac{4.493-5}{1 / \sqrt{100}} \\
		&= -5.07
		\end{align*}
		\item So we would have rejected the null hypothesis with a $p$-value of 0.0000002
	\end{itemize}
\end{frame}

\begin{frame}{Rejection \& Acceptance regions}{Types of errors}
	\begin{itemize}
		\item \textbf{\emph{Type I error:}} A Type I error is commited if we reject the null hypothesis when it is true. This is denoted $\alpha$
		\item[]
		\item \textbf{\emph{Type II error:}} A Type II error is commited if we accept the null hypothesis when it is false and the alternative hypothesis is true. This is denoted $\beta$
	\end{itemize}
	\begin{center}
		\begin{tabular}{c|cc}
			& \textbf{Null hypothesis} & \\
			\textbf{Decision} & \textbf{True} & \textbf{False} \\ \hline
			Reject $H_0$ & Type I error & Correct \\
			& $\alpha$ & $1-\beta$ \\ \hline
			Accept $H_0$ & Correct & Type II error \\
			& $1-\alpha$ & $\beta$ \\ \hline
		\end{tabular}
	\end{center}
\end{frame}

\begin{frame}{Acceptance Regions}{Function of sample size}
	\begin{center}
		\includegraphics[width=.9\linewidth]{overlap0}
	\end{center}
\end{frame}

\begin{frame}{Acceptance Regions}{Function of sample size}
	\begin{center}
		\includegraphics[width=.9\linewidth]{overlap1}
	\end{center}
\end{frame}

\begin{frame}{Acceptance Regions}{Function of sample size}
	\begin{center}
		\includegraphics[width=.9\linewidth]{overlap11}
	\end{center}
\end{frame}

\begin{frame}{Acceptance Regions}{Function of sample size}
	\begin{center}
		\includegraphics[width=.9\linewidth]{overlap2}
	\end{center}
\end{frame}

\begin{frame}{Acceptance Regions}{Function of sample size}
	\begin{center}
		\includegraphics[width=.9\linewidth]{overlap3}
	\end{center}
\end{frame}

\begin{frame}{Acceptance Regions}{Function of sample size}
	\begin{center}
		\includegraphics[width=.9\linewidth]{overlap4}
	\end{center}
\end{frame}

\begin{frame}{Acceptance Regions}{Function of difference between alternatives}
	\begin{center}
		\includegraphics[width=.9\linewidth]{overlap1}
	\end{center}
\end{frame}

\begin{frame}{Acceptance Regions}{Function of difference between alternatives}
	\begin{center}
		\includegraphics[width=.9\linewidth]{overlap1b}
	\end{center}
\end{frame}

\begin{frame}{Rejection \& Acceptance regions}{Types of errors}
	\begin{itemize}
		\item \textbf{\emph{Type I error:}} A Type I error is commited if we reject the null hypothesis when it is true. This is denoted $\alpha$
		\item[]
		\item \textbf{\emph{Type II error:}} A Type II error is commited if we accept the null hypothesis when it is false and the alternative hypothesis is true. This is denoted $\beta$
	\end{itemize}
	\begin{center}
		\begin{tabular}{c|cc}
			& \textbf{Null hypothesis} & \\
			\textbf{Decision} & \textbf{True} & \textbf{False} \\ \hline
			Reject $H_0$ & Type I error & Correct \\
			& $\alpha$ & $1-\beta$ \\ \hline
			Accept $H_0$ & Correct & Type II error \\
			& $1-\alpha$ & $\beta$ \\ \hline
		\end{tabular}
	\end{center}
\end{frame}

\end{document}