\documentclass[xcolor=dvipsnames]{beamer} 
\usetheme{AnnArbor}
\usecolortheme{beaver}

\usepackage{amsmath,graphicx,booktabs,tikz,subfig,color,lmodern}
\definecolor{mycol}{rgb}{.4,.85,1}
\setbeamercolor{title}{bg=mycol,fg=black} 
\setbeamercolor{palette primary}{use=structure,fg=white,bg=red}
\setbeamercolor{block title}{fg=white,bg=red!50!black}
% \setbeamercolor{block title}{fg=white,bg=blue!75!black}

\title[Lecture 12]{Lecture 12: Power \& Sample Size}
\author[Patrick Trainor]{Patrick Trainor, PhD, MS, MA}
\institute[NMSU]{New Mexico State University}
\date{October 2, 2019}

\begin{document}

\begin{frame}
\maketitle
\end{frame}

\begin{frame}{Outline}
\tableofcontents[hideallsubsections]
\end{frame}

\section{Introduction}
\begin{frame}{Outline}
\tableofcontents[currentsection,subsectionstyle=show/shaded/hide]
\end{frame}

\begin{frame}{Confidence intervals}{What is a 95\% CI?}
\begin{itemize}
	\item Back to confidence intervals for $\mu$, with $\sigma$ known:
	\begin{gather*}
		\bar{y}\pm z_{1-\alpha/2} \sigma \sqrt{n}
	\end{gather*}
	\item The idea was to fix $\alpha$ our probability of making a Type I error or equivalently to fix a level of confidence $100(1-\alpha)\%$
	\item[]
	\item For example, if we want a 95\% confidence interval for $\mu$ we have $\alpha = 0.05$
	\item[]
	\item We expect that \emph{on average} a 95\% CI will contain $\mu$,  95\% of the time
\end{itemize}
\end{frame}

\begin{frame}{Confidence intervals}{What is a 95\% CI?}
	\begin{center}
		\includegraphics[width=.9\linewidth]{coverage1}
	\end{center}
\end{frame}

\begin{frame}{Confidence intervals}{What is a 95\% CI?}
\begin{center}
	\includegraphics[width=.9\linewidth]{coverage2}
\end{center}
\end{frame}

\begin{frame}{Confidence intervals}{What is a 95\% CI?}
\begin{center}
	\includegraphics[width=.9\linewidth]{coverage3}
\end{center}
\end{frame}

\begin{frame}{Confidence intervals}{What do we control?}
\begin{itemize}
	\item Back to confidence intervals for $\mu$, with $\sigma$ known:
	\begin{gather*}
	\bar{y}\pm z_{1-\alpha/2} \sigma \sqrt{n}
	\end{gather*}
	\item There are two values that we exercise control over: $\alpha$ and $n$
	\item[]
	\item $\bar{y}$ is computed from the data and $\sigma$ is known (or estimated from the data). We cannot control these
\end{itemize}
\end{frame}

\begin{frame}{Changing $\alpha$}
	\begin{center}
			\includegraphics[width=.9\linewidth]{coverage1b}
	\end{center}
\end{frame}

\begin{frame}{Changing $\alpha$}
\begin{center}
	\includegraphics[width=.7\linewidth]{coverage1c}
\end{center}
\end{frame}

\begin{frame}{Changing $\alpha$}
\begin{itemize}
	\item You may notice that the smaller the interval the less likely the population parameter is to be in the interval 
	\begin{itemize}
		\item Small $\alpha$ $\rightarrow$ wide interval ($\mu$ is more likely to be in the CI)
		\item Large $\alpha$ $\rightarrow$ narrow interval ($\mu$ is less likely to be in the CI)
	\end{itemize}
	\item[]
	\item The level of $\alpha$ is ultimately \emph{your} choice
	\item[]
	\item Some conventions exist. ($\alpha = .05$) is pervasive
	\item[]
	\item Ultimately you decide the level of risk ($\alpha$) you can tolerate for committing a Type 1 error (having a population parameter $\mu$ not being in the CI)
\end{itemize}
\end{frame}

\begin{frame}{Changing $n$}
\begin{center}
	\includegraphics[width=.9\linewidth]{coverage1bb}
\end{center}
\end{frame}

\begin{frame}{Changing $n$}
\begin{center}
	\includegraphics[width=.9\linewidth]{coverage1bbb}
\end{center}
\end{frame}

\begin{frame}{Changing $n$}
\begin{center}
	\includegraphics[width=.7\linewidth]{coverage1cc}
\end{center}
\end{frame}

\end{document}