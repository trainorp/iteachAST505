\documentclass[xcolor=dvipsnames]{beamer} 
\usetheme{AnnArbor}
\usecolortheme{beaver}

\usepackage{amsmath,graphicx,booktabs,tikz,subfig,color,lmodern}
\definecolor{mycol}{rgb}{.4,.85,1}
\setbeamercolor{title}{bg=mycol,fg=black} 
\setbeamercolor{palette primary}{use=structure,fg=white,bg=red}
\setbeamercolor{block title}{fg=white,bg=red!50!black}
% \setbeamercolor{block title}{fg=white,bg=blue!75!black}

\title[Lecture 13]{Lecture 13: Inferences about population variances}
\author[Patrick Trainor]{Patrick Trainor, PhD, MS, MA}
\institute[NMSU]{New Mexico State University}
\date{October}

\begin{document}
	
\begin{frame}
	\maketitle
\end{frame}

\begin{frame}{Introduction}
	\begin{itemize}
		\item There are many instances in which the variability in a population may be of interest
		\item[]
		\item Examples:
		\begin{itemize}
			\item Manufacturing: The amount of active and inactive ingredients in pharmaceuticals must have low-variability
			\item[]
			\item Finance: Given two different portfolios with equal long run expected returns, the one with less variability is desirable
			\item[]
			\item Medicine: Given two treatments with equal overall efficacy, the one with the least variability in treatment response might be preferred
			\item[]
			\item Analytical chemistry: We always want instruments that give us less variability in measuring anything \& everything!
		\end{itemize}
	\end{itemize}
\end{frame}

\begin{frame}{Point estimates and sampling distribution for $\sigma^2$}
	\begin{itemize}
		\item In order to estimate $\sigma^2$ for a population that has a normal distribution, we want a point estimate and confidence intervals
		\item[]
		\item We already know the point estimate that is the best estimate of $\sigma^2$:
		\begin{gather*}
			s^2=\frac{\sum_{i=1}^{n}(y_i-\bar{y})^2}{n-1}
		\end{gather*}
		\item Towards constructing confidence intervals for $\sigma^2$, we need to introduce a new sampling distribution: the $\chi^2$-distribution 
		\item[]
		\item This distribution is the distribution of the sum of the squares from normal random variables
	\end{itemize}
\end{frame}

\begin{frame}{The  $\chi^2$-distribution}
	\begin{center}
		\includegraphics[width=.87 \linewidth]{chi}
	\end{center}
\end{frame}

\begin{frame}{The  $\chi^2$-distribution }
	\begin{itemize}
		\item Like the Student's $t$-distribution, the $\chi^2$-distribution has degrees of freedom $\text{df}=n-1$
		\item[]
		\item The df determine which of the many different $\chi^2$-distributions we are describing
		\item[]
		\item The distribution has values between $0$ and $\infty$
		\item[]
		\item The distribution has $\mu = \text{df}$, and $\sigma^2 = 2\text{df}$
	\end{itemize}
\end{frame}

\begin{frame}{Confidence intervals for $\sigma^2$}
	\begin{itemize}
		\item Given a confidence coefficient of $1-\alpha$, a confidence interval for $\sigma^2$ is:
		\begin{gather*}
		\left(\frac{(n-1) s^2}{\chi_U^2}, \frac{(n-1)s^2}{\chi_L^2} \right)
		\end{gather*}
		\item Critical values:
		\begin{itemize}
			\item $\chi^2_U$ corresponds to $\alpha / 2$
			\item $\chi^2_L$ corresponds to $1 - \alpha / 2$
			\item[]
		\end{itemize}
	\item Example: Let's say we want a 90\% confidence interval for $\sigma^2$ and we have a sample size of $n = 7$. Then:
	\begin{itemize}
		\item  $\text{df} = n-1 = 6$
		\item $\alpha = .10$, $\alpha / 2 = .05$, and $1-\alpha / 2 = .95$
	\end{itemize}
	\end{itemize}
\end{frame}

\begin{frame}{The  $\chi^2$-distribution}
	For $\chi^2_L$:
	\begin{center}
		\includegraphics[width=.95 \linewidth]{chiTable}
	\end{center}
\end{frame}

\begin{frame}{The  $\chi^2$-distribution}
	For $\chi^2_U$:
	\begin{center}
		\includegraphics[width=.95 \linewidth]{chiTable2}
	\end{center}
\end{frame}

\begin{frame}{Confidence intervals for $\sigma^2$}{Example}
	\begin{columns}
		\begin{column}{.5 \textwidth}
			 John Fenn's Nobel Prize winning single quadrupole
			\includegraphics[width = 1\linewidth]{Fenn_ESI_Instrument}
		\end{column}
		\begin{column}{.5\textwidth}
			\begin{itemize}
				\item Assume you have an old mass spectrometer
				\item[]
				\item You want to measure the mass of protonated alpha-ketoglutarate ions (147.0288)
				\item[]
				\item You make 7 measurements: \{147.0192, 147.0259, 147.0314, 147.0173, 147.0308, 147.0291, 147.0297\}
			\end{itemize}
		\end{column}
	\end{columns}
\end{frame}

\begin{frame}{Confidence intervals for $\sigma^2$}{Example}
	\begin{itemize}
		\item You make 7 measurements: \{147.0192, 147.0259, 147.0314, 147.0173, 147.0308, 147.0291, 147.0297\}
		\item[]
		\item What is the point estimate and 90\% confidence interval for the population variance $\sigma^2$?
		\begin{gather*}
			s^2 = 3.283322 \times 10^{-5}
		\end{gather*}
		\begin{gather*}
			\left(\frac{(n-1) s^2}{\chi_U^2}, \frac{(n-1)s^2}{\chi_L^2} \right)
		\end{gather*}
		
		\begin{itemize}
			\item $\chi^2_L = 1.635$ corresponds to $1 - \alpha / 2 =.95$ and $\text{df} = 6$
			\item $\chi^2_U = 12.59$ corresponds to $\alpha / 2 = .05$ $\text{df} = 6$
			\item[]
		\end{itemize}
	\end{itemize}
\end{frame}

\begin{frame}{Confidence intervals for $\sigma^2$}{Example}
	\begin{itemize}
		\item What is the point estimate and 90\% confidence interval for the population variance $\sigma^2$?
		\begin{gather*}
		s^2 = 3.283322 \times 10^{-5}
		\end{gather*}
		\begin{gather*}
		\left(\frac{(n-1) s^2}{\chi_U^2}, \frac{(n-1)s^2}{\chi_L^2} \right) = \\
		\left(\frac{(6) 3.283322 \times 10^{-5}}{12.59}, \frac{(6)3.283322 \times 10^{-5}}{1.635} \right) = \\
		(1.564729  \times 10^{-5}, 12.04889  \times 10^{-5})
		\end{gather*}
	\end{itemize}
\end{frame}


\end{document}