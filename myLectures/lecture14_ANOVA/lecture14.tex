\documentclass[xcolor=dvipsnames]{beamer} 
\usetheme{AnnArbor}
\usecolortheme{beaver}

\usepackage{amsmath,graphicx,booktabs,tikz,subfig,color,lmodern}
\definecolor{mycol}{rgb}{.4,.85,1}
\setbeamercolor{title}{bg=mycol,fg=black} 
\setbeamercolor{palette primary}{use=structure,fg=white,bg=red}
\setbeamercolor{block title}{fg=white,bg=red!50!black}
% \setbeamercolor{block title}{fg=white,bg=blue!75!black}

\title[Lecture 14]{Lecture 14: Analysis of Variance}
\author[Patrick Trainor]{Patrick Trainor, PhD, MS, MA}
\institute[NMSU]{New Mexico State University}
\date{October}

\begin{document}
	
\begin{frame}
	\maketitle
\end{frame}

\begin{frame}{Outline}
	\tableofcontents[hideallsubsections]
\end{frame}

\section{Introduction}

\begin{frame}{Introduction}
	\begin{itemize}
		\item Within-group (sample) variation: Variability of individual measurements that are all from one group (sample). 
		\begin{itemize}
			\item Not all of the measurements from one group will be equal to the mean in that group
			\item This type of variation may be related to the independent variable
		\end{itemize}
		
		\item Between-group (sample) variation: Variability between separate groups of interest. 
		\begin{itemize}
			\item Not all of the means for each group will be identical
			\item This type of variation may be related to the independent variable
		\end{itemize}
		
		\item When the between-group variation is large relative to the within-group variation we have evidence that population means are different 
		
	\end{itemize}
\end{frame}

\end{document}