\documentclass[xcolor=dvipsnames]{beamer} 
\usetheme{AnnArbor}
\usecolortheme{beaver}

\usepackage{amsmath,graphicx,booktabs,tikz,subfig,color,lmodern}
\definecolor{mycol}{rgb}{.4,.85,1}
\setbeamercolor{title}{bg=mycol,fg=black} 
\setbeamercolor{palette primary}{use=structure,fg=white,bg=red}
\setbeamercolor{block title}{fg=white,bg=red!50!black}
% \setbeamercolor{block title}{fg=white,bg=blue!75!black}

\newcommand\myeq{\mathrel{\overset{\makebox[0pt]{\mbox{\normalfont\tiny\sffamily D}}}{=}}}

\title[Lecture 15]{Lecture 15: Non-parametric alternative to ANOVA}
\author[Patrick Trainor]{Patrick Trainor, PhD, MS, MA}
\institute[NMSU]{New Mexico State University}
\date{October}

\begin{document}
	
\begin{frame}
	\maketitle
\end{frame}

\begin{frame}{Outline}
	\tableofcontents[hideallsubsections]
\end{frame}

\section{Introduction}

\begin{frame}{Outline}
	\tableofcontents[currentsection,subsectionstyle=show/shaded/hide]
\end{frame}

\begin{frame}{Introduction}
	\begin{itemize}
		\item Earlier in the course we studied the Wilcoxon rank sum test for comparing distributions of two populations that do not follow a normal distribution
		\item[]
		\item The idea was to determine if one population distribution was a location shift from the other: $y \myeq x + \Delta$
		\item[]
		\item Now we will extend this test to the case where we have 3 or more populations that do not follow a normal population
	\end{itemize}
\end{frame}

\begin{frame}{Introduction}{Example}
	\begin{itemize}
		\item The expression (or amount) of each gene in the human body (or other multicellular organism) differs by the type of tissue 
		\begin{center}
			\includegraphics[width=.75\linewidth]{GE}
		\end{center}
	\end{itemize}
\end{frame}

\begin{frame}{Introduction}{Example}
	\begin{itemize}
		\item You observe the following data for the expression of Gene X
		\begin{center}
			\includegraphics[width=.9\linewidth]{GE}
		\end{center}
	\end{itemize}
\end{frame}

\begin{frame}{Introduction}{Example}
	\begin{itemize}
		\item You want to test whether they each have the same distribution or not
		\begin{center}
			\includegraphics[width=.9\linewidth]{GE}
		\end{center}
	\end{itemize}
\end{frame}

\section{Kruskal-Wallis test}

\begin{frame}{Outline}
	\tableofcontents[currentsection,subsectionstyle=show/shaded/hide]
\end{frame}

\begin{frame}{Kruskal-Wallis test}{Procedure}
	\begin{itemize}
		\item Hypotheses:
		\begin{itemize}
			\item $H_0:$ The $p$ distributions are identical
			\item $H_a:$ Not all the distributions are the same
		\end{itemize}
		\item[]
		\item Test statistic:
		\begin{gather*}
		H = \frac{12}{N (N+1)} \sum_{j=1}^p \frac{T_j^2}{n_j} - 3(N + 1)
		\end{gather*}
		where: 
		\begin{itemize}
			\item $N$ is the total number of observations (all samples)
			\item $n_j$ is the number of observations in the sample from population $j$ ($j = 1, 2, \hdots p$)
			\item $T_j$ is the sum of ranks for the measurements in sample $j$ from the combined and ranked set of measurements
		\end{itemize}
	\end{itemize}
\end{frame}

\begin{frame}{Kruskal-Wallis test}{Procedure}
	\begin{itemize}
		\item Test statistic:
		\begin{gather*}
		H = \frac{12}{N (N+1)} \sum_{j=1}^p \frac{T_j^2}{n_j} - 3(N + 1)
		\end{gather*}
		where: 
		\begin{itemize}
			\item $N$ is the total number of observations (all samples)
			\item $n_j$ is the number of observations in the sample from population $j$ ($j = 1, 2, \hdots p$)
			\item $T_j$ is the sum of ranks for the measurements in sample $j$ from the combined and ranked set of measurements
			\item[]
		\end{itemize}
		
		\item Rejection rule: For a fixed $\alpha$, reject $H_0$ if $H$ exceeds the critical value of $\chi^2$ for $\alpha$ and $\text{df} = p - 1$
	\end{itemize}
\end{frame}

\begin{frame}{Kruskal-Wallis test}{Adjustment for ties}
	\begin{itemize}
		\item When there are a large number of ties in the ranks we adjust the test statistic:
		\begin{gather*}
			H' = \frac{H}{1 - \left[ \sum_{k=1}^K (t_k^3 - t_k)/(N^3 - N) \right]}
		\end{gather*}
		where $t_k$ is the number of observations in the $k$th group of tied ranks (out of $K$ groups of tied ranks)
	\end{itemize}
\end{frame}

\begin{frame}{Example}
	\begin{itemize}
		\item You want to test whether they each have the same distribution or not
		\begin{center}
			\includegraphics[width=.9\linewidth]{GE}
		\end{center}
	\end{itemize}
\end{frame}

\begin{frame}{Example}
	\begin{itemize}
		\item You want to test whether they each have the same distribution or not
		\begin{center}
			\includegraphics[width=.6\linewidth]{GEData}
		\end{center}
	\end{itemize}
\end{frame}

\begin{frame}{Example Kruskal-Wallis test}{Set up}
	\begin{itemize}
		\item Let's determine if there is evidence that the distribution of gene expression is not all the same across the tissue types with $\alpha = 0.05$
		\item Hypotheses:
		\begin{itemize}
			\item $H_0:$ The $p$ distributions are identical
			\item $H_a:$ Not all the distributions are the same
		\end{itemize}
		\item[]
		\item Rejection rule: Reject $H_0$ if $H$ exceeds the critical value of $\chi^2 = 7.815$ for $\alpha = 0.05$ and $\text{df} = p - 1 = 3$
	\end{itemize}
\end{frame}

\begin{frame}{Example Kruskal-Wallis test}{Test statistic}
	\begin{itemize}
		\item The sum of ranks in each group:
		\begin{center}
			\begin{tabular}{|c|c|c|c|}
				\hline
				\textbf{Tissue} &  $T_j$ & $n_j$ & $T_j^2 / n_j$  \\ \hline \hline  
				Brain & 135 & 10 & 1822.5 \\ \hline 
				Heart & 223 & 10 & 4972.9\\ \hline 
				Kidney & 279 & 10 & 7784.1\\ \hline 
				Lung & 183 & 10 & 3348.9\\ \hline 
			\end{tabular}
		\end{center}
		\item Test statistic:
		\begin{align*}
			H &= \frac{12}{N (N+1)} \sum_{j=1}^p \frac{T_j^2}{n_j} - 3(N + 1) \\
			&= \frac{12}{40(41)} (1822.5 + 4972.9 + 7784.1 + 3348.9) - 3 (41) \\
			&= 8.183
		\end{align*}
	\end{itemize}
\end{frame}

\begin{frame}{Example Kruskal-Wallis test}{Test conclusion}
	\begin{itemize}
		\item We have that $H = 8.183 > 7.815$, so we reject $H_0$
		\item[]
		\item We have evidence that not all of the gene expression distributions are the same across tissue types
		\item[]
		\item The question remains which pairs of groups / populations have different distributions? Using the notation $y \myeq x + \Delta$, which $x$ and $y$ have $\Delta \neq 0$?
		\begin{itemize}
			\item The book presents a multiple comparison procedure for the Kruskal-Wallis test in 9.7
		\end{itemize}
	\end{itemize}
\end{frame}


\end{document}