\documentclass[xcolor=dvipsnames]{beamer} 
\usetheme{AnnArbor}
\usecolortheme{beaver}

\usepackage{amsmath,graphicx,booktabs,tikz,subfig,color,lmodern}
\definecolor{mycol}{rgb}{.4,.85,1}
\setbeamercolor{title}{bg=mycol,fg=black} 
\setbeamercolor{palette primary}{use=structure,fg=white,bg=red}
\setbeamercolor{block title}{fg=white,bg=red!50!black}
% \setbeamercolor{block title}{fg=white,bg=blue!75!black}

\title[Lecture 1]{Lecture 1: Course introduction \& What is statistical inference?}
\author[Patrick Trainor]{Patrick Trainor, PhD, MS, MA}
\institute[NMSU]{New Mexico State University}
\date{January 22, 2019}
\begin{document}
	
\begin{frame}
    \maketitle
\end{frame}

\begin{frame}{What is statistics?}
	\begin{itemize}
		\item ``\textbf{\textit{Statistics}} is the science of designing studies or experiments, collecting data, and modeling/analyzing data for the purpose of decision making and scientific discovery when the available information is both limited and variable''
	\end{itemize}
\end{frame}

\begin{frame}{What is statistics?}
	\vspace{-12pt}
	The textbook has a very nice diagram showing the relationship between statistics and the scientific method: \vspace{5pt}
	
	\begin{center}
		\includegraphics[scale = .22]{ScientificMethod}
	\end{center}
\end{frame}

\begin{frame}{Statistics \& The Scientific method}
	\vspace{-12pt}
	\begin{center}
		\includegraphics[scale = .22]{ScientificMethod2}
	\end{center}
\end{frame}


\begin{frame}{Statistics \& The Scientific method}{Formulating research goals}
\begin{itemize}
	\item All research starts with a question \pause
	\item[]
	\item Example questions: 
	\begin{itemize}
		\item Does quitting smoking lead to weight gain?\pause
		\item Is compound X carcinogenic in humans? \pause
		\item Do most voters support candidate X? \pause
	\end{itemize}
	\item[]
	\item Research questions can then be reformulated as \emph{testable} hypotheses \pause
	\begin{itemize}
		\item Quitting smoking is related to weight gain. Quitting smoking is not related to weight gain \pause
		\item Compound X is related to the presence of cancer in humans (perhaps the development of malignant tumors). Compound X is not related the presence of cancer in humans \pause
		\item Most voters do not support candidate X. Most voters do support candidate X
	\end{itemize}
\end{itemize}
\end{frame}

\begin{frame}{Statistics \& The Scientific method}{Formulating research goals}
	\vspace{-12pt}
	{\Huge Discussion: What research question(s) do you have?}
\end{frame}

\begin{frame}{Statistics \& The Scientific method}{Formulating research goals}
	\vspace{-12pt}
	{\Huge Discussion: How can your research question be formulated as a hypothesis / hypotheses?}
\end{frame}

\begin{frame}{Statistics \& The Scientific method}
\vspace{-12pt}
\begin{center}
	\includegraphics[scale = .22]{ScientificMethod3}
\end{center}
\end{frame}

\begin{frame}{Statistics \& The Scientific method}{Formulating research goals}
	\begin{itemize}
		\item There are multiple types of studies (or experiments), which we will discuss in this course \pause
		\item[]
		\item If we want to test the hypothesis that quitting is related to weight gain: \pause
		\begin{itemize}
			\item We could survey individuals who quit smoking and ask how much weight they gained / lost during some time period of smoking cessation* \pause
			\item We could review the medical records of individuals enrolled in a medically supervised smoking cessation program and record weight loss / weight gain** \pause
			\item We could expose genetically inbred mice on a fixed diet to smoking vapors for a period, cease the exposure, and record weight loss / weight gain** \pause
		\end{itemize}
	\end{itemize}
{\tiny *s denote the degree of ethical risk (my appraisal)}
\end{frame}

\begin{frame}{Statistics \& The Scientific method}{Designing a study}
	\begin{itemize}
		\item If we want to test the hypothesis that quitting is related to weight gain:
		\begin{itemize}
			\item We could survey individuals who quit smoking and ask how much weight they gained / lost during some time period of smoking cessation*
			\item We could review the medical records of individuals enrolled in a medically supervised smoking cessation program and record weight loss / weight gain**
			\item We could expose genetically inbred mice on a fixed diet to smoking vapors for a period, cease the exposure, and record weight loss / weight gain**
			\item We could recruit a group of individuals who smoke and ask half to quit smoking and ask half to continue smoking and record weight loss / weight gain over an identical time period***
		\end{itemize}
	\end{itemize}
{\tiny *s denote the degree of ethical risk (my appraisal)}
\end{frame}

\begin{frame}{Statistics \& The Scientific method}{Designing a study}
	\vspace{-12pt}
	\begin{itemize}
		\item Each of those studies / experiments has a very different design, as well as different benefits and limitations  \pause
		\item[] 
		\item In the next lecture we will discuss (a) the difference between a study and an experiment (b) some common types of experiments and studies
	\end{itemize}
\end{frame}

\begin{frame}{Statistics \& The Scientific method}{Designing a study}
	\vspace{-12pt}
	Three critical design attributes we must define when designing a study or an experiment: \pause
	\begin{enumerate}
		\item Experimental unit (study unit) \pause
		\item[]
		\item Variables of interest \pause
		\item[]
		\item Sampling mechanism \& sample size
	\end{enumerate}
\end{frame}

\begin{frame}{Statistics \& The Scientific method}{Designing a study}
	\vspace{-12pt}
	\begin{itemize}
		\item The \textbf{\emph{experimental unit}} or \textbf{\emph{study unit}} is often natural to the research question \pause
		\item[]
		\item If we want to test the hypothesis that quitting smoking is related to weight gain it is likely that our \emph{study units} would be individual humans who have recently quit smoking \pause
		\item[]
		\item \textbf{\emph{Variables of interest}} are attributes that may vary from study unit to study unit that we need to measure in order to test a hypothesis \pause
		\item[]
		\item If we want to test the hypothesis that quitting smoking is related to weight gain our \emph{variables of interest} might include: weight loss / gain of each individual human or BMI
	\end{itemize}
\end{frame}

\begin{frame}{Statistics \& The Scientific method}{Designing a study}
	\vspace{-12pt}
	\begin{itemize}
		\item Problem: we are severely limited in how much data we can collect and how many measurements of variables we can make to answer research questions \pause
		\item[]
		\item To test the hypothesis that quitting smoking is related to weight gain, we could survey individuals who quit smoking and ask how much weight they gained / lost during some time period of smoking cessation \pause
		\item[]
		\item Can we survey every person who ever smoked and then quit smoking (past and present?) and record their weight gain / loss after they quit smoking?
	\end{itemize}
\end{frame}

\begin{frame}{Statistics \& The Scientific method}{Designing a study}
	\begin{itemize}
		\item Can we survey every person who ever smoked and then quit smoking (past and present?) and record their weight gain / loss after they quit smoking? \pause
		\item[]
		\item Can we review every medical record of every individual who participated in a smoking cessation program and record their weight gain or loss? \pause
		\item[]
		\item We must determine what is the ``population'' we are interested in and we must ``sample'' from it (take a smaller subset)
	\end{itemize}
\end{frame}

\begin{frame}{Statistics \& The Scientific method}{Designing a study}
	\vspace{-12pt}
	\begin{center}
		\includegraphics[scale = .4]{Sampling}
	\end{center}
\end{frame}

\begin{frame}{Statistics \& The Scientific method}{Designing a study}
	\begin{columns}
		\begin{column}{0.5 \textwidth}
			\includegraphics[scale = .3]{Sampling}
		\end{column}
		\begin{column}{0.5 \textwidth}
			\begin{itemize}
				\item \textbf{\emph{Population:}} a natural, geographical, or political collection of people, animals, plants, or objects \pause
				\item A population is the subject of interest for answering a research question \pause
				\item Example: People who quit smoking is the population that is of interest for our research question ``is quitting smoking related to weight gain?''
			\end{itemize}
		\end{column}
	\end{columns}
\end{frame}

\begin{frame}{Statistics \& The Scientific method}{Designing a study}
	\begin{columns}
		\begin{column}{0.5 \textwidth}
			\includegraphics[scale = .3]{Sampling}
		\end{column}
		\begin{column}{0.5 \textwidth}
			\begin{itemize}
				\item Example: All humans is the population that is of interest for our research question ``is compound X related to cancer in humans?'' \pause
				\item[]
				\item Example: All people who are going to vote in the election is the population that is of interest for our research question ``do most voters support candidate X?''
			\end{itemize}
		\end{column}
	\end{columns}
\end{frame}

\begin{frame}{Statistics \& The Scientific method}{Designing a study}
	\begin{columns}
		\begin{column}{0.5 \textwidth}
			\includegraphics[scale = .3]{Sampling}
		\end{column}
		\begin{column}{0.5 \textwidth}
			\begin{itemize}
				\item \textbf{\emph{Sample:}} A subset of the population \pause
				\item[]
				\item We want a sample (a smaller subset of the population) to learn about a population \pause
				\item[]
				\item Example: A survey of 500 people who quit smoking is a sample from the population people who quit smoking
			\end{itemize}
		\end{column}
	\end{columns}
\end{frame}

\begin{frame}{Statistics \& The Scientific method}{Designing a study}
	\begin{columns}
		\begin{column}{0.5 \textwidth}
			\includegraphics[scale = .3]{Sampling}
		\end{column}
		\begin{column}{0.5 \textwidth}
			\begin{itemize}
				\item Example: A survey of 4,000 voters is a sample from the population all people who are going to vote in the election \pause
				\item[]
				\item Samples from a population don't always exactly match the population--this will be important later
			\end{itemize}
		\end{column}
	\end{columns}
\end{frame}

\begin{frame}{Statistics \& The Scientific method}{Designing a study}
	\begin{itemize}
		\item We make measurements of the variables of interest in the sample to learn something about the population \pause
		\item[]
		\item \textbf{\emph{Statistical inference:}} the theory, methods, and practice of forming judgments about the [characteristics] of a population ... typically on the basis of random sampling
	\end{itemize}
\end{frame}

\begin{frame}{Statistics \& The Scientific method}
	\vspace{-12pt}
	\begin{center}
		\includegraphics[scale = .22]{ScientificMethod4}
	\end{center}
\end{frame}

\begin{frame}{Statistics \& The Scientific method}{Collecting data}
	\begin{itemize}
		\item The method of data collection depends on the study design / design of experiment \pause
		\item[]
		\item Example. To test our hypothesis that smoking cessation is related to weight gain: we could survey individuals who quit smoking and ask how much weight they gained / lost during some time period of smoking cessation \pause
		\item[]
		\item Data collection: Conduct the survey (make phone calls, etc), record the responses
	\end{itemize}
\end{frame}

\begin{frame}{Statistics \& The Scientific method}{Collecting data}
	\begin{itemize}
		\item There are many important ethical considerations in the collection of data (how it is collected, how it is stored \& protected) \pause
		\item[]
		\item Even seemingly benign data collection can have ethical implications: \pause
		\begin{itemize}
			\item Would you want the world to know that you are in a smoking cessation program? \pause
			\item Would you want the world to know which political candidate you prefer? \pause
			\item Would you want the world to know your telephone number?
		\end{itemize}
	\end{itemize}
\end{frame}

\begin{frame}{Statistics \& The Scientific method}
	\vspace{-12pt}
	\begin{center}
		\includegraphics[scale = .22]{ScientificMethod5}
	\end{center}
\end{frame}

\begin{frame}{Statistics \& The Scientific method}{Drawing inferences}
	\begin{itemize}
		\item An \textbf{\emph{inference}} is a conclusion about a population that we make from measurements made from the sample  \pause
		\item[]
		\item This is the subject of this course! We will learn how we make inferences about a population from measurements we make in a sample \pause
		\item[]
		\item Example inference: From our survey of 500 people who quit smoking (our sample) we find that the average weight change was +3 lbs at two months after having quit smoking. Our inference then is that within the population (all people who quit smoking) people gain about 3 lbs after quitting smoking 
	\end{itemize}
\end{frame}

\begin{frame}{Statistics \& The Scientific method}{Drawing inferences}
	\begin{itemize}
		\item A critical aspect of ``statistical inference'' is conveying the reliability of an inference \pause
		\item[]
		\item If we find that average weight change is +3 lbs from our sample, this value is hopefully close to the average weight change in the population \pause
		\item[]
		\item We can use mathematics to determine the reliability of such an inference. We can determine mathematically if this +3 lbs number from the sample is likely to be close to the number for the population \pause
		\item[]
		\item We will discuss measures of reliability soon, however you may recognize some already: 63\% of voters support proposition X with a margin of error of +/- 3\%
	\end{itemize}
\end{frame}

\begin{frame}{The end of Lecture \#1!}
	\begin{center}
		\includegraphics[width=.45 \linewidth]{horse}
	\end{center}
\end{frame}

\end{document}
