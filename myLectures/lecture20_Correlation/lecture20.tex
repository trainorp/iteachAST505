\documentclass[xcolor=dvipsnames]{beamer}
\usetheme{AnnArbor}
\usecolortheme{beaver}

\usepackage{amsmath,graphicx,booktabs,tikz,subfig,color,lmodern}
\definecolor{mycol}{rgb}{.4,.85,1}
\setbeamercolor{title}{bg=mycol,fg=black} 
\setbeamercolor{palette primary}{use=structure,fg=white,bg=red}
\setbeamercolor{block title}{fg=white,bg=red!50!black}
% \setbeamercolor{block title}{fg=white,bg=blue!75!black}

\newcommand\myeq{\mathrel{\overset{\makebox[0pt]{\mbox{\normalfont\tiny\sffamily D}}}{=}}}

\title[Lecture 20]{Lecture 20. Linear Regression: Correlation}
\author[Patrick Trainor]{Patrick Trainor, PhD, MS, MA}
\institute[NMSU]{New Mexico State University}
\date{May 5, 2020}

\begin{document}
\begin{frame}
\maketitle
\end{frame}

\begin{frame}{Outline}
\tableofcontents[hideallsubsections]
\end{frame}

\section{The Pearson correlation coefficient}
\begin{frame}{Outline}
	\tableofcontents[currentsection,subsectionstyle=show/shaded/hide]
\end{frame}

\begin{frame}{The correlation coefficient}{A blast from the past}
	\begin{itemize}
		\item \textbf{\emph{Correlation coefficient:}} The correlation coefficient measures the strength of the linear relationship between two quantitative variables. We denote it $r$ \pause
		\item[]
		\item Old formula: \pause
		\begin{gather*}
		r = \frac{\sum_{i=1}^n(x_i -\bar{x})(y_i - \bar{y})}{(n-1) s_x s_y}
		\end{gather*}
	\end{itemize}
\end{frame}

\begin{frame}{The correlation coefficient}
	\begin{center}
		\includegraphics[width=0.75\linewidth]{../lecture4_DescriptiveStatistics2/scatter}
	\end{center}
\end{frame}

\begin{frame}{The correlation coefficient}{A blast from the past}
	\begin{itemize}
		\item \textbf{\emph{Correlation coefficient:}} The correlation coefficient measures the strength of the linear relationship between two quantitative variables. We denote it $r$
		\begin{itemize}
					\item Old formula: 
			\begin{gather*}
			r = \frac{\sum_{i=1}^n(x_i -\bar{x})(y_i - \bar{y})}{(n-1) s_x s_y}
			\end{gather*} \pause
			\item New formula (relating to linear regression):
			\begin{gather*}
			r_{yx} = \frac{\sum_{i = 1}^n (x_i - \bar{x})(y_i - \bar{y})}{\sqrt{S_{xx}S_{yy}}} = \frac{S_{xy}}{\sqrt{S_{xx}S_{yy}}}
			\end{gather*} \pause
			\item The better $x$ predicts $y$, given $\hat{y}=\hat{\beta}_0 + \hat{\beta}_1 x$, the greater $r_{yx}$
		\end{itemize}
	\end{itemize}
\end{frame}

\begin{frame}{The correlation coefficient}{Example}
	\begin{center}
		\includegraphics[width=.75\linewidth]{../lecture18_LinearRegression/p9Beta}
	\end{center}
\end{frame}

\begin{frame}{The correlation coefficient}{Example}
	\begin{itemize}
	\item In our example of predicting blood pressure from cortisol: \pause
	\begin{itemize}
		\item $S_{xx} = 36.1084$ \pause
		\item $S_{xy} = 75.376$ \pause
		\item $S_{yy} = \text{SS}(\text{Total}) = 233.3806$ \pause
	\end{itemize}
	\item[]
		\item So then: \pause
		\begin{gather*}
		r_{yx} = \frac{S_{xy}}{\sqrt{S_{xx}S_{yy}}} = \frac{75.376}{\sqrt{(36.1084)(233.3806)}} = 0.8211
		\end{gather*}
	\end{itemize}
\end{frame}

\begin{frame}{The correlation coefficient}{Nonlinear}
	\begin{center}
		\includegraphics[width = .65 \linewidth]{../lecture18_LinearRegression/noCorr}
		
		$r = 0.04$
	\end{center}
\end{frame}

\section{The coefficient of determination}
\begin{frame}{Outline}
	\tableofcontents[currentsection,subsectionstyle=show/shaded/hide]
\end{frame}

\begin{frame}{The coefficient of determination}
	\begin{itemize}
		\item The coefficient of determination: \pause
		\begin{gather*}
		r^2_{yx} = \frac{\text{SS}(\text{Total})-\text{SS}(\text{Error})}{\text{SS}(\text{Total})}
		\end{gather*} \pause
		\item The correlation coefficient is the square root of the coefficient of determination
	\end{itemize}
\end{frame}

\begin{frame}{The coefficient of determination}
	\begin{itemize}
		\item Back to our example of predicting blood pressure from cortisol, we had the following sum of squares: \pause
		\begin{itemize}
			\item Error or residual sum of squares: $\sum_{i=1}^n (y_i - \hat{y}_i)^2=76.034$ \pause
			\item[]
			\item Model or regression sum of squares: $\sum_{i=1}^n (\hat{y}_i-\bar{y})^2=157.347$ \pause
			\item[]
			\item Total sum of squares: $\sum_{i=1}^n (y_i -\bar{y})^2 = \text{SS}(\text{Regression})+\text{SS}(\text{Error}) = 233.3806$ \pause
			\item[]
		\end{itemize}
		\item The coefficient of determination:
		\begin{gather*}
		r^2_{yx} = \frac{\text{SS}(\text{Total})-\text{SS}(\text{Error})}{\text{SS}(\text{Total})} = \frac{233.3806 - 76.034}{233.3806} = 0.6742 
		\end{gather*}
	\end{itemize}
\end{frame}

\begin{frame}{The coefficient of determination}{Important interpretation}
	\begin{itemize}
		\item \textbf{Intercept only model}: The model $y = \beta_0 + \varepsilon$ \pause
		\begin{itemize}
			\item Corresponds to $\hat{y} = \bar{y}$ \pause
			\item[]
		\end{itemize}
		\item The coefficient of determination represents the proportionate reduction in error in using the model $\hat{y}=\hat{\beta}_0 + \hat{\beta}_1 x$ instead of $\hat{y}=\hat{\beta}_0 =\bar{y}$ \pause
		\item[]
		\item In our example of predicting blood pressure from cortisol, there is a 67.4\% reduction in the error in predicting $\hat{y}$ by using $\hat{y}=\hat{\beta}_0 + \hat{\beta}_1 x$ instead of $\hat{y} = \bar{y}$
	\end{itemize}
\end{frame}


\section{Correlation inference}
\begin{frame}{Outline}
	\tableofcontents[currentsection,subsectionstyle=show/shaded/hide]
\end{frame}

\begin{frame}{Correlation inference}
	\begin{itemize}
		\item The correlation coefficient $r_{yx}$ is an estimate determined from a sample of the population parameter $\rho_{yx}$ \pause
		\item[]
		\item We can test hypotheses regarding the value of $\rho_{yx}$
	\end{itemize}
\end{frame}

\begin{frame}{Correlation inference}{Process}
	\begin{itemize}
	\item Hypotheses: \pause
	\begin{itemize}
		\item \textbf{Case 1.} $H_0: \rho_{yx} \leq 0$ versus $H_a: \rho_{yx} > 0$ \pause
		\item \textbf{Case 2.} $H_0: \rho_{yx} \geq 0$ versus $H_a: \rho_{yx} < 0$ \pause
		\item \textbf{Case 3.} $H_0: \rho_{yx} = 0$ versus $H_a: \rho_{yx} \neq 0$ \pause
	\end{itemize}
	\item[]
	\item Test statistic:
	\begin{gather*}
	t^* = r_{yx} \frac{\sqrt{n-2}}{\sqrt{1-r^2_{yx}}}
	\end{gather*} \pause
	
	\item RR: For a fixed Type I error probability $\alpha$ and $\text{df}=n-2$, \pause
	\begin{itemize}
		\item \textbf{Case 1.} Reject $H_0$ if $t^* > t_{\alpha}$ with p-value: $P(t\geq t^*)$ \pause
		\item \textbf{Case 2.} Reject $H_0$ if $t^* < -t_{\alpha}$ with p-value: $P(t\leq t^*)$ \pause
		\item \textbf{Case 3.} Reject $H_0$ if $|t^*| > t_{\alpha / 2}$ with p-value: $2\times P(t\geq |t^*|)$
	\end{itemize}
	\end{itemize}
\end{frame}

\begin{frame}{Correlation inference}{Example}
	\begin{itemize}
		\item Research question, ``is there a positive correlation between cortisol levels and blood pressure?''. Let's test with $\alpha = 0.05$ \pause
		\item[]
		\item Hypotheses: \textbf{Case 1.} $H_0: \rho_{yx} \leq 0$ versus $H_a: \rho_{yx} > 0$\pause
		\item[]
		\item Test statistic: \pause
		\begin{gather*}
		t^* = r_{yx} \frac{\sqrt{n-2}}{\sqrt{1-r^2_{yx}}} = 0.8211 \frac{\sqrt{6}}{\sqrt{1-0.6742}} = 3.5237
		\end{gather*} \pause
		
		\item Rejection rule: Reject $H_0$ if $t^* > t_{\alpha} = 1.943$ \pause
		\item[]
		\item Conclusion. We do have evidence of a positive correlation. The p-value is $P(t\geq 3.5237)=0.006$
	\end{itemize}
\end{frame}

\begin{frame}{Correlation inference}{More on the relationship with linear regression}
	\begin{itemize}
		\item The observed test statistic from the correlation test matches the test statistic from the test of $H_0: \beta_1 \leq 0$ versus $H_a: \beta_1 >0$ \pause
		\item[]
		\item In general: $\hat{\beta}_1 = r_{yx} \sqrt{\frac{S_{yy}}{S_{xx}}}$
	\end{itemize}
\end{frame}

\begin{frame}{Correlation coefficient}{Confidence intervals}
	\begin{itemize}
		\item A confidence interval can be determined for a correlation coefficient given that $n$ is ``fairly large'' and $(x,y)$ is bivariate normal \pause
		\item[]
		\item A $100(1-\alpha)\%$ confidence interval for $\rho_{yx}$ is given by: \pause
		\begin{gather*}
		\left(\frac{e^{2 z_1}-1}{e^{2 z_1}+1}, \frac{e^{2 z_2}-1}{e^{2 z_2}+1} \right)
		\end{gather*} \pause
		where:
		\begin{gather*}
		z = \frac{1}{2} \ln \left( \frac{1+r_{yx}}{1-r_{yx}} \right) \\
		z_1 = z - \frac{z_{\alpha / 2}}{\sqrt{n-3}} \\
		z_2 = z + \frac{z_{\alpha / 2}}{\sqrt{n-3}}
		\end{gather*}
	\end{itemize}
\end{frame}

\begin{frame}{The correlation coefficient}{Confidence interval example}
	\begin{center}
		\includegraphics[width=.75\linewidth]{../lecture18_LinearRegression/p9Beta}
	\end{center}
\end{frame}

\begin{frame}{Correlation coefficient}{Confidence interval example}
	\begin{itemize}
		\item Let's determine a 90\% confidence interval for the correlation between cortisol levels and blood pressure:
		\begin{gather*}
		z = \frac{1}{2} \ln \left( \frac{1+r_{yx}}{1-r_{yx}} \right) =  \frac{1}{2} \ln \left( \frac{1+0.8211}{1-0.8211} \right) = 1.160185\\
		z_1 = z - \frac{z_{\alpha / 2}}{\sqrt{n-3}} = 1.160185 - \frac{1.645}{\sqrt{5}} = 0.42452 \\
		z_2 = z + \frac{z_{\alpha / 2}}{\sqrt{n-3}} = 1.160185 + \frac{1.645}{\sqrt{5}} = 1.89585
		\end{gather*}
		Then: 
		\begin{gather*}
		\left(\frac{e^{2 z_1}-1}{e^{2 z_1}+1}, \frac{e^{2 z_2}-1}{e^{2 z_2}+1} \right) = (0.4008, 0.9559)
		\end{gather*}

	\end{itemize}
\end{frame}

\section{Statistical significance versus practical significance}
\begin{frame}{Outline}
	\tableofcontents[currentsection,subsectionstyle=show/shaded/hide]
\end{frame}

\begin{frame}{Statistical significance versus practical significance}
	\begin{itemize}
		\item Research question: Is there is a correlation between the expression (or amount) of Gene 1 and Gene 2 (at $\alpha = 0.05$)? \pause
		\item[]
		\item Hypotheses: \textbf{Case 3.} $H_0: \rho_{yx} = 0$ versus $H_a: \rho_{yx} \neq 0$ \pause
		\item[]
		\item Rejection rule: Reject $H_0$ if $|t^*| > t_{\alpha / 2}$ with p-value: $2\times P(t\geq |t^*|)$ \pause
		\item[]
		\item Approach: You measure the expression of both genes with a sample size of $n = 2000$ and observe the data on the following slide
	\end{itemize}
\end{frame}

\begin{frame}{Statistical significance versus practical significance}
	\begin{center}
		\includegraphics[width = .7\linewidth]{../lecture18_LinearRegression/noCorr3}
	\end{center}
\end{frame}

\begin{frame}{Statistical significance versus practical significance}
	\begin{itemize}
	\item You determine that $r_{yx} = 0.06548$ \pause
	\item[]
	\item Test statistic: \pause
	\begin{gather*}
		t^* = r_{yx} \frac{\sqrt{n-2}}{\sqrt{1-r^2_{yx}}} = 0.06548 \frac{\sqrt{1998}}{\sqrt{1-(0.06548)^2}} = 2.9332
	\end{gather*} \pause
	\item Conclusion: Reject $H_0$, we have evidence that there is a correlation, with a p-value of 0.0034 \pause
	\item[]
	\item However, $ r_{yx}^2 = 0.0042$. So the proportional reduction in the error in predicting Gene 2 expression using Gene 1 expression is 0.42\% 
	\end{itemize}
\end{frame}

\begin{frame}{Statistical significance versus practical significance}
	\begin{center}
		\includegraphics[width = .7\linewidth]{../lecture18_LinearRegression/noCorr4}
	\end{center}
\end{frame}

\begin{frame}{Statistical significance versus practical significance}
	\begin{itemize}
		\item Recall the residual standard error: \pause
		\begin{gather*}
		s_{\varepsilon} = \sqrt{s^2_{\varepsilon}} = \sqrt{\frac{\sum_{i=1}^{n}\left(y_i - \hat{y}_i\right)^2}{n-2}}
		\end{gather*} \pause
		\item The residual standard error from predicting Gene 2 expression from Gene 1 using $\hat{y} = \hat{\beta}_0 + \hat{\beta}_1 x$, is 0.9849 \pause
		\item[]
		\item The residual standard error from predicting Gene 2 expression using the intercept only model $\hat{y}=\hat{\beta}_0 = \bar{y}$, is 0.9868
	\end{itemize}
\end{frame}

\section{The Spearman rank correlation coefficient}
\begin{frame}{Outline}
	\tableofcontents[currentsection,subsectionstyle=show/shaded/hide]
\end{frame}

\begin{frame}{The Spearman rank correlation coefficient}{Motivation}
	\begin{center}
		\includegraphics[width = .7 \linewidth]{../lecture18_LinearRegression/noCorr2}
	\end{center}
\end{frame}

\begin{frame}{The Spearman rank correlation coefficient}
	\begin{itemize}
		\item The Pearson correlation coefficient measures the relationship between two continuous random variables when: \pause
		\begin{itemize}
			\item $(x,y)$ are bivariate normal \pause
			\item The relationship between $x$ and $y$ is linear \pause
			\item[]
		\end{itemize}
		\item When these conditions do not hold, but the relationship between $x$ and $y$ is monotonic, then the Spearman rank correlation coefficient can be used to measure association \pause
		\item[]
		\item \textbf{Process:} Determine the ranks for the $x$ and $y$ values, and then determine the Pearson correlation coefficient between the ranks
	\end{itemize}
\end{frame}

\begin{frame}{The Spearman rank correlation coefficient}{Example}
	\begin{center}
		\includegraphics[width = .7 \linewidth]{../lecture18_LinearRegression/noCorr5}
	\end{center}
\end{frame}

\begin{frame}{The Spearman rank correlation coefficient}{Example}
\vspace{-5mm}
	\begin{center}
		\begin{tabular}{cc}
			\hline
			\textbf{Concentration} & \textbf{Rate} \\ 
			\hline \hline
			1.00 & 0.00274 \\ 
			2.00 & 0.00273 \\ 
			3.00 & 0.03109 \\ 
			4.00 & 0.07119 \\ 
			5.00 & 0.14157 \\ 
			6.00 & 0.34110 \\ 
			7.00 & 0.37684 \\ 
			8.00 & 0.83709 \\ 
			9.00 & 0.76581 \\ 
			10.00 & 0.95283 \\ 
			11.00 & 0.98664 \\ 
			12.00 & 0.98858 \\ 
			13.00 & 0.99784 \\ 
			14.00 & 0.99801 \\ \hline
		\end{tabular}
	\end{center}
\end{frame}


\begin{frame}{The Spearman rank correlation coefficient}{Example}
	\vspace{-5mm}
	\begin{center}
		\begin{tabular}{cccc}
			\hline
			\textbf{Concentration} & \textbf{Concentration Rank} & \textbf{Rate}  & \textbf{Rate Rank} \\ 
			\hline \hline
			1.00 & 1 &  0.00274 & 2 \\ 
			2.00 & 2 &0.00273 & 1 \\ 
			3.00 & 3 &0.03109  & 3\\ 
			4.00 & 4 &0.07119  & 4\\ 
			5.00 & 5 &0.14157  & 5\\ 
			6.00 & 6 &0.34110  & 6\\ 
			7.00 & 7 &0.37684  & 7\\ 
			8.00 & 8 &0.83709  & 8\\ 
			9.00 & 9 &0.76851  & 9\\ 
			10.00 & 10 & 0.95283 & 10 \\ 
			11.00 & 11 &0.98664  & 11\\ 
			12.00 & 12 &0.98858  & 12\\ 
			13.00 & 13 &0.99784  & 13\\ 
			14.00 & 14 &0.99801  & 14\\  \hline
		\end{tabular}
	\end{center}
\end{frame}

\begin{frame}{The Spearman rank correlation coefficient}{Example}
	\begin{itemize}
		\item Using the ranks from before: $S_{xy} = 225.5$, $S_{xx}=227.5$, and $S_{yy}=227.5$ \pause
		\item[]
		\item So then: \pause
		\begin{gather*}
			r_{s}=\frac{S_{xy}}{\sqrt{S_{xx}S_{yy}}} = \frac{225.5}{\sqrt{(227.5)(227.5)}} = 0.9912
		\end{gather*} 
		is the Spearman rank correlation coefficient \pause
		\item[]
		\item In comparison, the Pearson correlation coefficient is: 0.9544
	\end{itemize}
\end{frame}

\begin{frame}{The end of Lecture \#20}
	\begin{center}
		\includegraphics[width = .8\linewidth]{IMG_3403_v2}
	\end{center}
\end{frame}

\end{document}