\documentclass[xcolor=dvipsnames]{beamer}
\usetheme{AnnArbor}
\usecolortheme{beaver}

\usepackage{amsmath,graphicx,booktabs,tikz,subfig,color,lmodern}
\definecolor{mycol}{rgb}{.4,.85,1}
\setbeamercolor{title}{bg=mycol,fg=black} 
\setbeamercolor{palette primary}{use=structure,fg=white,bg=red}
\setbeamercolor{block title}{fg=white,bg=red!50!black}
% \setbeamercolor{block title}{fg=white,bg=blue!75!black}

\newcommand\myeq{\mathrel{\overset{\makebox[0pt]{\mbox{\normalfont\tiny\sffamily D}}}{=}}}

\title[Lecture 21]{Lecture 21. Linear Models (generally)}
\author[Patrick Trainor]{Patrick Trainor, PhD, MS, MA}
\institute[NMSU]{New Mexico State University}
\date{Decebmer, 2019}

\begin{document}
\begin{frame}
\maketitle
\end{frame}

\begin{frame}{Outline}
\tableofcontents[hideallsubsections]
\end{frame}

\section{``General linear models''}
\begin{frame}{Outline}
	\tableofcontents[currentsection,subsectionstyle=show/shaded/hide]
\end{frame}

\begin{frame}{``General linear models'' / Linear models}
	\begin{itemize}
		\item We started with the simple linear regression model $y = \beta_0 + \beta_1 x + \varepsilon$
		\begin{itemize}
			\item We can model the relationship between an independent variable $x$ and the expected value of a dependent variable $y$ with a straight line
			\item[]
		\end{itemize}
		\item Such a straight line relationship can be found in few real-world situations
		\item[]
		\item We can consider a broader class of linear models (General linear models)
	\end{itemize}
\end{frame}

\begin{frame}{``Generalized linear model'' / Linear models}
	\begin{itemize}
		\item \textbf{General linear model:} The general linear model has the form: $y = \beta_0 + \beta_1 x_1 + \beta_2 x_2 + \hdots + \beta_k x_k + \varepsilon$
		\item[]
		\item We can call a ``general linear model'', a ``linear model'' to avoid confusion with ``generalized linear models''
		\item[]
		\item We can use linear models to describe a wide range of relationships between variables
	\end{itemize}
\end{frame}

\begin{frame}{Linear models}
	content...
\end{frame}

\end{document}