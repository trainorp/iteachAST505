\documentclass[xcolor=dvipsnames]{beamer} 
\usetheme{AnnArbor}
\usecolortheme{beaver}

\usepackage{amsmath,graphicx,booktabs,tikz,subfig,color,lmodern}
\definecolor{mycol}{rgb}{.4,.85,1}
\setbeamercolor{title}{bg=mycol,fg=black} 
\setbeamercolor{palette primary}{use=structure,fg=white,bg=red}
\setbeamercolor{block title}{fg=white,bg=red!50!black}
% \setbeamercolor{block title}{fg=white,bg=blue!75!black}

\title[Lecture 2]{Lecture 2: Study Design / Design of Experiments}
\author[Patrick Trainor]{Patrick Trainor, PhD, MS, MA}
\institute[NMSU]{New Mexico State University}
\date{August 22, 2019}

\begin{document}

\begin{frame}
\maketitle
\end{frame}

\begin{frame}{Outline}
\tableofcontents[hideallsubsections]
\end{frame}

\section{Introduction}
\begin{frame}{Outline}
\tableofcontents[currentsection,subsectionstyle=show/shaded/hide]
\end{frame}

\begin{frame}{Introduction}
	\begin{itemize}
		\item \textbf{\emph{Observational study:}} In an observational study, the researcher records information, but does not in any way interfere with the process that is generating the information
		\item[]
		\item Examples:
		\begin{itemize}
			\item The Bureau of Labor Statistics sends personnel to supermarkets \& stores to record the prices of selected goods--the act of showing up and recording prices \emph{should not} interfere with the processes generating the data
			\item In the Multi-Ethnic Study of Atherosclerosis (MESA), $\approx$ 6000 men and women are followed to see who has heart attacks, strokes, death from coronary artery disease, etc.--the act of interviewing subjects each year (along with physical examination) should not interfere with who has a heart attack (or other event)
			\item A health insurer analyzes their medical claims data to determine attributes associated with members who have more expensive hip replacement surgeries than those who have less expensive surgeries
		\end{itemize}
	\end{itemize}
\end{frame}

\begin{frame}{Introduction}
	\begin{itemize}
		\item \textbf{\emph{Experimental study:}} In an experimental study, the researcher does manipulate \textbf{\emph{explanatory variables}}, and then records their effects on \textbf{\emph{response variables}}
		\item[]
		\item Examples:
		\begin{itemize}
			\item A city's housing agency distributes housing vouchers differently to two groups of families. One group has the stipulation that they can only use the vouchers in high SES neighborhoods. The other group does not have the stipulation (regular voucher). The city measures household income a year later
			\begin{itemize}
				\item Explanatory variable: An indicator of whether the family recieved the high SES neighborhood voucher versus regular voucher
				\item Response variable: Household income a year after voucher was recieved
				\item The response variable could be influenced by the researcher's manipulation 
			\end{itemize}
	\end{itemize}
\end{itemize}
\end{frame}

\begin{frame}{Introduction}
\begin{itemize}
	\item \textbf{\emph{Experimental study:}} In an experimental study, the researcher does manipulate \textbf{\emph{explanatory variables}}, and then records their effects on \textbf{\emph{response variables}}
	\item[]
	\item Examples:
	\begin{itemize}
		\item A cancer researcher uses CRISPR (a genetic engineering technique) to knock out a specific gene in a cancer cell line and records the effect on cell viability
		\begin{itemize}
			\item Explanatory variable: An indicator of whether the gene is expressed (present) or not-expressed (not present following CRISPR knockout) in the cell samples
			\item Response variable: Cell viability
			\item The response variable could be influenced by the researcher's manipulation 
		\end{itemize}
	\end{itemize}
\end{itemize}
\end{frame}

\section{Observational Studies}
\begin{frame}{Outline}
\tableofcontents[currentsection,subsectionstyle=show/shaded/hide]
\end{frame}

\begin{frame}{Objective of an Observational Study}
Observational studies can be categorized based on their objectives into descriptive studies and comparative studies
		\begin{itemize}
			\item \textbf{\emph{Descriptive studies:}} Descriptive studies seek to describe a process (which may not be well understood
			\item[]
			\item Examples:
			\begin{itemize}
				\item Determining the demographic composition of a public university like NMSU
				\item Determining if HPV genes are present in oral lessions from patients at a dental clinic
				\item Determining the number of cases of Rocky Mountain Spotted Fever in New Mexico this year
			\end{itemize}			
		\end{itemize}
\end{frame}

\begin{frame}{Objective of an Observational Study}
Observational studies can be categorized based on their objectives into descriptive studies and comparative studies
\begin{itemize}
\item \textbf{\emph{Comparative study:}} Comparative studies seek to compare an attribute or process between two or more groups
\item[]
\item Examples:
\begin{itemize}
	\item The university administration commissions a study of post-graduation incomes comparing western universities
	\item A health insurer compares ten year healthcare spend for extreme-BMI members who have bariatric surgery versus those who do not
\end{itemize}
\end{itemize}
\end{frame}

\subsection{Association versus Causation}
\begin{frame}{Outline}
\tableofcontents[currentsection,subsectionstyle=show/shaded/hide]
\end{frame}

\begin{frame}{Association versus Causation}
	\begin{itemize}
		\item Observational studies are often utilized to answer questions such as ``does smoking cause cancer''
		\item[]
		\item Unfortunately, from most observational study designs we cannot establish \textbf{\emph{causation}} (showing that one thing causes another thing) but rather can only show \textbf{\emph{association}}
		\item[]
		\item In the case of ``does smoking cause cancer'' many early studies could clearly demonstrate that smoking and cancer were associated but could not show causation
		\begin{itemize}
			\item Association: Incidence of cancer (especially lung cancer) was much higher in those who smoked than didn't smoke
		\end{itemize}
		\item[]
		\item Why can't most observational studies be utilized to show causation?
	\end{itemize}
\end{frame}

\begin{frame}{Association versus Causation}
	\begin{itemize}
		\item The response variable that a researcher measures in an observational study may be influenced by \textbf{\emph{confounding variables}}. These are variables that are not under the control of the researcher and are not explanatory variables
		\item[]
		\item Example: You want to know if sulfur dioxide from coal-fired power plants causes heart attacks. You study 1,000 individuals who have high sulfur dioxide exposure and 1,000 individuals who have low sulfur dioxide exposure and you find the rate of heart attacks is 10 times greater in those with high exposure.
	\end{itemize}
\end{frame}

\begin{frame}{Association versus Causation}
\begin{itemize}
	\item Example: You want to know if sulfur dioxide from coal-fired power plants causes heart attacks. You study 1,000 individuals who have high sulfur dioxide exposure and 1,000 individuals who have low sulfur dioxide exposure and you find the rate of heart attacks is 10 times greater in those with high exposure.
	\begin{center}
		\includegraphics[width=.9 \linewidth]{confounding}
	\end{center}
\end{itemize}
\end{frame}

\begin{frame}{Association versus Causation}
	\begin{center}
		\includegraphics[width=.9 \linewidth]{confounding2}
	\end{center}
\end{frame}

\begin{frame}{Association versus Causation}
	\begin{itemize}
		\item Possible confounding variables
		\begin{itemize}
			\item Neighborhoods near power plants are likely to be lower SES than neighborhoods that are not near power plants
			\item Many variables that are causally linked to heart attacks are related to SES including smoking, obesity, and diet
			\item[]
		\end{itemize}
		\item Establishing causation requires removing all possible confounding 
		\item[]
		\item There are statistical methods that help deal with confounding variables
		\item[]
		\item Experimental studies allow for isolating the relationship between an explanatory variable and a response variable but may not be ethically feasible 
	\end{itemize}
\end{frame}

\subsection{Types of Observational Studies}
\begin{frame}{Outline}
\tableofcontents[currentsection,subsectionstyle=show/shaded/hide]
\end{frame}

\begin{frame}{Types of Observational Studies}
	\begin{itemize}
		\item \textbf{\emph{Sample survey:}} A sample survey is a study that provides information about a population at a particular point in time
		\item[]
		\item \textbf{\emph{Prospective study:}} A prospective study often begins with a sample survey and then proceeds to follow the study subject in the sample in time to record the occurence of specific outcomes
		\item[]
		\item \textbf{\emph{Retrospective study:}} A retrospective study is a study that often starts with a sample survey in the present, and then collects information about the study subjects regarding specific outcomes that took place in the past
	\end{itemize}
\end{frame}

\begin{frame}{Types of Observational Studies:}
	\begin{itemize}
		\item Cohort studies are a type of prospective study used to determine factors that influence the propensity for an event to occur (such as the incidence of cancer, incidence of stroke, an arrest for a crime, divorce, etc)
		\item[]
		\item Case-control studies are a type of retrospective study in which a group of subjects are identified with a disease (or the occurrence of another event) and another group of subjects is identified that doesn't have the disease (or event). Information is then gathered about risk factors (this information is often retrospective)
	\end{itemize}
\end{frame}

\subsection{Surveys}
\begin{frame}{Outline}
\tableofcontents[currentsection,subsectionstyle=show/shaded/hide]
\end{frame}

\begin{frame}{Surveys}
	\begin{itemize}
		\item The textbook has a wealth of examples of government and media survey examples--please read on your own (p. 25)
		\item[]
		\item Terms important for survey sampling design:
		\begin{itemize}
			\item \textbf{\emph{Target population:}} The complete collection of ``objects'' whose description is the major goal of the study
			\item \textbf{\emph{Sample:}} A subset of the target population
			\item \textbf{\emph{Sampled population:}} The complete collection of objects that have the potential of being selected in the sample; the population from which the sample is actually selected
			\item \textbf{\emph{Observation unit:}} The object about which data are collected 
			\item \textbf{\emph{Sampling unit:}} The object that is actually sampled
			\item \textbf{\emph{Sampling frame:}} The list of sampling units
		\end{itemize}
	\end{itemize}
\end{frame}

\begin{frame}{Surveys}{Example}
	\begin{itemize}
		\item Example: Delta airlines wants to evaluate whether their zero free checked bags policy encourages people to fly on other airlines
		\item[]
		\item Potential target populations:
		\begin{itemize}
			\item Anyone potential passenger who has the option to take a Delta route to their destination
			\item Frequent travelers
			\item Managers at companies / government agencies who make travel decisions
		\end{itemize}
		\item[]
		\item Sample:
		\begin{itemize}
			\item Select 50 passengers on a specific recent flight 
			\item Select 500 members of their frequent flyer program who fly $>$20 times a year
		\end{itemize}
	\end{itemize}
\end{frame}

\begin{frame}{Surveys}{Example}
	\begin{itemize}
		\item Sampled population:
		\begin{itemize}
			\item 50 passengers on a specific recent flight would include members of the target population ``Anyone potential passenger who has the option to take a Delta route to their destination'', but would not be representative of that population
			\item Select 500 members of their frequent flyer program who fly
		\end{itemize}
	
	\item Let's assume Delta wants to use frequent travelers as the target population and will use members of their frequent flyer program with $>$20 flights per year to administer survey
	
	\item Observation unit / sampling unit: In this case they are the same. The observation unit would be the person in the sample
	
	\item Sampling frame: The sampling frame be Delta's list of frequent flyer program members with $>$20 flights per year
	\end{itemize}
\end{frame}

\begin{frame}{Surveys}{Observation unit versus sample unit}
\begin{itemize}
	\item  Entomology example. A forest manager might divide a forest into small geographic grids to survey the number of trees that have succumbed to a bark beetle epidemic 
\end{itemize}
	\begin{itemize}
		\item The sample unit: Geographic grids that will be selected to sample from
		\item[]
		\item Observation unit: Individual trees within the grid for which an observation is made (dead from bark beetle or not)
	\end{itemize}
\end{frame}

\begin{frame}{Sampling techniques}
	\begin{itemize}
		\item \textbf{\emph{Simple random sampling:}} From a sample frame you select $n$ objects, such that each possible sample of $n$ objects has the same chance of being selected
		
		\item[]
		
		\item \textbf{\emph{Stratified random sample:}} We divide a population into two (or more) groups known as strata and then sample from each group yielding a stratified random sample
		\begin{itemize}
			\item Sometimes done if certain strata would be underrepresented given simple random sampling
			\item Often utilized if there is a natural subgroups for which our responses will be very different which we need to account for
			\item Example. Suppose you want to estimate the presidential approval rate. You likely want to use stratified random sampling
		\end{itemize}
	\end{itemize}
\end{frame}

\begin{frame}{Sampling techniques}
\begin{itemize}
	\item \textbf{\emph{Cluster sampling:}} With cluster sampling we first randomly select clusters (or groups of objects) and then sample all objects within the group
	\item[]
	\item \textbf{\emph{Systematic sampling:}} Systematic sampling is any non-random sampling mechanism
	\begin{itemize}
		\item Example. In an exit poll, the pollster selects every fifth voter exiting the polling place
	\end{itemize}
\end{itemize}
\end{frame}

\begin{frame}{Sampling techniques}{Exercise}
What type of sampling is employed in the following examples?

{\scriptsize
	\begin{itemize}
		\item A forest manager is interested in bark beetle-induced mortality rates in pine trees. The manager randomly selects 10 grids from a map of the forest and then surveys each tree in the grid for dead trees
		\item[]
		\item NMSU wants to know how often students visit Zuhl library. NMSU uses a computer program to randomly select 200 student ID's of registered students and then emails these students a questionaire asking about how often they visit the library
		\item[]
		\item A vegetable processing plant selects every 250th head of lettuce to test for the presence of \emph{E. coli}
		\item[]
		\item NMSU wants to know how often students visit Zuhl library and knows that library habits may differ by student classification. NMSU uses a computer program to randomly select 100 undgraduate resident student IDs, 100 undergraduate commuter student IDs, and 100 graduate students IDs and then emails these students a questionaire asking about how often they visit the library

	\end{itemize}
}
\end{frame}

\begin{frame}{Issues with surveys (with human participants)}
	\begin{columns}
		\begin{column}{.4 \textwidth}
			\includegraphics[width=1\linewidth]{digest}
		\end{column}
		\begin{column}{.55 \textwidth}
			\begin{itemize}
				\item The sampled population is not representative of the target population
				
				\begin{itemize}
					\item In 1936, The Literary Digest forecasted that Alfred Landon would win the presidencey (predicting 57\% to 43\% in the popular vote). Franklin Delano Rosevelt actually won 46/48 states
					\item Literary Digest had polled 10 million individuals!
					\item Was biased by it's own readership having much higher incomes than the general population
				\end{itemize}
			\end{itemize}

		\end{column}
	\end{columns}
		
\end{frame}

\begin{frame}{Issues with surveys (with human participants)}
\begin{itemize}
	\item Survey nonresponse
	\begin{itemize}
		\item There are signifcant sex differences in response rates which may bias survey results
	\end{itemize}
	\item[]
	\item Measurement problems: respondents do not provide the information the survey seeks
	\begin{itemize}
		\item Poor recall 
		\item[]
		\item Leading questions. Example: ``Do you support Virgin Galactic choosing Las Cruces as their headquarters given that they will not provide new jobs for locals and will drive housing prices higher?''
		\item[]
		\item Unclear wording of questions / lack of precise definitions 
	\end{itemize}
\end{itemize}
\end{frame}

\section{Experimental Studies}
\begin{frame}{Outline}
\tableofcontents[currentsection,subsectionstyle=show/shaded/hide]
\end{frame}
\begin{frame}{Experimental Studies}
	\begin{itemize}
		\item Experimental studies allow for an isolation of effects by removing possible sources of confounding
		
		\item[]
		
		\item Example. We use inbred mice in cancer biology that have virtually the same genome, that are maintained in identical conditions. So if we knock out a tumor suppressor gene, and treat with a carcinogen, we can elucidate the effects of the gene knockout and carcinogen without interference from other factors (especially potential genetic variation).
		
	\end{itemize}
\end{frame}

\begin{frame}{Experimental Studies}{Some terminology}
	\begin{itemize}
			\item \textbf{\emph{Factors:}} Controlled variables that are selected by the researcher for comparison are called factors
		\begin{itemize}
			\item Factors are selected to allow for testing hypotheses formulated from the research question
		\end{itemize}
		\item[]
		\item \textbf{\emph{Response variables:}} Response variables are measurements and observations that are recorded but not controlled by the researcher
		\item[]
		\item \textbf{\emph{Treatments:}} The treatments in an experimental study are the conditions constructed from the factors
		\begin{itemize}
			\item When there is only one factor in an experiment the treatment and factor is the same
		\end{itemize}
	\end{itemize}
\end{frame}

\begin{frame}{Experimental Studies}{Terminology examples}
	\begin{itemize}
		\item Example 1. A researcher wants to know whether a new drug helps improve glucose tolerance in type 2 diabetes. The researcher uses a sample of 10 diabetic mice with nearly identical background (C57BLKS/J \emph{Lepr}\textsuperscript{db}) and administers the new drug to half of the mice. The researcher then measures the concentration of glucose in the blood of each mouse following a meal. 
		\begin{itemize}
			\item The only factor in the experiment (also the treatment) is whether or not the drug was administered to the experimental unit (a mouse). We would say the ``treatment'' is treatment with the new drug
			\item[]
			\item The response variable is the measurement of blood glucose from each mouse
		\end{itemize}
	\end{itemize}
\end{frame}

\begin{frame}{Experimental Studies}{Terminology examples}
\begin{itemize}
	\item Example 2. A researcher wants to know whether a new drug helps improve glucose tolerance in type 2 diabetes and if the hypothesized improvement is greater (or less) when combined with an exercise intervention. The researcher uses a sample of 20 diabetic mice with nearly identical background (C57BLKS/J \emph{Lepr}\textsuperscript{db}) and administers the new drug to half of the mice. Half of the mice in the group that were administered the drug and half of the mice that were not are subjected to daily exercise. The researcher then measures the concentration of glucose in the blood of each mouse following a meal. 
	\begin{itemize}
		\item The response variable is still the measurement of blood glucose from each mouse
		\item[]
		\item Now we have two factors (drug \{yes, no\}, and exercise \{yes, no\})
		\item[]
		\item The four different treatments are \{(drug = yes, exercise = yes), (drug = no, exercise = yes), (drug = yes, exercise = no), (drug = no, exercise = no) \}
	\end{itemize}
\end{itemize}
\end{frame}

\begin{frame}{Factorial designs}
	\begin{itemize}
		\item \textbf{\emph{Factor levels:}} The ``levels'' of a factor are the possible values it can take
		\begin{itemize}
			\item In example 1 and 2, the factor ``drug'' has two levels: yes and no
			\item In example 2, the factor ``excercise'' has two levels: yes and no
		\end{itemize}
		\item[]
		\item \textbf{\emph{Factorial treatment design:}} A treatment design in which all possible factor combinations are considered is a factorial treatment design
		\item[]
		\item To determine the number of different treatments in a full factorial treatment design you multiply the number of levels for each factor together
		\begin{itemize}
			\item In example 2, there were $2 \times 2 = 4$ different treatments
			\item[]
			\item Full factorial designs can become infeasible very quickly. Example: A design with 4 factors with 5 levels would have 625 different treatments
		\end{itemize}		
	\end{itemize}
\end{frame}

\end{document}