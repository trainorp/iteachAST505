\documentclass[xcolor=dvipsnames]{beamer} 
\usetheme{AnnArbor}
\usecolortheme{beaver}

\usepackage{amsmath,graphicx,booktabs,tikz,subfig,color,lmodern}
\definecolor{mycol}{rgb}{.4,.85,1}
\setbeamercolor{title}{bg=mycol,fg=black} 
\setbeamercolor{palette primary}{use=structure,fg=white,bg=red}
\setbeamercolor{block title}{fg=white,bg=red!50!black}
% \setbeamercolor{block title}{fg=white,bg=blue!75!black}

\title[Lecture 2]{Lecture 2: Study Design / Design of Experiments}
\author[Patrick Trainor]{Patrick Trainor, PhD, MS, MA}
\institute[NMSU]{New Mexico State University}
\date{August 22, 2019}

\begin{document}

\begin{frame}
\maketitle
\end{frame}

\begin{frame}{Outline}
\tableofcontents[hideallsubsections]
\end{frame}

\section{Introduction}
\begin{frame}{Outline}
\tableofcontents[currentsection,subsectionstyle=show/shaded/hide]
\end{frame}

\begin{frame}{Introduction}
	\begin{itemize}
		\item \textbf{\emph{Observational study:}} In an observational study, the researcher records information, but does not in any way interfere with the process that is generating the information
		\item[]
		\item Examples:
		\begin{itemize}
			\item The Bureau of Labor Statistics sends personnel to supermarkets \& stores to record the prices of selected goods--the act of showing up and recording prices \emph{should not} interfere with the processes generating the data
			\item In the Multi-Ethnic Study of Atherosclerosis (MESA), $\approx$ 6000 men and women are followed to see who has heart attacks, strokes, death from coronary artery disease, etc.--the act of interviewing subjects each year (along with physical examination) should not interfere with who has a heart attack (or other event)
			\item A health insurer analyzes their medical claims data to determine attributes associated with members who have more expensive hip replacement surgeries than those who have less expensive surgeries
		\end{itemize}
	\end{itemize}
\end{frame}

\begin{frame}{Introduction}
	\begin{itemize}
		\item \textbf{\emph{Experimental study:}} In an experimental study, the researcher does manipulate \textbf{\emph{explanatory variables}}, and then records their effects on \textbf{\emph{response variables}}
		\item[]
		\item Examples:
		\begin{itemize}
			\item A city's housing agency distributes housing vouchers differently to two groups of families. One group has the stipulation that they can only use the vouchers in high SES neighborhoods. The other group does not have the stipulation (regular voucher). The city measures household income a year later
			\begin{itemize}
				\item Explanatory variable: An indicator of whether the family recieved the high SES neighborhood voucher versus regular voucher
				\item Response variable: Household income a year after voucher was recieved
				\item The response variable could be influenced by the researcher's manipulation 
			\end{itemize}
	\end{itemize}
\end{itemize}
\end{frame}

\begin{frame}{Introduction}
\begin{itemize}
	\item \textbf{\emph{Experimental study:}} In an experimental study, the researcher does manipulate \textbf{\emph{explanatory variables}}, and then records their effects on \textbf{\emph{response variables}}
	\item[]
	\item Examples:
	\begin{itemize}
		\item A cancer researcher uses CRISPR (a genetic engineering technique) to knock out a specific gene in a cancer cell line and records the effect on cell viability
		\begin{itemize}
			\item Explanatory variable: An indicator of whether the gene is expressed (present) or not-expressed (not present following CRISPR knockout) in the cell samples
			\item Response variable: Cell viability
			\item The response variable could be influenced by the researcher's manipulation 
		\end{itemize}
	\end{itemize}
\end{itemize}
\end{frame}

\section{Observational Studies}
\begin{frame}{Outline}
\tableofcontents[currentsection,subsectionstyle=show/shaded/hide]
\end{frame}

\begin{frame}{Objective of an Observational Study}
Observational studies can be categorized based on their objectives into descriptive studies and comparative studies
		\begin{itemize}
			\item \textbf{\emph{Descriptive studies:}} Descriptive studies seek to describe a process (which may not be well understood
			\item[]
			\item Examples:
			\begin{itemize}
				\item Determining the demographic composition of a public university like NMSU
				\item Determining if HPV genes are present in oral lessions from patients at a dental clinic
				\item Determining the number of cases of Rocky Mountain Spotted Fever in New Mexico this year
			\end{itemize}			
		\end{itemize}
\end{frame}

\begin{frame}{Objective of an Observational Study}
Observational studies can be categorized based on their objectives into descriptive studies and comparative studies
\begin{itemize}
\item \textbf{\emph{Comparative study:}} Comparative studies seek to compare an attribute or process between two or more groups
\item[]
\item Examples:
\begin{itemize}
	\item The university administration commissions a study of post-graduation incomes comparing western universities
	\item A health insurer compares ten year healthcare spend for extreme-BMI members who have bariatric surgery versus those who do not
\end{itemize}
\end{itemize}
\end{frame}

\begin{frame}{Association versus Causation}
	\begin{itemize}
		\item Observational studies are often utilized to answer questions such as ``does smoking cause cancer''
		\item[]
		\item Unfortunately, from most observational study designs we cannot establish \textbf{\emph{causation}} (showing that one thing causes another thing) but rather can only show \textbf{\emph{association}}
		\item[]
		\item In the case of ``does smoking cause cancer'' many early studies could clearly demonstrate that smoking and cancer were associated but could not show causation
		\begin{itemize}
			\item Association: Incidence of cancer (especially lung cancer) was much higher in those who smoked than didn't smoke
		\end{itemize}
		\item[]
		\item Why can't most observational studies be utilized to show causation?
	\end{itemize}
\end{frame}

\begin{frame}{Association versus Causation}
	\begin{itemize}
		\item The response variable that a researcher measures in an observational study may be influenced by \textbf{\emph{confounding variables}}. These are variables that are not under the control of the researcher and are not explanatory variables
		\item[]
		\item Example: You want to know if sulfur dioxide from coal-fired power plants causes heart attacks. You study 1,000 individuals who have high sulfur dioxide exposure and 1,000 individuals who have low sulfur dioxide exposure and you find the rate of heart attacks is 10 times greater in those with high exposure.
	\end{itemize}
\end{frame}

\begin{frame}{Association versus Causation}
\begin{itemize}
	\item Example: You want to know if sulfur dioxide from coal-fired power plants causes heart attacks. You study 1,000 individuals who have high sulfur dioxide exposure and 1,000 individuals who have low sulfur dioxide exposure and you find the rate of heart attacks is 10 times greater in those with high exposure.
	\begin{center}
		\includegraphics[width=.9 \linewidth]{confounding}
	\end{center}
\end{itemize}
\end{frame}

\begin{frame}{Association versus Causation}
	\begin{center}
		\includegraphics[width=.9 \linewidth]{confounding2}
	\end{center}
\end{frame}

\begin{frame}{Association versus Causation}
	\begin{itemize}
		\item Possible confounding variables
		\begin{itemize}
			\item Neighborhoods near power plants are likely to be lower SES than neighborhoods that are not near power plants
			\item Many variables that are causally linked to heart attacks are related to SES including smoking, obesity, and diet
			\item[]
		\end{itemize}
		\item Establishing causation requires removing all possible confounding 
		\item[]
		\item There are statistical methods that help deal with confounding variables
		\item[]
		\item Experimental studies allow for isolating the relationship between an explanatory variable and a response variable but may not be ethically feasible 
	\end{itemize}
\end{frame}

\begin{frame}{Types of Observational Studies}
	\begin{itemize}
		\item \textbf{\emph{Sample survey:}} A sample survey is a study that provides information about a population at a particular point in time
		\item[]
		\item \textbf{\emph{Prospective study:}} A prospective study often begins with a sample survey and then proceeds to follow the study subject in the sample in time to record the occurence of specific outcomes
		\item[]
		\item \textbf{\emph{Retrospective study:}} A retrospective study is a study that often starts with a sample survey in the present, and then collects information about the study subjects regarding specific outcomes that took place in the past
	\end{itemize}
\end{frame}

\begin{frame}{Types of Observational Studies:}
	\begin{itemize}
		\item Cohort studies are a type of prospective study used to determine factors that influence the propensity for an event to occur (such as the incidence of cancer, incidence of stroke, an arrest for a crime, divorce, etc)
		\item[]
		\item Case-control studies are a type of retrospective study in which a group of subjects are identified with a disease (or the occurrence of another event) and another group of subjects is identified that doesn't have the disease (or event). Information is then gathered about risk factors (this information is often retrospective)
	\end{itemize}
\end{frame}

\end{document}