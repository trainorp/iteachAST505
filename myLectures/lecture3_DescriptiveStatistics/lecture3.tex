\documentclass[xcolor=dvipsnames]{beamer} 
\usetheme{AnnArbor}
\usecolortheme{beaver}

\usepackage{amsmath,graphicx,booktabs,tikz,subfig,color,lmodern}
\definecolor{mycol}{rgb}{.4,.85,1}
\setbeamercolor{title}{bg=mycol,fg=black} 
\setbeamercolor{palette primary}{use=structure,fg=white,bg=red}
\setbeamercolor{block title}{fg=white,bg=red!50!black}
% \setbeamercolor{block title}{fg=white,bg=blue!75!black}

\title[Lecture 3]{Lecture 3: Descriptive Statistics I}
\author[Patrick Trainor]{Patrick Trainor, PhD, MS, MA}
\institute[NMSU]{New Mexico State University}
\date{August 26, 2019}

\begin{document}
	
\begin{frame}
	\maketitle
\end{frame}

\begin{frame}{Outline}
\tableofcontents[hideallsubsections]
\end{frame}

\section{Descriptive vs. Inferential Statistics}
\begin{frame}{Outline}
\tableofcontents[currentsection,subsectionstyle=show/shaded/hide]
\end{frame}

\begin{frame}{Descriptive vs. Inferential Statistics}
\begin{columns}
	\begin{column}{0.5 \textwidth}
		\includegraphics[scale = .3]{../lecture1_Introduction/Sampling}
	\end{column}
	\begin{column}{0.5 \textwidth}
		\begin{itemize}
		\item In lecture \#1 we said that:
		\begin{itemize}
			\item \textbf{\emph{Statistical inference:}} the theory, methods, and practice of forming judgments about the [characteristics] of a population ... typically on the basis of random sampling
		\end{itemize}
		\item Statistical inference $\approx$ inferential statistics
		\item[]
		\item In chapter 3 of the text, the authors claim there are two kinds of statistical methods: inferential statistics \& descriptive statistics 
		\end{itemize}
	\end{column}
\end{columns}
\end{frame}


\begin{frame}{Descriptive vs. Inferential Statistics}
	\begin{itemize}
		\item In chapter 3 of the text, the authors claim there are two kinds of statistical methods: inferential statistics \& descriptive statistics
		\item[]
		\item Book example of descriptive statistics: 
		\begin{itemize}
			\item An HMO (a type of health insurance product) has complete data on all of the medical claims it has paid on behalf of its members / beneficiaries
			\item[]
			\item This is a rare case in which to make an conclusion about a population (the members / beneficiaries of the HMO) sampling is not needed
			\item[] 
			\item If you have a research question (Do 70-75 year olds have more medical claims than 20-25 year olds?), you can answer it without sampling
		\end{itemize}
	\end{itemize}
\end{frame}

\begin{frame}{Descriptive vs. Inferential Statistics}
\begin{itemize}
	\item In chapter 3 of the text, the authors claim there are two kinds of statistical methods: inferential statistics \& descriptive statistics
	\item[]
	\item Book example of descriptive statistics: 
	\begin{itemize}
		\item An HMO (a type of health insurance product) has complete data on all of the medical claims it has paid on behalf of its members / beneficiaries
		\item[]
		\item This is a rare case in which to make an conclusion about a population (the members / beneficiaries of the HMO) sampling is not needed
		\item[] 
		\item If you have a research question (Do 70-75 year olds have more medical claims than 20-25 year olds?), you can answer it without sampling
	\end{itemize}
\end{itemize}
\end{frame}

\begin{frame}{Exploratory data analysis}
	\begin{itemize}
		\item \textbf{\emph{Descriptive statistics:}} the process of organizing, summarizing, and describing data, often without sampling from a population
		\item[]
		\item Even if we do utilize sampling from a population (inferential statistics), we often want to summarize and describe data graphically before we do other analyses
		\item[]
		\item \textbf{\emph{Exploratory data analysis:}} consists of analyzing data sets to summarize their main characteristics, often with visual methods
	\end{itemize}
\end{frame}

\section{Types of variables}
\begin{frame}{Outline}
\tableofcontents[currentsection,subsectionstyle=show/shaded/hide]
\end{frame}

\begin{frame}{Types of variables}
\begin{itemize}
	\item Categorical variables (qualitative):
	\begin{itemize}
		\item Nominal: There is no natural ordering among the categories
		\item Ordinal: There is a natural ordering among the categories
	\end{itemize}
	\item[]
	\item Numerical variables (quantitative):
	\begin{itemize}
		\item Discrete: There are a countable number of values
		\item Continuous: Any value in a range of values
	\end{itemize}
\end{itemize}
\end{frame}

\section{Descriptive statistics for categorical variables}
\subsection{Frequency Tables}
\begin{frame}{Outline}
\tableofcontents[currentsection,subsectionstyle=show/shaded/hide]
\end{frame}

\begin{frame}{Frequency tables}
	\begin{itemize}
		\item Example: Titanic data set
		\begin{itemize}
			\item There is disagreement over how many people were actually on board, we will use a dataset from the British Board of Trade with 2,201
			\item Variables: Passenger Ticket (1\textsuperscript{st} Class, 2\textsuperscript{nd} Class, 3\textsuperscript{rd} Class, Crew), Sex (Male or Female), Age (Child or Adult), Survived (Yes or No)
		\end{itemize}
	\begin{center}
				\begin{tabular}{|c|c|c|c|}
			\hline 
			\textbf{Ticket} & \textbf{Sex} & \textbf{Age} & \textbf{Survived}  \\ 
			\hline \hline
			1st & Male & Child & Yes \\ 
			\hline 
			3rd & Female & Child & No \\ 
			\hline 
			2nd & Female & Adult & Yes \\ 
			\hline 
			Crew & Male & Adult & Yes \\ 
			\hline
			\vdots & \vdots & \vdots & \vdots \\
			\hline 
		\end{tabular} 
	\end{center}
		\item The first step of \emph{virtually all} descriptive statistics is to make a frequency table
	\end{itemize}
\end{frame}

\begin{frame}{Frequency tables}
	\begin{itemize}
		\item To make a frequency table we add up the number of occurrences for each combination of categorical variables (or combination that we are interested in)
	\end{itemize}
\begin{center}
	\begin{tabular}{|c|c|c|}
		\hline
		\textbf{Ticket} & \textbf{Sex} & \textbf{Frequency} \\
		\hline \hline
		1st & Female & 145\\
		\hline
		1st &  Male & 180\\
		\hline
		2nd & Female & 106\\
		\hline
		2nd  & Male & 179\\
		\hline
		3rd & Female & 196\\
		\hline
		3rd  & Male & 510\\
		\hline
		Crew & Female &  23\\
		\hline
		Crew  & Male & 862\\
		\hline
	\end{tabular}
\end{center}
\end{frame}

\subsection{Barplots}
\begin{frame}{Outline}
\tableofcontents[currentsection,subsectionstyle=show/shaded/hide]
\end{frame}

\begin{frame}{Barplots}
\begin{itemize}
	\item Barplots are a way of visualizing data that has categorical variables
	\item[]
	\item Process for constructing a barplot from a frequency table
		\begin{itemize}
		\item Label frequencies on one axis and categories of the variables on the other
		\item[] 
		\item Construct a rectangle at each category with a height equal to the frequency (number of observations) in the category or other measure
		\item[]
		\item Leave space between each category
	\end{itemize}
\end{itemize}
\end{frame}

\begin{frame}{Barplots}
\begin{columns}
	\begin{column}{0.43 \textwidth}
		\begin{tabular}{|c|c|c|}
			\hline
			\textbf{Ticket} & \textbf{Sex} & \textbf{Frequency} \\
			\hline \hline
			1st & Female & 145\\
			\hline
			1st &  Male & 180\\
			\hline
			2nd & Female & 106\\
			\hline
			2nd  & Male & 179\\
			\hline
			3rd & Female & 196\\
			\hline
			3rd  & Male & 510\\
			\hline
			Crew & Female &  23\\
			\hline
			Crew  & Male & 862\\
			\hline
		\end{tabular}
	\end{column}
	\begin{column}{0.57 \textwidth}
		\includegraphics[scale=.425]{TitanicClassSex}
	\end{column}
\end{columns}
\end{frame}

\begin{frame}{Barplots}
\includegraphics[width=1\linewidth]{TitanicClassSex}
\end{frame}

\begin{frame}{Relative frequency}
	\begin{itemize}
		\item Relative frequency is the frequency of each category divided by the total frequency (for our example 2201)
		\item Relative frequencies should add up to 1
		\item Relative frequencies = Proportions
	\end{itemize}
	\begin{center}
			\begin{tabular}{|c|c|c|c|}
			\hline
			\textbf{Ticket} & \textbf{Sex} & \textbf{Frequency} & \textbf{Relative Frequency}\\
			\hline \hline
			1st & Female & 145 & $145 / 2201 = 0.066$\\
			\hline
			1st &  Male & 180 & $180 / 2201 = 0.082$\\
			\hline
			2nd & Female & 106 & $106 / 2201 = 0.048$\\
			\hline
			2nd  & Male & 179 & $179 / 2201 = 0.081$\\
			\hline
			3rd & Female & 196 & $196 / 2201 = 0.089$\\
			\hline
			3rd  & Male & 510 & $510 / 2201 = 0.232$\\
			\hline
			Crew & Female &  23 & $23 / 2201 = 0.010$\\
			\hline
			Crew  & Male & 862 & $862 / 2201 = 0.392$\\
			\hline
		\end{tabular}
	\end{center}
\end{frame}

\begin{frame}{Relative frequency}
\begin{itemize}
	\item Relative frequency is the basis of pie charts
	\item The book outlines a process for making pie charts, but the best advice that can be given is ``don't make pie charts''
	\item It is hard for us to think of areas in terms of angles and polar coordinates
\end{itemize}
\begin{center}
	\includegraphics[scale=.4]{TitanicPieChart}
\end{center}
\end{frame}

\begin{frame}{Barplots}
\includegraphics[width=1\linewidth]{TitanicClassSex3}
\end{frame}

\begin{frame}{Barplots}
\includegraphics[width=1\linewidth]{TitanicClassSex3b}
\end{frame}

\begin{frame}{Barplots}
\includegraphics[width=1\linewidth]{TitanicClassSex2}
\end{frame}

\section{Descriptive statistics for numeric variables}
\subsection{Histograms}
\begin{frame}{Outline}
\tableofcontents[currentsection,subsectionstyle=show/shaded/hide]
\end{frame}

\begin{frame}{Histograms}{Example dataset}
\begin{itemize}
	\item Histograms look very similar to barplots, but they are for numeric (often continous) data
	\item[]
	\item To illustrate the process of constructing a histogram we will data: O. Nativ, Y., et al. (1988). Prognostic value of flow cytometric nuclear DNA analysis in stage C prostate carcinoma. \emph{Surgical Forum, 39}.
	\item[]
	\item Let's make a histogram for one of the variables for the 139 patients in this dataset. The variable is the percentage of cells in each sample that are in G2 phase
	\item[]
	\item Here are some of the values: 10.26, 9.99, 3.57, 22.56, 6.14, 13.69, 11.77, 27.27, 19.34, 14.82 
\end{itemize}
\end{frame}

\begin{frame}{Histograms}{Process}
	\begin{itemize}
		\item The first step in constructing a histogram is to divide the range of numbers (the difference between the smallest and largest measurements) into \textbf{\emph{class intervals}}
		\item[]
		\item From our example data:
		\begin{itemize}
			\item Smallest number: 2.4
			\item Largest number: 54.93
			\item Range: $54.93 - 2.4 = 52.53$
		\end{itemize}	
		\item[]
		\item We typically want between 5 and 20 class intervals. Let's choose 10. Then to find the width of each class interval:
		\begin{itemize}
			\item $52.53 / 10 = 5.253$
			\item Rounded up (which may give us 9 intervals), the width of each class interval is then 6
		\end{itemize}
	\end{itemize}
\end{frame}

\begin{frame}{Histograms}{Process}
	\begin{itemize}
		\item We now need to set the first class interval so that it contains the smallest number and has a width of 6: 2.001 -- 8.000
		\item[]
		\item So now the class intervals are: 2.001 -- 8.000, 8.001 -- 14.000, 14.001 -- 20.000, 20.001 -- 26.000, 26.001 -- 32.000, ... , 50.001 -- 56.000
		\item[]
		\item Now we will make a frequency table just as before 
	\end{itemize}
\end{frame}

\begin{frame}{Histograms}{Process: Frequency table}
Let's assign the first 14 numbers into which class interval they belong:
10.26,  9.99,  3.57, 22.56,  6.14, 13.69, 11.77, 27.27, 19.34, 14.82, 10.22, 15.66, 17.79, 11.11
\begin{center}
	\begin{tabular}{|c|c|c|}
		\hline
		\textbf{Class interval} & \textbf{Numbers} & \textbf{Frequency} \\
		\hline \hline
		2.001 -- 8.000 & & \\  \hline 
		8.001 -- 14.000 & & \\  \hline 
		14.001 -- 20.000 & & \\  \hline 
		20.001 -- 26.000 & & \\  \hline 
		26.001 -- 32.000 & & \\  \hline 
		32.001 -- 38.000 & & \\  \hline 
		38.001 -- 44.000 & & \\  \hline 
		44.001 -- 50.000 & & \\  \hline 
		50.001 -- 56.000 & & \\ \hline 
	\end{tabular}
\end{center}
\end{frame}

\begin{frame}{Histograms}{Process: Frequency table}
	Let's assign the first 14 numbers into which class interval they belong:
	10.26,  9.99,  3.57, 22.56,  6.14, 13.69, 11.77, 27.27, 19.34, 14.82, 10.22, 15.66, 17.79, 11.11
	\begin{center}
		\begin{tabular}{|c|c|c|}
			\hline
			\textbf{Class interval} & \textbf{Numbers} & \textbf{Frequency} \\
			\hline \hline
			2.001 -- 8.000 & 3.57, 6.14 & 2\\ \hline 
			8.001 -- 14.000 & 9.99, 10.22, 10.26, 11.11, 11.77, 13.69 & 6 \\ \hline 
			14.001 -- 20.000 & 14.82, 15.66, 17.79, 19.34 & 4\\ \hline 
			20.001 -- 26.000 & 22.56 & 1 \\ \hline 
			26.001 -- 32.000 & 27.27 & 1 \\ \hline 
			32.001 -- 38.000 &  & 0 \\ \hline 
			38.001 -- 44.000 & & 0 \\ \hline 
			44.001 -- 50.000 & & 0 \\ \hline 
			50.001 -- 56.000 & & 0\\ \hline 
		\end{tabular}
	\end{center}
\end{frame}

\begin{frame}{Histograms}{Process: Frequency table}
If we did this with all of the data we would have:
	\begin{center}
		\begin{tabular}{|c|c|}
			\hline
			\textbf{Class interval} & \textbf{Frequency } \\
			\hline \hline
			2.001 -- 8.000 & 27 \\ \hline 
			8.001 -- 14.000 & 57 \\ \hline 
			14.001 -- 20.000 & 32 \\ \hline 
			20.001 -- 26.000 & 12 \\ \hline 
			26.001 -- 32.000 & 5 \\ \hline 
			32.001 -- 38.000 & 2 \\ \hline 
			38.001 -- 44.000 & 2 \\ \hline 
			44.001 -- 50.000 & 1 \\ \hline 
			50.001 -- 56.000 & 1\\ \hline 
		\end{tabular}
	\end{center}
\end{frame}

\begin{frame}{Histograms}{Process: Frequency table}
	Now we compute the relative frequency:
	\begin{center}
		\begin{tabular}{|c|c|c|}
			\hline
			\textbf{Class interval} & \textbf{Frequency} & \textbf{Relative Frequency} \\
			\hline \hline
			2.001 -- 8.000 & 27 & $27 / 139 = 0.194$ \\ \hline 
			8.001 -- 14.000 & 57 & \\ \hline 
			14.001 -- 20.000 & 32 &\\ \hline 
			20.001 -- 26.000 & 12 &\\ \hline 
			26.001 -- 32.000 & 5 &\\ \hline 
			32.001 -- 38.000 & 2& \\ \hline 
			38.001 -- 44.000 & 2 &\\ \hline 
			44.001 -- 50.000 & 1& \\ \hline 
			50.001 -- 56.000 & 1&\\ \hline 
		\end{tabular}
	\end{center}
\end{frame}

\begin{frame}{Histograms}{Process: Frequency table}
	Now we compute the relative frequency:
	\begin{center}
		\begin{tabular}{|c|c|c|}
			\hline
			\textbf{Class interval} & \textbf{Frequency} & \textbf{Relative Frequency} \\
			\hline \hline
			2.001 -- 8.000 & 27 & 0.194 \\ \hline 
			8.001 -- 14.000 & 57 & 0.410\\ \hline 
			14.001 -- 20.000 & 32 & 0.230\\ \hline 
			20.001 -- 26.000 & 12 & 0.086\\ \hline 
			26.001 -- 32.000 & 5 & 0.036\\ \hline 
			32.001 -- 38.000 & 2& 0.014\\ \hline 
			38.001 -- 44.000 & 2 & 0.014\\ \hline 
			44.001 -- 50.000 & 1& 0.007\\ \hline 
			50.001 -- 56.000 & 1& 0.007\\ \hline 
		\end{tabular}
	\end{center}
\end{frame}

\begin{frame}{Histograms}{Frequency Histogram}
\begin{center}
	\includegraphics[width = .9\linewidth]{g2Hist}
\end{center}
\end{frame}

\begin{frame}{Histograms}{Relative Frequency Histogram}
	\begin{center}
		\includegraphics[width = .9\linewidth]{g2Hist2}
	\end{center}
\end{frame}

\begin{frame}{Histograms}
Some fun facts about histograms:
	\begin{itemize}
		\item If we select a measurement at random from the set of sample measurements, the \textbf{\emph{probability}} (or chance), that that measurement is in a specific interval is equal to the relative frequency for that interval
		\begin{itemize}
			\item For example, if we select a measurement at random the probability that it between 8.001 -- 14.000 is 0.41 or 41\%
		\end{itemize}
		\item[]
		\item Relative frequency histograms are useful as they can be compared between samples with different sample sizes
	\end{itemize}
\end{frame}

\subsection{The distribution of data}
\begin{frame}{Outline}
	\tableofcontents[currentsection,subsectionstyle=show/shaded/hide]
\end{frame}

\begin{frame}{Distribution of data}
	\begin{itemize}
		\item The \textbf{\emph{distribution of data}} from a population or sample is a listing, mathematical function, or plot showing all the possible values (or intervals) of the data and how frequently they occur
		\item For example we would say the distribution of our variable of interest (\% of cells in G2 phase) is as follows:
	\end{itemize}
\begin{center}
	\includegraphics[width = .4\linewidth]{g2Hist}
\end{center}
\begin{itemize}
	\item Later we will learn about distributions that are generated by natural, biological, and physical processes that are commonly seen in the sciences
\end{itemize}
\end{frame}

\begin{frame}{Distribution of data}
	\begin{itemize}
		\item There are some attributes of data distributions that we often wish to describe
		\item[]
		\item \textbf{\emph{Unimodal}}: If the data distribution appears to have only one peak we call it unimodal
	\end{itemize}
\end{frame}

\begin{frame}{Distribution of data}{Unimodal}
	\begin{center}
		\includegraphics[width = .9\linewidth]{histNormal}
	\end{center}
\end{frame}

\begin{frame}{Distribution of data}
	\begin{itemize}
		\item There are some attributes of data distributions that we often wish to describe
		\item[]
		\item \textbf{\emph{Bimodal}}: If the data distribution appears to two peaks we call it bimodal
	\end{itemize}
\end{frame}

\begin{frame}{Distribution of data}{Bimodal}
	\begin{center}
		\includegraphics[width = .9\linewidth]{histNormal2}
	\end{center}
\end{frame}

\begin{frame}{Distribution of data}{Bimodal}
	\begin{center}
			\includegraphics[width = .95\linewidth]{histNormal2b}
	\end{center}
\end{frame}

\begin{frame}{Distribution of data}
	\begin{itemize}
		\item There are some attributes of data distributions that we often wish to describe
		\item[]
		\item \textbf{\emph{Symmetric}}: The data distribution has equal mass to the left and right side of the mode. The right and left tails have similar (equal) length
	\end{itemize}
\end{frame}

\begin{frame}{Distribution of data}{Symmetric}
	\begin{center}
		\includegraphics[width = .9\linewidth]{histNormalSym}
	\end{center}
\end{frame}

\begin{frame}{Distribution of data}
	\begin{itemize}
		\item There are some attributes of data distributions that we often wish to describe
		\item[]
		\item \textbf{\emph{Right skew}}: If the data distribution has a long right tail we say it is skewed to the right
	\end{itemize}
\end{frame}

\begin{frame}{Distribution of data}{Right skew}
	\begin{center}
		\includegraphics[width = .9\linewidth]{histRightSkew}
	\end{center}
\end{frame}

\begin{frame}{Distribution of data}
	\begin{itemize}
		\item There are some attributes of data distributions that we often wish to describe
		\item[]
		\item \textbf{\emph{Left skew}}: If the data distribution has a long left tail we say it is skewed to the left
	\end{itemize}
\end{frame}

\begin{frame}{Distribution of data}{Left skew}
	\begin{center}
		\includegraphics[width = .9\linewidth]{histLeftSkew}
	\end{center}
\end{frame}

\end{document}
