\documentclass[xcolor=dvipsnames]{beamer} 
\usetheme{AnnArbor}
\usecolortheme{beaver}

\usepackage{amsmath,graphicx,booktabs,tikz,subfig,color,lmodern}
\definecolor{mycol}{rgb}{.4,.85,1}
\setbeamercolor{title}{bg=mycol,fg=black} 
\setbeamercolor{palette primary}{use=structure,fg=white,bg=red}
\setbeamercolor{block title}{fg=white,bg=red!50!black}
% \setbeamercolor{block title}{fg=white,bg=blue!75!black}

\title[Lecture 1]{Lecture 3: Descriptive Statistics I}
\author[Patrick Trainor]{Patrick Trainor, PhD, MS, MA}
\institute[NMSU]{New Mexico State University}
\date{August 26, 2019}

\begin{document}
	
\begin{frame}
	\maketitle
\end{frame}

\begin{frame}{Descriptive vs. Inferential Statistics}
\begin{columns}
	\begin{column}{0.5 \textwidth}
		\includegraphics[scale = .3]{../lecture1_Introduction/Sampling}
	\end{column}
	\begin{column}{0.5 \textwidth}
		\begin{itemize}
		\item In lecture \#1 we said that:
		\begin{itemize}
			\item \textbf{\emph{Statistical inference:}} the theory, methods, and practice of forming judgments about the [characteristics] of a population ... typically on the basis of random sampling
		\end{itemize}
		\item Statistical inference $\approx$ inferential statistics
		\item[]
		\item In chapter 3 of the text, the authors claim there are two kinds of statistical methods: inferential statistics \& descriptive statistics 
		\end{itemize}
	\end{column}
\end{columns}
\end{frame}


\begin{frame}{Descriptive vs. Inferential Statistics}
	\begin{itemize}
		\item In chapter 3 of the text, the authors claim there are two kinds of statistical methods: inferential statistics \& descriptive statistics
		\item[]
		\item Book example of descriptive statistics: 
		\begin{itemize}
			\item An HMO (a type of health insurance product) has complete data on all of the medical claims it has paid on behalf of its members / beneficiaries
			\item[]
			\item This is a rare case in which to make an conclusion about a population (the members / beneficiaries of the HMO) sampling is not needed
			\item[] 
			\item If you have a research question (Do 70-75 year olds have more medical claims than 20-25 year olds?), you can answer it without sampling
		\end{itemize}
	\end{itemize}
\end{frame}

\begin{frame}{Descriptive vs. Inferential Statistics}
\begin{itemize}
	\item In chapter 3 of the text, the authors claim there are two kinds of statistical methods: inferential statistics \& descriptive statistics
	\item[]
	\item Book example of descriptive statistics: 
	\begin{itemize}
		\item An HMO (a type of health insurance product) has complete data on all of the medical claims it has paid on behalf of its members / beneficiaries
		\item[]
		\item This is a rare case in which to make an conclusion about a population (the members / beneficiaries of the HMO) sampling is not needed
		\item[] 
		\item If you have a research question (Do 70-75 year olds have more medical claims than 20-25 year olds?), you can answer it without sampling
	\end{itemize}
\end{itemize}
\end{frame}

\begin{frame}{Descriptive statistics}{Exploratory data analysis}
	\begin{itemize}
		\item \textbf{\emph{Descriptive statistics:}} the process of organizing, summarizing, and describing data, often without sampling from a population
		\item[]
		\item Even if we do utilize sampling from a population (inferential statistics), we often want to summarize and describe data graphically before we do other analyses
		\item[]
		\item \textbf{\emph{Exploratory data analysis:}} consists of analyzing data sets to summarize their main characteristics, often with visual methods
	\end{itemize}
\end{frame}

\begin{frame}{Descriptive statistics}{Frequency tables}
	\begin{itemize}
		\item Example: Titanic data set
		\begin{itemize}
			\item There is disagreement over how many people were actually on board, we will use a dataset from the British Board of Trade with 2,201
			\item Variables: Passenger Ticket (1\textsuperscript{st} Class, 2\textsuperscript{nd} Class, 3\textsuperscript{rd} Class, Crew), Sex (Male or Female), Age (Child or Adult), Survived (Yes or No)
		\end{itemize}
	\begin{center}
				\begin{tabular}{|c|c|c|c|}
			\hline 
			\textbf{Ticket} & \textbf{Sex} & \textbf{Age} & \textbf{Survived}  \\ 
			\hline \hline
			1st & Male & Child & Yes \\ 
			\hline 
			3rd & Female & Child & No \\ 
			\hline 
			2nd & Female & Adult & Yes \\ 
			\hline 
			Crew & Male & Adult & Yes \\ 
			\hline
			\vdots & \vdots & \vdots & \vdots \\
			\hline 
		\end{tabular} 
	\end{center}
		\item The first step of \emph{virtually all} descriptive statistics is to make a frequency table
	\end{itemize}
\end{frame}

\begin{frame}{Descriptive statistics}{Frequency tables}
	\begin{itemize}
		\item \textbf{\emph{Factors:}} variables that are not numbers or numeric
		\item[]
		\item To make a frequency table we add up the number of occurrences for each combination of factors (or combination that we are interested in)
	\end{itemize}
\begin{center}
	\begin{tabular}{|c|c|c|}
		\hline
		\textbf{Ticket} & \textbf{Sex} & \textbf{Frequency} \\
		\hline \hline
		1st & Female & 145\\
		\hline
		1st &  Male & 180\\
		\hline
		2nd & Female & 106\\
		\hline
		2nd  & Male & 179\\
		\hline
		3rd & Female & 196\\
		\hline
		3rd  & Male & 510\\
		\hline
		Crew & Female &  23\\
		\hline
		Crew  & Male & 862\\
		\hline
	\end{tabular}
\end{center}
\end{frame}

\begin{frame}{Descriptive statistics}{Barplots}

\begin{itemize}
	\item Barplots are a way of visualizing data that has factor variables
	\item[]
	\item Process for constructing a barplot from a frequency table
		\begin{itemize}
		\item Label frequencies on one axis and categories of the variables on the other
		\item[] 
		\item Construct a rectangle at each category with a height equal to the frequency (number of observations) in the category or other measure
		\item[]
		\item Leave space between each category
	\end{itemize}
\end{itemize}
\end{frame}

\begin{frame}{Descriptive statistics}{Barplots}
\begin{columns}
	\begin{column}{0.43 \textwidth}
		\begin{tabular}{|c|c|c|}
			\hline
			\textbf{Ticket} & \textbf{Sex} & \textbf{Frequency} \\
			\hline \hline
			1st & Female & 145\\
			\hline
			1st &  Male & 180\\
			\hline
			2nd & Female & 106\\
			\hline
			2nd  & Male & 179\\
			\hline
			3rd & Female & 196\\
			\hline
			3rd  & Male & 510\\
			\hline
			Crew & Female &  23\\
			\hline
			Crew  & Male & 862\\
			\hline
		\end{tabular}
	\end{column}
	\begin{column}{0.57 \textwidth}
		\includegraphics[scale=.425]{TitanicClassSex}
	\end{column}
\end{columns}
\end{frame}

\begin{frame}{Descriptive statistics}{Barplots}
\includegraphics[width=1\linewidth]{TitanicClassSex}
\end{frame}

\begin{frame}{Descriptive statistics}{Relative frequency}
	\begin{itemize}
		\item Relative frequency is the frequency of each category divided by the total frequency (for our example 2201)
		\item Relative frequencies should add up to 1
		\item Relative frequencies = Proportions
	\end{itemize}
	\begin{center}
			\begin{tabular}{|c|c|c|c|}
			\hline
			\textbf{Ticket} & \textbf{Sex} & \textbf{Frequency} & \textbf{Relative Frequency}\\
			\hline \hline
			1st & Female & 145 & $145 / 2201 = 0.066$\\
			\hline
			1st &  Male & 180 & $180 / 2201 = 0.082$\\
			\hline
			2nd & Female & 106 & $106 / 2201 = 0.048$\\
			\hline
			2nd  & Male & 179 & $179 / 2201 = 0.081$\\
			\hline
			3rd & Female & 196 & $196 / 2201 = 0.089$\\
			\hline
			3rd  & Male & 510 & $510 / 2201 = 0.232$\\
			\hline
			Crew & Female &  23 & $23 / 2201 = 0.010$\\
			\hline
			Crew  & Male & 862 & $862 / 2201 = 0.392$\\
			\hline
		\end{tabular}
	\end{center}
\end{frame}

\begin{frame}{Descriptive statistics}{Relative frequency}
\begin{itemize}
	\item Relative frequency is the basis of pie charts
	\item The book outlines a process for making pie charts, but the best advice that can be given is ``don't make pie charts''
	\item It is hard for us to think of areas in terms of angles and polar coordinates
\end{itemize}
\begin{center}
	\includegraphics[scale=.4]{TitanicPieChart}
\end{center}
\end{frame}

\begin{frame}{Descriptive statistics}{Barplots}
\includegraphics[width=1\linewidth]{TitanicClassSex3}
\end{frame}

\begin{frame}{Descriptive statistics}{Barplots}
\includegraphics[width=1\linewidth]{TitanicClassSex3b}
\end{frame}

\begin{frame}{Descriptive statistics}{Barplots}
\includegraphics[width=1\linewidth]{TitanicClassSex2}
\end{frame}

\end{document}
