\documentclass[xcolor=dvipsnames]{beamer} 
\usetheme{AnnArbor}
\usecolortheme{beaver}

\usepackage{amsmath,graphicx,booktabs,tikz,subfig,color,lmodern}
\definecolor{mycol}{rgb}{.4,.85,1}
\setbeamercolor{title}{bg=mycol,fg=black} 
\setbeamercolor{palette primary}{use=structure,fg=white,bg=red}
\setbeamercolor{block title}{fg=white,bg=red!50!black}
% \setbeamercolor{block title}{fg=white,bg=blue!75!black}

\title[Lecture 4]{Lecture 4: Descriptive Statistics II}
\author[Patrick Trainor]{Patrick Trainor, PhD, MS, MA}
\institute[NMSU]{New Mexico State University}
\date{August 28, 2019}

\begin{document}
	
\begin{frame}
	\maketitle
\end{frame}

\begin{frame}{Outline}
	\tableofcontents[hideallsubsections]
\end{frame}

\section{Numerical descriptive measures}
\begin{frame}{Outline}
	\tableofcontents[currentsection,subsectionstyle=show/shaded/hide]
\end{frame}

\begin{frame}{Numerical descriptive measures}
	\begin{itemize}
		\item We often want numerical descriptive measures (or summaries) that tell us something about the distribution of data
		\item[]
		\item The most common reason that we want numerical descriptive measures is to convey information quickly and easily
		\item[]
		\item Example: The average height of women in the US is 63.7 inches; the average height of men is 69.7 inches
		\item[]
		\item These two numbers allow us to make a ``quick and easy'' judgment that ``on average'' men are taller than women without comparing histograms showing the data distribution for each sex*
	\end{itemize}
{\tiny *There are often non-trivial ethical issues with using averages to describe any group or population}
\end{frame}

\begin{frame}{Numerical descriptive measures}
	\begin{itemize}
		\item We often want numerical descriptive measures (or summaries) that tell us something about the distribution of data
		\item[]
		\item When these measures describe a population, they are called \textbf{\emph{parameters}}
		\item[]
		\item When these measures describe a sample from a population, they are called \textbf{\emph{statistics}}
	\end{itemize}
\end{frame}

\section{Measures of central tendency}
\subsection{The mode}

\begin{frame}{Outline}
	\tableofcontents[currentsection,subsectionstyle=show/shaded/hide]
\end{frame}

\begin{frame}{The mode}
	\begin{itemize}
		\item The \emph{\textbf{mode}} of a set of measurements is the measurement that occurs most often (with highest frequency)
		\item[]
		\item Occasionally there is more than one mode 
		\item[]
		\item Example set of measurements: $\{1, 2, 3, 3, 4, 5, 5, 5, 5, 6, 6, 7, 7, 8\}$\\
		The mode is 5
	\end{itemize}
\end{frame}


\begin{frame}{The mode}
	\begin{itemize}
		\item The \emph{\textbf{mode}} of a set of measurements is the measurement that occurs most often (with highest frequency)
	\end{itemize}
	\begin{center}
			\includegraphics[width=0.7\linewidth]{../lecture3_DescriptiveStatistics/histRightSkew}
	\end{center}
\end{frame}

\begin{frame}{Pause for data}
dataset from the 1992 \emph{Statistical Abstract of the United States}. Selected Mountain West states only.
	\begin{center}
			\begin{tabular}{|c|c|c|c|}
			\hline 
			\textbf{State} & \textbf{SAT Math} & \textbf{\$ per pupil (1,000's)} & \textbf{Teacher pay (1,000's)} \\ 
			\hline \hline
			AZ & 497  & 4.231 & 30 \\ \hline 
			CO & 513  & 4.809 & 31 \\ \hline 
			ID & 502  & 3.200 & 25  \\  \hline 
			MT & 523  & 5.184 & 26 \\ \hline 
			NV & 487  & 4.564 & 32 \\ \hline 
			NM & 527  & 4.446 & 26 \\ \hline 
			UT & 539  & 2.993 & 25 \\ \hline 
			WY & 519  & 5.255 & 29 \\ \hline 
		\end{tabular} 
	\end{center}
\end{frame}

\begin{frame}{The mode}
What is the mode for teacher pay?
	\begin{center}
	\begin{tabular}{|c|c|}
		\hline 
		\textbf{State} & \textbf{Teacher pay (1,000's)} \\ 
		\hline \hline
		AZ & 30 \\ \hline 
		CO &  31 \\ \hline 
		ID & 25  \\  \hline 
		MT &  26 \\ \hline 
		NV & 32 \\ \hline 
		NM &  26 \\ \hline 
		UT &  25 \\ \hline 
		WY &  29 \\ \hline 
	\end{tabular} 
\end{center}
\end{frame}

\begin{frame}{The mode}
	The mode values are 25 and 26.
	\begin{center}
		\begin{tabular}{|c|c|}
			\hline 
			\textbf{State} & \textbf{Teacher pay (1,000's)} \\ 
			\hline \hline
			AZ & 30 \\ \hline 
			CO &  31 \\ \hline 
			ID & 25  \\  \hline 
			MT &  26 \\ \hline 
			NV & 32 \\ \hline 
			NM &  26 \\ \hline 
			UT &  25 \\ \hline 
			WY &  29 \\ \hline 
		\end{tabular} 
	\end{center}
\end{frame}

\subsection{The median}
\begin{frame}{Outline}
	\tableofcontents[currentsection,subsectionstyle=show/shaded/hide]
\end{frame}

\begin{frame}{The median}
	\begin{itemize}
		\item The \emph{\textbf{median}} of a set of measurements is the measurement that is the middle value when the measurements are arranged from least to greatest
		\item[]
		\item If there are an even number we take the average of two numbers
	\end{itemize}
\end{frame}

\begin{frame}{The median}
	What is the median of teacher pay? First, please order the numbers from least to greatest, then select the middle number (or average of middle two)
	\begin{center}
		\begin{tabular}{|c|c|}
			\hline 
			\textbf{State} & \textbf{Teacher pay (1,000's)} \\ 
			\hline \hline
			AZ & 30 \\ \hline 
			CO &  31 \\ \hline 
			ID & 25  \\  \hline 
			MT &  26 \\ \hline 
			NV & 32 \\ \hline 
			NM &  26 \\ \hline 
			UT &  25 \\ \hline 
			WY &  29 \\ \hline 
		\end{tabular} 
	\end{center}
\end{frame}

\begin{frame}{The median}
		\begin{center}
		\begin{tabular}{|c|c|}
			\hline 
			\textbf{State} & \textbf{Teacher pay (1,000's)} \\ 
			\hline \hline
			AZ & 30 \\ \hline 
			CO &  31 \\ \hline 
			ID & 25  \\  \hline 
			MT &  26 \\ \hline 
			NV & 32 \\ \hline 
			NM &  26 \\ \hline 
			UT &  25 \\ \hline 
			WY &  29 \\ \hline 
		\end{tabular} 
	\end{center}
	\begin{itemize}
		\item Ordered: 	$\{25, 25, 26, \textbf{26}, \textbf{29}, 30, 31, 32\} $
		\item[]
		\item $(26+29)/2 = 27.5$
	\end{itemize}
\end{frame}

\subsection{The mean}
\begin{frame}{Outline}
	\tableofcontents[currentsection,subsectionstyle=show/shaded/hide]
\end{frame}

\begin{frame}{The mean}
	\begin{itemize}
		\item The \emph{\textbf{mean}} (or arithmetic mean) of a set of measurements is the sum of all measurements divided by the total number of measurements 
		\item Let's calculate the mean of teacher pay now!
	\end{itemize}
		\begin{center}
			\begin{tabular}{|c|c|}
				\hline 
				\textbf{State} & \textbf{Teacher pay (1,000's)} \\ 
				\hline \hline
				AZ & 30 \\ \hline 
				CO &  31 \\ \hline 
				ID & 25  \\  \hline 
				MT &  26 \\ \hline 
				NV & 32 \\ \hline 
				NM &  26 \\ \hline 
				UT &  25 \\ \hline 
				WY &  29 \\ \hline 
			\end{tabular} 
		\end{center}
\end{frame}

\begin{frame}{The mean}
	\begin{itemize}
		\item Let's calculate the mean of teacher pay now!
	\end{itemize}
	\begin{center}
		\begin{tabular}{|c|c|}
			\hline 
			\textbf{State} & \textbf{Teacher pay (1,000's)} \\ 
			\hline \hline
			AZ & 30 \\ \hline 
			CO &  31 \\ \hline 
			ID & 25  \\  \hline 
			MT &  26 \\ \hline 
			NV & 32 \\ \hline 
			NM &  26 \\ \hline 
			UT &  25 \\ \hline 
			WY &  29 \\ \hline 
		\end{tabular} 
	\end{center}
\begin{gather*}
	\frac{30+31+25+26+32+26+25+29}{8} = \frac{224}{8} = 28.0
\end{gather*}
\end{frame}

\begin{frame}{Notation}
	\begin{itemize}
		\item We use some notation to simplify our calculations
		\item[]
		\item You do need to know this notation...
		\item[]
		\item \textbf{\emph{Summation:}} If we have $n$ measurements which we label $\{i = 1, 2, \hdots, n\}$, and the measurements are denoted by $y_i$, then:
		\begin{gather*}
		\sum_{i=1}^{n} y_i = y_1 + y_2 + y_3 + \hdots + y_n
		\end{gather*}
	\end{itemize}
\end{frame}

\begin{frame}{Notation example}
	Note with this example that $n = 8$.
		\begin{center}
		\begin{tabular}{|c|c|c|c|}
			\hline 
			\textbf{State} & \textbf{Teacher pay (1,000's)} & $y_i$ \\ 
			\hline \hline
			AZ & 30 & $y_1$\\ \hline 
			CO &  31 & $y_2$ \\ \hline 
			ID & 25 & $y_3$ \\  \hline 
			MT &  26 & $y_4$ \\ \hline 
			NV & 32 & $y_5$ \\ \hline 
			NM &  26 & $y_6$\\ \hline 
			UT &  25 & $y_7$\\ \hline 
			WY &  29 & $y_8$\\ \hline 
		\end{tabular} 
	\end{center}
\vspace{-7 pt}
\begin{gather*}
	\sum_{i=1}^{n} y_i = y_1 + y_2 + \hdots + y_8 \\
		\sum_{i=1}^{n} y_i = 30+31+25+26+32+26+25+29 = 224
\end{gather*}
\end{frame}

\begin{frame}{Notation example}
	Note with this example that $n = 8$.
	\begin{center}
		\begin{tabular}{|c|c|c|c|}
			\hline 
			\textbf{State} & \textbf{Teacher pay (1,000's)} & $y_i$ \\ 
			\hline \hline
			AZ & 30 & $y_1$\\ \hline 
			CO &  31 & $y_2$ \\ \hline 
			ID & 25 & $y_3$ \\  \hline 
			MT &  26 & $y_4$ \\ \hline 
			NV & 32 & $y_5$ \\ \hline 
			NM &  26 & $y_6$\\ \hline 
			UT &  25 & $y_7$\\ \hline 
			WY &  29 & $y_8$\\ \hline 
		\end{tabular} 
	\end{center}
	\begin{gather*}
	\sum_{i=1}^{n} y_i = 224
	\end{gather*}
\end{frame}

\begin{frame}{Notation}
	\begin{itemize}
		\item Occasionally the $n$ on top is omitted:
		\begin{gather*}
		\sum_{i=1}^{n} y_i = \sum_{i=1} y_i 
		\end{gather*}
		\item[]
		\item A population mean is denoted $\mu$
		\item[]
		\item A sample mean is often denoted $\bar{y}$, so:
		\begin{gather*}
		\bar{y} = \frac{\sum_{i=1}^{n} y_i}{n}
		\end{gather*}
	\end{itemize}
\end{frame}

\begin{frame}{The mean}{Foreshadowing}
	\begin{itemize}
		\item Suppose we knew the weight of every fish in the pond by the horseshoe. This group of fish would constitute a \textbf{population}
		\item The weights of the fish are: \\
 498.64 499.59 510.11 498.42 478.43 504.99 492.45 507.79 507.55 489.00 501.67 499.71 518.76 502.45 507.02 499.85
498.57 503.21 501.22 494.05 495.58 502.91 507.24 504.60 501.85 502.34 505.93 520.01 481.63 491.38 515.83 501.55
 497.25 507.88 497.77 513.92 495.11 501.37 500.04 492.73 492.79 498.09 513.35 503.56 508.43 507.75 500.80 493.27
518.36 497.93 494.60 484.99 502.68 500.34 490.05 492.30 494.22 490.67 478.23 494.86 500.89 502.94 506.92 483.88
501.88 500.77 498.26 486.01 496.31 504.51 516.26 519.23 498.37 495.61 511.91 516.70 511.33 484.83 527.31 504.99
 485.08 509.22 494.83 521.05 491.13 498.32 509.09 486.05 508.14 490.43 496.30 514.79 493.96 497.09 482.16 522.58
 493.33 504.78 488.39 496.20
	\end{itemize}
\end{frame}

\begin{frame}{The mean}{Foreshadowing}
	\begin{itemize}
		\item Suppose we knew the weight of every fish in the pond by the horseshoe. This group of fish would constitute a \textbf{population}
		\item[]
		\item The population mean weight is, $\mu = 500.595$
		\item[]
		\item If I take a random sample of ten of these: \\
		502.94 490.43 497.77 498.64 503.56 516.70 497.93 481.63 499.59 501.37
		\item[]
		\item The sample mean would be $\bar{y} = 499.056$
		\item[]
		\item Wow this is close!
	\end{itemize}
\end{frame}

\begin{frame}{The mean}{Outliers}
\begin{itemize}
	\item As a measure of the center of a distribution, the mean is susceptible to extreme influence from extreme values (called \textbf{\emph{outliers}})
\end{itemize}
\begin{center}
	\includegraphics[width=0.7\linewidth]{fishweight}
\end{center}
\end{frame}

\begin{frame}{The mean}{Outliers}
\begin{itemize}
	\item As a measure of the center of a distribution, the mean is susceptible to extreme influence from extreme values (called \textbf{\emph{outliers}})
\end{itemize}
\begin{center}
	\includegraphics[width=0.7\linewidth]{fishweightMeans}
\end{center}
\end{frame}

\begin{frame}{The mean}{Outliers \& Trimmed means}
\begin{itemize}
	\item As a measure of the center of a distribution, the mean is susceptible to extreme influence from extreme values (called \textbf{\emph{outliers}})
	\item[]
	\item \textbf{\emph{Trimmed mean:}} A trimmed mean is computed by first removing the least and greatest 5\% (or 10\%) of the values, and then computing the mean
	\item[]
	\item Removing data from any analysis always requires justification--I don't recommend using trimmed means
\end{itemize}
\end{frame}

\begin{frame}{The median handling outliers}
\begin{itemize}
	\item The median is less susceptible to extreme influence from extreme values (called \textbf{\emph{outliers}})
\end{itemize}
\begin{center}
	\includegraphics[width=0.7\linewidth]{fishweightMedians}
\end{center}
\end{frame}

\subsection{Skewness and measures of central tendency}
\begin{frame}{Outline}
\tableofcontents[currentsection,subsectionstyle=show/shaded/hide]
\end{frame}

\begin{frame}{Skewness and measures of central tendency}
	\begin{center}
		\includegraphics[width=0.9\linewidth]{allSkew}
	\end{center}
\end{frame}

\begin{frame}{Summary facts about measures of central tendency}
	\begin{itemize}
		\item The mode:
		\begin{itemize}
			\item It is the most frequent / probable measurement in the dataset
			\item There can be more than one for a dataset
			\item It is not influenced by outliers
			\item It is applicable for both qualitative and quantitative data
		\end{itemize}
		\item The median:
		\begin{itemize}
			\item It is the middle value; 50\% of the measurements lie above it and 50\% below it
			\item There is only one for a dataset
			\item It is not influenced by outliers
			\item It is applicable to quantitative data only 
		\end{itemize}
		\item The mean:
		\begin{itemize}
			\item It is the arithmetic average of the measurements in the dataset
			\item There is only one for a dataset
			\item The value is influenced by outliers
			\item It is applicable to quantitative data only 
		\end{itemize}			
	\end{itemize}
\end{frame}

\section{Measures of variability}
\begin{frame}{Outline}
	\tableofcontents[currentsection,subsectionstyle=show/shaded/hide]
\end{frame}

\end{document}