\documentclass[xcolor=dvipsnames]{beamer} 
\usetheme{AnnArbor}
\usecolortheme{beaver}

\usepackage{amsmath,graphicx,booktabs,tikz,subfig,color,lmodern}
\definecolor{mycol}{rgb}{.4,.85,1}
\setbeamercolor{title}{bg=mycol,fg=black} 
\setbeamercolor{palette primary}{use=structure,fg=white,bg=red}
\setbeamercolor{block title}{fg=white,bg=red!50!black}
% \setbeamercolor{block title}{fg=white,bg=blue!75!black}

\title[Lecture 4]{Lecture 4: Descriptive Statistics II}
\author[Patrick Trainor]{Patrick Trainor, PhD, MS, MA}
\institute[NMSU]{New Mexico State University}
\date{August 28, 2019}

\begin{document}
	
\begin{frame}
	\maketitle
\end{frame}

\begin{frame}{Outline}
	\tableofcontents[hideallsubsections]
\end{frame}

\section{Numerical descriptive measures}
\begin{frame}{Outline}
	\tableofcontents[currentsection,subsectionstyle=show/shaded/hide]
\end{frame}

\begin{frame}{Numerical descriptive measures}
	\begin{itemize}
		\item We often want numerical descriptive measures (or summaries) that tell us something about the distribution of data
		\item[]
		\item The most common reason that we want numerical descriptive measures is to convey information quickly and easily
		\item[]
		\item Example: The average height of women in the US is 63.7 inches; the average height of men is 69.7 inches
		\item[]
		\item These two numbers allow us to make a ``quick and easy'' judgment that ``on average'' men are taller than women without comparing histograms showing the data distribution for each sex*
	\end{itemize}
{\tiny *There are often non-trivial ethical issues with using averages to describe any group or population}
\end{frame}

\begin{frame}{Numerical descriptive measures}
	\begin{itemize}
		\item We often want numerical descriptive measures (or summaries) that tell us something about the distribution of data
		\item[]
		\item When these measures describe a population, they are called \textbf{\emph{parameters}}
		\item[]
		\item When these measures describe a sample from a population, they are called \textbf{\emph{statistics}}
	\end{itemize}
\end{frame}

\section{Measures of central tendency}
\subsection{The mode}

\begin{frame}{Outline}
	\tableofcontents[currentsection,subsectionstyle=show/shaded/hide]
\end{frame}

\begin{frame}{The mode}
	\begin{itemize}
		\item The \emph{\textbf{mode}} of a set of measurements is the measurement that occurs most often (with highest frequency)
		\item[]
		\item Occasionally there is more than one mode 
		\item[]
		\item Example set of measurements: $\{1, 2, 3, 3, 4, 5, 5, 5, 5, 6, 6, 7, 7, 8\}$\\
		The mode is 5
	\end{itemize}
\end{frame}


\begin{frame}{The mode}
	\begin{itemize}
		\item The \emph{\textbf{mode}} of a set of measurements is the measurement that occurs most often (with highest frequency)
	\end{itemize}
	\begin{center}
			\includegraphics[width=0.7\linewidth]{../lecture3_DescriptiveStatistics/histRightSkew}
	\end{center}
\end{frame}

\begin{frame}{Pause for data}
dataset from the 1992 \emph{Statistical Abstract of the United States}. Selected Mountain West states only.
	\begin{center}
			\begin{tabular}{|c|c|c|c|}
			\hline 
			\textbf{State} & \textbf{SAT Math} & \textbf{\$ per pupil (1,000's)} & \textbf{Teacher pay (1,000's)} \\ 
			\hline \hline
			AZ & 497  & 4.231 & 30 \\ \hline 
			CO & 513  & 4.809 & 31 \\ \hline 
			ID & 502  & 3.200 & 25  \\  \hline 
			MT & 523  & 5.184 & 26 \\ \hline 
			NV & 487  & 4.564 & 32 \\ \hline 
			NM & 527  & 4.446 & 26 \\ \hline 
			UT & 539  & 2.993 & 25 \\ \hline 
			WY & 519  & 5.255 & 29 \\ \hline 
		\end{tabular} 
	\end{center}
\end{frame}

\begin{frame}{The mode}
What is the mode for teacher pay?
	\begin{center}
	\begin{tabular}{|c|c|}
		\hline 
		\textbf{State} & \textbf{Teacher pay (1,000's)} \\ 
		\hline \hline
		AZ & 30 \\ \hline 
		CO &  31 \\ \hline 
		ID & 25  \\  \hline 
		MT &  26 \\ \hline 
		NV & 32 \\ \hline 
		NM &  26 \\ \hline 
		UT &  25 \\ \hline 
		WY &  29 \\ \hline 
	\end{tabular} 
\end{center}
\end{frame}

\begin{frame}{The mode}
	The mode values are 25 and 26.
	\begin{center}
		\begin{tabular}{|c|c|}
			\hline 
			\textbf{State} & \textbf{Teacher pay (1,000's)} \\ 
			\hline \hline
			AZ & 30 \\ \hline 
			CO &  31 \\ \hline 
			ID & 25  \\  \hline 
			MT &  26 \\ \hline 
			NV & 32 \\ \hline 
			NM &  26 \\ \hline 
			UT &  25 \\ \hline 
			WY &  29 \\ \hline 
		\end{tabular} 
	\end{center}
\end{frame}

\subsection{The median}
\begin{frame}{Outline}
	\tableofcontents[currentsection,subsectionstyle=show/shaded/hide]
\end{frame}

\begin{frame}{The median}
	\begin{itemize}
		\item The \emph{\textbf{median}} of a set of measurements is the measurement that is the middle value when the measurements are arranged from least to greatest
		\item[]
		\item If there are an even number we take the average of two numbers
	\end{itemize}
\end{frame}

\begin{frame}{The median}
	What is the median of teacher pay? First, please order the numbers from least to greatest, then select the middle number (or average of middle two)
	\begin{center}
		\begin{tabular}{|c|c|}
			\hline 
			\textbf{State} & \textbf{Teacher pay (1,000's)} \\ 
			\hline \hline
			AZ & 30 \\ \hline 
			CO &  31 \\ \hline 
			ID & 25  \\  \hline 
			MT &  26 \\ \hline 
			NV & 32 \\ \hline 
			NM &  26 \\ \hline 
			UT &  25 \\ \hline 
			WY &  29 \\ \hline 
		\end{tabular} 
	\end{center}
\end{frame}

\begin{frame}{The median}
		\begin{center}
		\begin{tabular}{|c|c|}
			\hline 
			\textbf{State} & \textbf{Teacher pay (1,000's)} \\ 
			\hline \hline
			AZ & 30 \\ \hline 
			CO &  31 \\ \hline 
			ID & 25  \\  \hline 
			MT &  26 \\ \hline 
			NV & 32 \\ \hline 
			NM &  26 \\ \hline 
			UT &  25 \\ \hline 
			WY &  29 \\ \hline 
		\end{tabular} 
	\end{center}
	\begin{itemize}
		\item Ordered: 	$\{25, 25, 26, \textbf{26}, \textbf{29}, 30, 31, 32\} $
		\item[]
		\item $(26+29)/2 = 27.5$
	\end{itemize}
\end{frame}

\subsection{The mean}
\begin{frame}{Outline}
	\tableofcontents[currentsection,subsectionstyle=show/shaded/hide]
\end{frame}

\begin{frame}{The mean}
	\begin{itemize}
		\item The \emph{\textbf{mean}} (or arithmetic mean) of a set of measurements is the sum of all measurements divided by the total number of measurements 
		\item Let's calculate the mean of teacher pay now!
	\end{itemize}
		\begin{center}
			\begin{tabular}{|c|c|}
				\hline 
				\textbf{State} & \textbf{Teacher pay (1,000's)} \\ 
				\hline \hline
				AZ & 30 \\ \hline 
				CO &  31 \\ \hline 
				ID & 25  \\  \hline 
				MT &  26 \\ \hline 
				NV & 32 \\ \hline 
				NM &  26 \\ \hline 
				UT &  25 \\ \hline 
				WY &  29 \\ \hline 
			\end{tabular} 
		\end{center}
\end{frame}

\begin{frame}{The mean}
	\begin{itemize}
		\item Let's calculate the mean of teacher pay now!
	\end{itemize}
	\begin{center}
		\begin{tabular}{|c|c|}
			\hline 
			\textbf{State} & \textbf{Teacher pay (1,000's)} \\ 
			\hline \hline
			AZ & 30 \\ \hline 
			CO &  31 \\ \hline 
			ID & 25  \\  \hline 
			MT &  26 \\ \hline 
			NV & 32 \\ \hline 
			NM &  26 \\ \hline 
			UT &  25 \\ \hline 
			WY &  29 \\ \hline 
		\end{tabular} 
	\end{center}
\begin{gather*}
	\frac{30+31+25+26+32+26+25+29}{8} = \frac{224}{8} = 28.0
\end{gather*}
\end{frame}

\begin{frame}{Notation}
	\begin{itemize}
		\item We use some notation to simplify our calculations
		\item[]
		\item You do need to know this notation...
		\item[]
		\item \textbf{\emph{Summation:}} If we have $n$ measurements which we label $\{i = 1, 2, \hdots, n\}$, and the measurements are denoted by $y_i$, then:
		\begin{gather*}
		\sum_{i=1}^{n} y_i = y_1 + y_2 + y_3 + \hdots + y_n
		\end{gather*}
	\end{itemize}
\end{frame}

\begin{frame}{Notation example}
	Note with this example that $n = 8$.
		\begin{center}
		\begin{tabular}{|c|c|c|c|}
			\hline 
			\textbf{State} & \textbf{Teacher pay (1,000's)} & $y_i$ \\ 
			\hline \hline
			AZ & 30 & $y_1$\\ \hline 
			CO &  31 & $y_2$ \\ \hline 
			ID & 25 & $y_3$ \\  \hline 
			MT &  26 & $y_4$ \\ \hline 
			NV & 32 & $y_5$ \\ \hline 
			NM &  26 & $y_6$\\ \hline 
			UT &  25 & $y_7$\\ \hline 
			WY &  29 & $y_8$\\ \hline 
		\end{tabular} 
	\end{center}
\vspace{-7 pt}
\begin{gather*}
	\sum_{i=1}^{n} y_i = y_1 + y_2 + \hdots + y_8 \\
		\sum_{i=1}^{n} y_i = 30+31+25+26+32+26+25+29 = 224
\end{gather*}
\end{frame}

\begin{frame}{Notation example}
	Note with this example that $n = 8$.
	\begin{center}
		\begin{tabular}{|c|c|c|c|}
			\hline 
			\textbf{State} & \textbf{Teacher pay (1,000's)} & $y_i$ \\ 
			\hline \hline
			AZ & 30 & $y_1$\\ \hline 
			CO &  31 & $y_2$ \\ \hline 
			ID & 25 & $y_3$ \\  \hline 
			MT &  26 & $y_4$ \\ \hline 
			NV & 32 & $y_5$ \\ \hline 
			NM &  26 & $y_6$\\ \hline 
			UT &  25 & $y_7$\\ \hline 
			WY &  29 & $y_8$\\ \hline 
		\end{tabular} 
	\end{center}
	\begin{gather*}
	\sum_{i=1}^{n} y_i = 224
	\end{gather*}
\end{frame}

\begin{frame}{Notation}
	\begin{itemize}
		\item Occasionally the $n$ on top is omitted:
		\begin{gather*}
		\sum_{i=1}^{n} y_i = \sum_{i=1} y_i 
		\end{gather*}
		\item[]
		\item A population mean is denoted $\mu$
		\item[]
		\item A sample mean is often denoted $\bar{y}$, so:
		\begin{gather*}
		\bar{y} = \frac{\sum_{i=1}^{n} y_i}{n}
		\end{gather*}
	\end{itemize}
\end{frame}

\begin{frame}{The mean}{Foreshadowing}
	\begin{itemize}
		\item Suppose we knew the weight of every fish in the pond by the horseshoe. This group of fish would constitute a \textbf{population}
		\item The weights of the fish are: \\
 498.64 499.59 510.11 498.42 478.43 504.99 492.45 507.79 507.55 489.00 501.67 499.71 518.76 502.45 507.02 499.85
498.57 503.21 501.22 494.05 495.58 502.91 507.24 504.60 501.85 502.34 505.93 520.01 481.63 491.38 515.83 501.55
 497.25 507.88 497.77 513.92 495.11 501.37 500.04 492.73 492.79 498.09 513.35 503.56 508.43 507.75 500.80 493.27
518.36 497.93 494.60 484.99 502.68 500.34 490.05 492.30 494.22 490.67 478.23 494.86 500.89 502.94 506.92 483.88
501.88 500.77 498.26 486.01 496.31 504.51 516.26 519.23 498.37 495.61 511.91 516.70 511.33 484.83 527.31 504.99
 485.08 509.22 494.83 521.05 491.13 498.32 509.09 486.05 508.14 490.43 496.30 514.79 493.96 497.09 482.16 522.58
 493.33 504.78 488.39 496.20
	\end{itemize}
\end{frame}

\begin{frame}{The mean}{Foreshadowing}
	\begin{itemize}
		\item Suppose we knew the weight of every fish in the pond by the horseshoe. This group of fish would constitute a \textbf{population}
		\item[]
		\item The population mean weight is, $\mu = 500.595$
		\item[]
		\item If I take a random sample of ten of these: \\
		502.94 490.43 497.77 498.64 503.56 516.70 497.93 481.63 499.59 501.37
		\item[]
		\item The sample mean would be $\bar{y} = 499.056$
		\item[]
		\item Wow this is close!
	\end{itemize}
\end{frame}

\begin{frame}{The mean}{Outliers}
\begin{itemize}
	\item As a measure of the center of a distribution, the mean is susceptible to extreme influence from extreme values (called \textbf{\emph{outliers}})
\end{itemize}
\begin{center}
	\includegraphics[width=0.7\linewidth]{fishweight}
\end{center}
\end{frame}

\begin{frame}{The mean}{Outliers}
\begin{itemize}
	\item As a measure of the center of a distribution, the mean is susceptible to extreme influence from extreme values (called \textbf{\emph{outliers}})
\end{itemize}
\begin{center}
	\includegraphics[width=0.7\linewidth]{fishweightMeans}
\end{center}
\end{frame}

\begin{frame}{The mean}{Outliers \& Trimmed means}
\begin{itemize}
	\item As a measure of the center of a distribution, the mean is susceptible to extreme influence from extreme values (called \textbf{\emph{outliers}})
	\item[]
	\item \textbf{\emph{Trimmed mean:}} A trimmed mean is computed by first removing the least and greatest 5\% (or 10\%) of the values, and then computing the mean
	\item[]
	\item Removing data from any analysis always requires justification--I don't recommend using trimmed means
\end{itemize}
\end{frame}

\begin{frame}{The median handling outliers}
\begin{itemize}
	\item The median is less susceptible to extreme influence from extreme values (called \textbf{\emph{outliers}})
\end{itemize}
\begin{center}
	\includegraphics[width=0.7\linewidth]{fishweightMedians}
\end{center}
\end{frame}

\subsection{Skewness and measures of central tendency}
\begin{frame}{Outline}
\tableofcontents[currentsection,subsectionstyle=show/shaded/hide]
\end{frame}

\begin{frame}{Skewness and measures of central tendency}
	\begin{center}
		\includegraphics[width=0.9\linewidth]{allSkew}
	\end{center}
\end{frame}

\begin{frame}{Summary facts about measures of central tendency}
	\begin{itemize}
		\item The mode:
		\begin{itemize}
			\item It is the most frequent / probable measurement in the dataset
			\item There can be more than one for a dataset
			\item It is not influenced by outliers
			\item It is applicable for both qualitative and quantitative data
		\end{itemize}
		\item The median:
		\begin{itemize}
			\item It is the middle value; 50\% of the measurements lie above it and 50\% below it
			\item There is only one for a dataset
			\item It is not influenced by outliers
			\item It is applicable to quantitative data only 
		\end{itemize}
		\item The mean:
		\begin{itemize}
			\item It is the arithmetic average of the measurements in the dataset
			\item There is only one for a dataset
			\item The value is influenced by outliers
			\item It is applicable to quantitative data only 
		\end{itemize}			
	\end{itemize}
\end{frame}

\section{Measures of variability}
\begin{frame}{Outline}
	\tableofcontents[currentsection,subsectionstyle=show/shaded/hide]
\end{frame}

\begin{frame}{Variability}{Why do we care?}
Here's three distributions with the same mean, median, and mode:
\begin{center}
	\includegraphics[width=0.7\linewidth]{VarDens}
\end{center}
\end{frame}

\begin{frame}{Variability}{Why do we care?}
The inferences we make when comparing groups can be totally different:
\begin{center}
\includegraphics[width=1\linewidth]{histNormal2b}
\end{center}
\end{frame}

\subsection{The Range}
\begin{frame}{Outline}
	\tableofcontents[currentsection,subsectionstyle=show/shaded/hide]
\end{frame}

\begin{frame}{Some more notation}
	\begin{itemize}
		\item If we have $n$ measurements which we label $\{i = 1, 2, \hdots, n\}$ and the measurements are denoted by $y_i$, then $y_{(1)}$ is the smallest measurement (the \textbf{\emph{minimum}}) and $y_{(n)}$ is the largest (the \textbf{\emph{maximum}})
			\begin{center}
			\begin{tabular}{|c|c|c|c|}
				\hline 
				\textbf{State} & \textbf{Teacher pay (1,000's)} & $y_i$ \\ 
				\hline \hline
				AZ & 30 & $y_1$\\ \hline 
				CO &  31 & $y_2$ \\ \hline 
				ID & 25 & $y_3$ \\  \hline 
				MT &  26 & $y_4$ \\ \hline 
				NV & 32 & $y_5$ \\ \hline 
				NM &  26 & $y_6$\\ \hline 
				UT &  25 & $y_7$\\ \hline 
				WY &  29 & $y_8$\\ \hline 
			\end{tabular} 
		\end{center}
	\item[]
	\item Let's calculate the minimum, $y_{(1)}$ and maximum $y_{(n)}$
	\end{itemize}
\end{frame}

\begin{frame}{Some more notation}
	\vspace{-5pt}
	\begin{itemize}
		\item The minimum, $y_{(1)} = 25$
		\item The maximum, $y_{(n)} = 32$
		\vspace{5 pt}
		\begin{center}
			\begin{tabular}{|c|c|c|c|}
				\hline 
				\textbf{State} & \textbf{Teacher pay (1,000's)} & $y_i$ \\ 
				\hline \hline
				AZ & 30 & $y_1$\\ \hline 
				CO &  31 & $y_2$ \\ \hline 
				ID & 25 & $y_3$ \\  \hline 
				MT &  26 & $y_4$ \\ \hline 
				NV & 32 & $y_5$ \\ \hline 
				NM &  26 & $y_6$\\ \hline 
				UT &  25 & $y_7$\\ \hline 
				WY &  29 & $y_8$\\ \hline 
			\end{tabular} 
		\end{center}
	\end{itemize}
\end{frame}

\begin{frame}{The Range}
	\begin{itemize}
		\item The \textbf{\emph{range}} of a set of measurements is defined to be the difference between the largest and the smallest measurements of the set
		\item Formula: $range = y_{(n)} - y_{(1)}$
		\item Let's calculate the range of teacher pay:
	\end{itemize}
	\begin{center}
		\begin{tabular}{|c|c|}
			\hline 
			\textbf{State} & \textbf{Teacher pay (1,000's)} \\ 
			\hline \hline
			AZ & 30 \\ \hline 
			CO &  31 \\ \hline 
			ID & 25  \\  \hline 
			MT &  26 \\ \hline 
			NV & 32 \\ \hline 
			NM &  26 \\ \hline 
			UT &  25 \\ \hline 
			WY &  29 \\ \hline 
		\end{tabular} 
	\end{center}
\end{frame}

\begin{frame}{The Range}
Let's calculate the range of teacher pay:
	\begin{center}
		\begin{tabular}{|c|c|}
			\hline 
			\textbf{State} & \textbf{Teacher pay (1,000's)} \\ 
			\hline \hline
			AZ & 30 \\ \hline 
			CO &  31 \\ \hline 
			ID & 25  \\  \hline 
			MT &  26 \\ \hline 
			NV & 32 \\ \hline 
			NM &  26 \\ \hline 
			UT &  25 \\ \hline 
			WY &  29 \\ \hline 
		\end{tabular} 
	\end{center}
	\begin{gather*}
		range = y_{(n)} - y_{(i)} \\
		range = 32 - 25 = 7
	\end{gather*}
\end{frame}

\subsection{Percentiles}
\begin{frame}{Outline}
	\tableofcontents[currentsection,subsectionstyle=show/shaded/hide]
\end{frame}

\begin{frame}{Percentiles}
	\begin{itemize}
		\item The $p$\textsuperscript{th} \textbf{\emph{percentile}} of a set of $n$ measurements arranged in order of from least to greatest is the value that has at most $p$\% of the measurements below it and at most $(100 - p$)\% above it
		\item[]
		\item This gets complicated...
		\item[]
		\item To calculate the percentile of one of the measurements, $y_{(i)}$ we use the formula: $100(i - 0.5)/n$
	\end{itemize}
\end{frame}

\begin{frame}{Percentiles}
	Let's calculate the percentiles for the teacher pay data:
		\begin{center}
		\begin{tabular}{|c|c|}
			\hline 
			\textbf{State} & \textbf{Teacher pay (1,000's)} \\ 
			\hline \hline
			AZ & 30 \\ \hline 
			CO &  31 \\ \hline 
			ID & 25  \\  \hline 
			MT &  26 \\ \hline 
			NV & 32 \\ \hline 
			NM &  26 \\ \hline 
			UT &  25 \\ \hline 
			WY &  29 \\ \hline 
		\end{tabular} 
	\end{center}
\end{frame}

\begin{frame}{Percentiles}
\begin{itemize}
	\item First, we will reorder the data
	\item Remember $n = 8$
\end{itemize}
	\begin{center}
		\begin{tabular}{|c|c|c|c|}
			\hline 
			\textbf{State} & \textbf{Teacher pay (1,000's)} & $y_{(i)}$ & $i$ \\ 
			\hline \hline 
			ID & 25 & $y_{(1)}$& 1 \\  \hline 
			UT &  25  & $y_{(2)}$& 2\\ \hline 
			MT &  26  & $y_{(3)}$& 3\\ \hline 
			NM &  26  & $y_{(4)}$& 4\\ \hline 
			WY &  29  & $y_{(5)}$& 5\\ \hline 
			AZ & 30  & $y_{(6)}$& 6\\ \hline 
			CO &  31 & $y_{(7)}$& 7\\ \hline
			NV & 32  & $y_{(8)}$& 8\\ \hline 
		\end{tabular} 
	\end{center}
\end{frame}

\begin{frame}{Percentiles}
	\begin{itemize}
		\item Now we add the formula
	\end{itemize}
	\begin{center}
		\begin{tabular}{|c|c|c|c|c|}
			\hline 
			\textbf{State} & \textbf{Teacher pay (1,000's)} & $y_{(i)}$ & $i$ & $100(i - 0.5)/n$ \\ 
			\hline \hline 
			ID & 25 & $y_{(1)}$& 1 & $100(1 - 0.5)/8 = 6.25$ \\  \hline 
			UT &  25  & $y_{(2)}$& 2& $100(2 - 0.5)/8 = 18.75$\\ \hline 
			MT &  26  & $y_{(3)}$& 3& $100(3 - 0.5)/8 = 31.25$\\ \hline 
			NM &  26  & $y_{(4)}$& 4& $100(4 - 0.5)/8 = 43.75$\\ \hline 
			WY &  29  & $y_{(5)}$& 5& $100(5 - 0.5)/8 = 56.25$\\ \hline 
			AZ & 30  & $y_{(6)}$& 6& $100(6 - 0.5)/8 = 68.75$\\ \hline 
			CO &  31 & $y_{(7)}$& 7& $100(7 - 0.5)/8 = 81.25$\\ \hline
			NV & 32  & $y_{(8)}$& 8& $100(8 - 0.5)/8 = 93.75$\\ \hline 
		\end{tabular} 
	\end{center}
\end{frame}

\subsection{Quartiles}
\begin{frame}{Quartiles}
	\begin{itemize}
		\item We just discussed how when given a measurement you can determine the percentile of that measurement
		\item[]
		\item We can also go the other direction (determine the measurement value for a given percentile)
		\item[]
		\item We have already done this with the median (it is the 50\textsuperscript{th} percentile)
		\item[]
		\item In addition to the median, we are often interested in the quartiles of a set of measurements		
	\end{itemize}
\end{frame}

\begin{frame}{Quartiles}
	\begin{itemize}
		\item \textbf{\emph{Quartiles:}} the quartiles of a set of measurements are the three numbers (the 25\textsuperscript{th}, 50\textsuperscript{th}, and 75\textsuperscript{th} percentiles) that break the dataset into four parts
			\item[]
		\item The process for finding the 25\textsuperscript{th} percentile (called Q1 or the first quartile), and the 75\textsuperscript{th} percentile (called Q3 or the third quartile) is actually easy:
			\begin{enumerate}
			\item We find the median (this splits the dataset into two sets: the measurements below the median and the measurements above)
			\item We find the median of the first set (measurements below the median). This value is Q1
			\item We find the median of the second set (measurements above the median). This value is Q3
		\end{enumerate}
	\item[]
	\item If there is an odd number of measurements in the dataset, include the median in both the first and second sets
	\end{itemize}
\end{frame}

\begin{frame}{Quartiles}
	\begin{center}
		\begin{tabular}{|c|c|c|c|}
			\hline 
			\textbf{State} & \textbf{Teacher pay (1,000's)} & $y_{(i)}$ & $i$ \\ 
			\hline \hline 
			ID & 25 & $y_{(1)}$& 1 \\  \hline 
			UT &  25  & $y_{(2)}$& 2\\ \hline 
			MT &  26  & $y_{(3)}$& 3\\ \hline 
			NM &  26  & $y_{(4)}$& 4\\ \hline 
			WY &  29  & $y_{(5)}$& 5\\ \hline 
			AZ & 30  & $y_{(6)}$& 6\\ \hline 
			CO &  31 & $y_{(7)}$& 7\\ \hline
			NV & 32  & $y_{(8)}$& 8\\ \hline 
		\end{tabular} 
	\end{center}
\begin{itemize}
	\item The set of numbers below the median is $\{25, 25, 26, 26\}$
	\item The median of $\{25, 25, 26, 26\}$ is 25.5. So this is Q1
	\item The set of numbers below the median is $\{25, 25, 26, 26\}$
	\item The median of $\{29, 30, 31, 32\}$ is 30.5. So this is Q3
\end{itemize}
\end{frame}

\subsection{The Interquartile Range}
\begin{frame}{Outline}
	\tableofcontents[currentsection,subsectionstyle=show/shaded/hide]
\end{frame}

\begin{frame}{The Interquartile Range}
	\begin{itemize}
		\item \textbf{\emph{Interquartile Range:}} The interquartile range (IQR) of a set of measurements is the difference between Q3 (75th percentile) and Q1 (25th percentile)
		\item Let's calculate the IQR for the teacher pay data:
			\begin{center}
			\begin{tabular}{|c|c|c|c|}
				\hline 
				\textbf{State} & \textbf{Teacher pay (1,000's)} & $y_{(i)}$ & $i$ \\ 
				\hline \hline 
				ID & 25 & $y_{(1)}$& 1 \\  \hline 
				UT &  25  & $y_{(2)}$& 2\\ \hline 
				MT &  26  & $y_{(3)}$& 3\\ \hline 
				NM &  26  & $y_{(4)}$& 4\\ \hline 
				WY &  29  & $y_{(5)}$& 5\\ \hline 
				AZ & 30  & $y_{(6)}$& 6\\ \hline 
				CO &  31 & $y_{(7)}$& 7\\ \hline
				NV & 32  & $y_{(8)}$& 8\\ \hline 
			\end{tabular} 
		\end{center}
	\end{itemize}
\end{frame}

\begin{frame}{The Interquartile Range}
Let's calculate the IQR for the teacher pay data:
		\begin{center}
			\begin{tabular}{|c|c|c|c|}
				\hline 
				\textbf{State} & \textbf{Teacher pay (1,000's)} & $y_{(i)}$ & $i$ \\ 
				\hline \hline 
				ID & 25 & $y_{(1)}$& 1 \\  \hline 
				UT &  25  & $y_{(2)}$& 2\\ \hline 
				MT &  26  & $y_{(3)}$& 3\\ \hline 
				NM &  26  & $y_{(4)}$& 4\\ \hline 
				WY &  29  & $y_{(5)}$& 5\\ \hline 
				AZ & 30  & $y_{(6)}$& 6\\ \hline 
				CO &  31 & $y_{(7)}$& 7\\ \hline
				NV & 32  & $y_{(8)}$& 8\\ \hline 
			\end{tabular} 
		\end{center}
	\begin{gather*}
		IQR = Q3 - Q1 = 30.5 - 25.5 = 5
	\end{gather*}
\end{frame}

\subsection{The Variance \& Standard Deviation}
\begin{frame}{Outline}
	\tableofcontents[currentsection,subsectionstyle=show/shaded/hide]
\end{frame}

\begin{frame}{Deviations}
	\begin{itemize}
		\item We are going to discuss two very important measures of variability, the variance and the standard deviation
		\item[]
		\item To get there first we need to discuss deviations
		\item[]
		\item \textbf{\emph{Deviation:}} A deviation is defined as the difference between a measurement and the average of measurements
		\item[]
		\item Formula: $y_i - \bar{y}$
	\end{itemize}
\end{frame}

\begin{frame}{Deviations from the mean}
	\begin{itemize}
		\item Let's calculate the deviations from the mean for the teacher pay data.
		\item Remember the mean is 28.0.
		\item 
			\begin{center}
			\begin{tabular}{|c|c|c|}
				\hline 
				\textbf{State} & \textbf{Teacher pay (1,000's)} &  $y_i - \bar{y}$ \\ 
				\hline \hline
				AZ & 30 &\\ \hline 
				CO &  31 &\\ \hline 
				ID & 25  &\\  \hline 
				MT &  26 &\\ \hline 
				NV & 32 &\\ \hline 
				NM &  26 &\\ \hline 
				UT &  25 &\\ \hline 
				WY &  29 &\\ \hline 
			\end{tabular} 
		\end{center}
	\end{itemize}
\end{frame}

\begin{frame}{Deviations from the mean}
	\begin{itemize}
		\item Let's calculate the deviations from the mean for the teacher pay data.
		\item Remember the mean is 28.0.
\vspace{5 pt}
		\begin{center}
			\begin{tabular}{|c|c|c|}
				\hline 
				\textbf{State} & \textbf{Teacher pay (1,000's)} &  $y_i - \bar{y}$ \\ 
				\hline \hline
				AZ & 30 & 2\\ \hline 
				CO &  31 & 3\\ \hline 
				ID & 25  & -3\\  \hline 
				MT &  26 & -2\\ \hline 
				NV & 32 & 4\\ \hline 
				NM &  26 & -2\\ \hline 
				UT &  25 & -3\\ \hline 
				WY &  29 & 1\\ \hline 
			\end{tabular} 
		\end{center}
	\end{itemize}
\end{frame}

\begin{frame}{The Sample Variance}
\begin{itemize}
	\item \emph{\textbf{Variance:}} The variance of a variable in a population is the mean of the squared deviations from the population mean.
	\item[]
	\item Denoted $\sigma^2$ for a population and $s^2$ for a sample 
	\item[]
	\item However, we will use a slightly adjusted formula* for computing the \textbf{\emph{sample variance}}, which we denote $s^2$:
	\begin{gather*}
		s^2 = \frac{\sum_{i=1}^n (y_i - \bar{y})^2}{n-1}
	\end{gather*}
	* An unadjusted version would only divide by $n$ rather than $n-1$
\end{itemize}
\end{frame}

\begin{frame}{The Sample Variance}
 Let's calculate the sample variance:
		\begin{center}
			\begin{tabular}{|c|c|c|c|}
				\hline 
				\textbf{State} & \textbf{Teacher pay (1,000's)} &  $y_i-\bar{y}$ & $(y_i-\bar{y})^2$\\ 
				\hline \hline
				AZ & 30 & 2 &\\ \hline 
				CO &  31 & 3 &\\ \hline 
				ID & 25  & -3 &\\  \hline 
				MT &  26 & -2 &\\ \hline 
				NV & 32 & 4 &\\ \hline 
				NM &  26 & -2 &\\ \hline 
				UT &  25 & -3 &\\ \hline 
				WY &  29 & 1 &\\ \hline 
			\end{tabular} 
		\end{center}
\end{frame}

\begin{frame}{The Standard Deviation}
	\begin{itemize}
		\item \textbf{\emph{Standard deviation:}} The standard deviation is the square root of the variance 
		\item[]
		\item Denoted $\sigma$ for a population and $s$ for a sample
	\end{itemize}
\end{frame}

\begin{frame}{The Sample Standard Deviation}
	Let's calculate the sample standard deviation:
	\begin{center}
		\begin{tabular}{|c|c|c|c|}
			\hline 
			\textbf{State} & \textbf{Teacher pay (1,000's)} &  $y_i-\bar{y}$ & $(y_i-\bar{y})^2$\\ 
			\hline \hline
			AZ & 30 & 2 &4\\ \hline 
			CO &  31 & 3 &9\\ \hline 
			ID & 25  & -3 &9\\  \hline 
			MT &  26 & -2 &4\\ \hline 
			NV & 32 & 4 &16\\ \hline 
			NM &  26 & -2 &4\\ \hline 
			UT &  25 & -3 &9\\ \hline 
			WY &  29 & 1 &1\\ \hline 
		\end{tabular} 
	\end{center}
	\begin{gather*}
	s = \sqrt{s^2} = \sqrt{8} = 2.828
	\end{gather*}
\end{frame}

\subsection{The Coefficient of Variation}
\begin{frame}{Outline}
	\tableofcontents[currentsection,subsectionstyle=show/shaded/hide]
\end{frame}

\begin{frame}{The Coefficient of Variation}
	\begin{itemize}
		\item \textbf{\emph{Coefficient of Variation:}} The coefficient of variation measures the variability of values in a population (or sample) relative to the magnitude of the population (or sample) mean 
		\item[]
		\item Formula:
		\begin{align*}
			CV_{population} = \frac{\sigma}{|\mu|} \quad CV_{sample} = \frac{s}{|\bar{y}|}
		\end{align*}
		\item We often express the CV as a percent
		\item[]
		\item Example. I pipette 10$\mu$L 4 times and the actual volume is: $\{10.03, 9.98, 10.01, 9.97\}$. What is the CV? $CV = 0.02753785 / 9.9975 = 0.002754474 = 0.28 \%$
	\end{itemize}
\end{frame}

\subsection{The Median Absolute Deviation}
\begin{frame}{Outline}
	\tableofcontents[currentsection,subsectionstyle=show/shaded/hide]
\end{frame}

\begin{frame}{The Median Absolute Deviation}
	\begin{itemize}
		\item \textbf{\emph{Median absolute deviation:}} The median absolute deviation, or MAD, is the median of the absolute deviations of the $n$ measurements about the median divided by 0.6745
		\item[]
		\item Formula: $MAD = median(\{|y_1-\tilde{y}|,|y_2 - \tilde{y}|,\hdots,|y_n - \tilde{y}|\}) / 0.6745$
		where $\tilde{y}$ is the median of the set of measurements
	\end{itemize}
\end{frame}

\begin{frame}{The Median Absolute Deviation}
	Let's calculate the median absolute deviation of the teacher pay data (recall that the median is 27.5):
	\begin{center}
		\begin{tabular}{|c|c|c|}
			\hline 
			\textbf{State} & \textbf{Teacher pay (1,000's)} &  $|y_i-\tilde{y}|$ \\ 
			\hline \hline
			AZ & 30 & \\ \hline 
			CO &  31 & \\ \hline 
			ID & 25  & \\  \hline 
			MT &  26 & \\ \hline 
			NV & 32 & \\ \hline 
			NM &  26 & \\ \hline 
			UT &  25 & \\ \hline 
			WY &  29 & \\ \hline 
		\end{tabular} 
	\end{center}
\end{frame}

\begin{frame}{The Median Absolute Deviation}
	Let's calculate the median absolute deviation of the teacher pay data:
	\begin{center}
		\begin{tabular}{|c|c|c|}
			\hline 
			\textbf{State} & \textbf{Teacher pay (1,000's)} &  $|y_i-\tilde{y}|$ \\ 
			\hline \hline
			AZ & 30 & 2.5\\ \hline 
			CO &  31 & 3.5\\ \hline 
			ID & 25  & 2.5\\  \hline 
			MT &  26 & 1.5 \\ \hline 
			NV & 32 & 4.5\\ \hline 
			NM &  26 & 1.5\\ \hline 
			UT &  25 & 2.5\\ \hline 
			WY &  29 & 1.5\\ \hline 
		\end{tabular} 
	\end{center}
\begin{gather*}
	MAD = median(\{|y_1-\tilde{y}|,|y_2 - \tilde{y}|,\hdots,|y_n - \tilde{y}|\}) / 0.6745 \\
	MAD = 2.5 / 0.6745 = 3.7065
\end{gather*}
\end{frame}

\section{The Empirical rule}
\begin{frame}{Outline}
	\tableofcontents[currentsection,subsectionstyle=show/shaded/hide]
\end{frame}

\begin{frame}{The Empirical rule}
	\begin{itemize}
		\item \textbf{\emph{The empirical rule:}} Given a set of $n$ measurements having a ``bell shape'' or mound-shaped distribution, the $\bar{y} \pm s$ interval contains approximately 68\% of the measurements; the $\bar{y} \pm 2s$ interval contains approximately 95\%; the $\bar{y} \pm 3s$ interval contains approximately 99.7\%
	\end{itemize}
\end{frame}

\begin{frame}{The Empirical rule}
\begin{center}
	\includegraphics[width=0.95\linewidth]{fishEmp00}
\end{center}
\end{frame}

\begin{frame}{The Empirical rule}
	\begin{center}
		\includegraphics[width=0.95\linewidth]{fishEmp0}
	\end{center}
\end{frame}

\begin{frame}{The Empirical rule}
	\begin{center}
		\includegraphics[width=0.95\linewidth]{fishEmp}
	\end{center}
\end{frame}

\section{The Boxplot}
\begin{frame}{Outline}
	\tableofcontents[currentsection,subsectionstyle=show/shaded/hide]
\end{frame}

\begin{frame}{The Boxplot}
	\begin{itemize}
		\item The \textbf{\emph{boxplot}} is a graphical representation of a distribution that shows the quartiles of the distribution (Q1, median, and Q3), and demonstrates what values are outliers (and which are not)
		\item[] 
		\item Before we can draw:
		\begin{itemize}
			\item Quartiles: (Q1, Median, Q3)
			\item Lower inner fence (LIF): $Q_1 - 1.5 \times IQR$
			\item Lower outer fence (LOF): $Q_1 - 3 \times IQR$
			\item Upper inner fence (UIF): $Q_3 + 1.5 \times IQR$
			\item Upper outer fence (LIF): $Q_3 + 3 \times IQR$
		\end{itemize}
	\item[]
	\item Lower adjacent value: smallest data value that isn't an outlier (less than LIF)
	\item Upper adjacent value: greatest data value that isn't an outlier (greater than UIF)
	\end{itemize}
\end{frame}

\begin{frame}{The Boxplot}{Before we draw}
	\begin{itemize}
		\item Example data. Simulated flow rate data from the Salt River, Kentucky ($m^3 / s$): \\
		$\{0.14, 0.22, 0.45, 0.75, 0.91, 0.92, 1.15, 1.27, 2.80, 3.85\}$
		\item[]
		\item Median: 0.915
		\item Q1: 0.45
		\item Q3: 1.27
		\item Lower inner fence (LIF): $Q_1 - 1.5 \times IQR = -0.78 $
		\item Lower outer fence (LOF): $Q_1 - 3 \times IQR = -2.01$
		\item Upper inner fence (UIF): $Q_3 + 1.5 \times IQR = 2.50 $
		\item Upper outer fence (LIF): $Q_3 + 3 \times IQR = 3.73 $
		\item Lower adjacent value: $0.14$
		\item Upper adjacent value: $1.27$
	\end{itemize}
\end{frame}

\begin{frame}{The Boxplot}{Drawing step 1: The box}
	We draw a rectangle with the bottom at Q1 and the top at Q3. Width is your choice--be reasonable:
	\begin{center}
		\includegraphics[width=0.7\linewidth]{saltRiver1}
	\end{center}
\end{frame}

\begin{frame}{The Boxplot}{Drawing step 2: Add the median}
	Now we add a solid horizontal line at the median:
	\begin{center}
		\includegraphics[width=0.7\linewidth]{saltRiver2}
	\end{center}
\end{frame}

\begin{frame}{The Boxplot}{Drawing step 3: Draw lines to adjacent values}
	Now we draw a line below the box to the lower adjacent value, and above the box to the upper adjacent value. If the adjacent value equals the quartile we omit the line:
	\begin{center}
		\includegraphics[width=0.7\linewidth]{saltRiver3}
	\end{center}
\end{frame}

\begin{frame}{The Boxplot}{Drawing step 4: Add outliers}
	If any of the measurements lie between inner and outer fences (\textbf{\emph{mild outliers}}) we draw an open circle, if they lie beyond the outer fences (\textbf{\emph{extreme outliers}}) we draw a closed circle:
	\begin{center}
		\includegraphics[width=0.7\linewidth]{saltRiver4}
	\end{center}
\end{frame}

\begin{frame}{The Boxplot}{Advantages of Boxplots}
Shows: Measure of central tendency, measure of variability, presence / absence of outliers, and skewness. Also can be utilized to compare groups:

\begin{center}
	\includegraphics[width=0.7\linewidth]{fibrinogen}
\end{center}
\end{frame}

\section{Scatterplots and correlation}
\begin{frame}{Outline}
	\tableofcontents[currentsection,subsectionstyle=show/shaded/hide]
\end{frame}

\begin{frame}{Scatterplots}
	\begin{itemize}
		\item \textbf{\emph{Scatterplot:}} A scatterplot is a two-dimensional plot where the horizontal and vertical axes represent the measurement scales for two variables. Each point in the plot corresponds to the measurement of both variables from one experimental unit / study unit.
	\end{itemize}
\end{frame}

\begin{frame}{Scatterplots}
	The following data is from a clinical trial of a new therapeutic for HIV. The data represents the measurements of CD4\textsuperscript{+} T cells in a volume of blood:
	{\tiny
	\begin{center}
		\begin{tabular}{|c|c|}
			\hline
			   \textbf{Baseline} & \textbf{Post-treatment} \\ \hline \hline
			     2.12  &  2.47\\ \hline
			     4.35  & 4.61\\ \hline
			     3.39  &  5.26\\ \hline
			     2.51  & 3.02\\ \hline
			     4.04  &  6.36\\ \hline
			     5.10  &  5.93\\ \hline
			     3.77  &  3.93\\ \hline
			     3.35  &  4.09\\ \hline
			     4.10  &  4.88\\ \hline
			     3.35  &  3.81\\ \hline
			     4.15  &  4.74\\ \hline
			     3.56  &  3.29\\ \hline
			     3.39  &  5.55\\ \hline
			     1.88  &  2.82\\ \hline
			     2.56  &  4.23\\ \hline
			     2.96  &  3.23\\ \hline
			     2.49  &  2.56\\ \hline
			     3.03  &  4.31\\ \hline
			     2.66  &  4.37\\ \hline
			     3.00  &  2.40\\ \hline
		\end{tabular}
	\end{center}}
Ref: DiCiccio, T.J. and Efron B. (1996) Bootstrap confidence intervals (with Discussion). \emph{Statistical Science, 11}.
\end{frame}

\begin{frame}{Scatterplots}
``Using a scatterplot, the general shape and direction of the relationship between two quantitative variables can be displayed''
	\begin{center}
		\includegraphics[width=0.7\linewidth]{cd4}
	\end{center}
\end{frame}

\begin{frame}{The Correlation coefficient}
	\begin{itemize}
		\item \textbf{\emph{Correlation coefficient:}} The correlation coefficient measures the strength of the linear relationship between two quantitative variables. We denote it $r$
		\item[]
		\item Formula: 
		\begin{gather*}
			r = \frac{\sum_{i=1}(x_i -\bar{x})(y_i - \bar{y})}{(n-1) s_x s_y}
		\end{gather*}
	\end{itemize}
\end{frame}

\begin{frame}{The Correlation coefficient}
	\begin{center}
		\includegraphics[width=0.75\linewidth]{scatter}
	\end{center}
\end{frame}

\begin{frame}{The Correlation coefficient}
	\begin{itemize}
		\item Let's compute the correlation coefficient between teacher pay and math SAT scores
		\item Recall that if $x_i$ is teacher pay, $s_x = \sqrt{8}$
	\end{itemize}
		\begin{center}
		\begin{tabular}{|c|c|c|c|}
			\hline 
			\textbf{State} & \textbf{SAT Math} & \textbf{\$ per pupil (1,000's)} & \textbf{Teacher pay (1,000's)} \\ 
			\hline \hline
			AZ & 497  & 4.231 & 30 \\ \hline 
			CO & 513  & 4.809 & 31 \\ \hline 
			ID & 502  & 3.200 & 25  \\  \hline 
			MT & 523  & 5.184 & 26 \\ \hline 
			NV & 487  & 4.564 & 32 \\ \hline 
			NM & 527  & 4.446 & 26 \\ \hline 
			UT & 539  & 2.993 & 25 \\ \hline 
			WY & 519  & 5.255 & 29 \\ \hline 
		\end{tabular} 
	\end{center}
\end{frame}

\begin{frame}{The Correlation coefficient}
	\begin{itemize}
		\item Let's compute the correlation coefficient between teacher pay and math SAT scores
		\item Recall that if $x$ is teacher pay, $s_x = \sqrt{8}$ and $\bar{x} = 28$
	\end{itemize}
	\begin{center}
		\begin{tabular}{|c|c|c|}
			\hline 
			\textbf{State} & \textbf{SAT Math} & \textbf{Teacher pay (1,000's)} \\ 
			\hline \hline
			AZ & 497  & 30 \\ \hline 
			CO & 513  & 31 \\ \hline 
			ID & 502  & 25  \\  \hline 
			MT & 523  & 26 \\ \hline 
			NV & 487  & 32 \\ \hline 
			NM & 527  & 26 \\ \hline 
			UT & 539  & 25 \\ \hline 
			WY & 519  & 29 \\ \hline 
		\end{tabular} 
	\end{center}
\end{frame}

\begin{frame}{The Correlation coefficient}
	\begin{itemize}
		\item Recall that if $x$ is teacher pay, $s_x = \sqrt{8}$ and $\bar{x} = 28$
		\item If $y$ is math SAT score, $\bar{y} = 513.375$ and $s_y = 17.15424$
	\end{itemize}
		{\small 
		\begin{tabular}{|c|c|c|c|}
			\hline 
			\textbf{State} & \textbf{SAT Math} & \textbf{Teacher pay (1,000's)} & $(x_i -\bar{x})(y_i - \bar{y})$\\ 
			\hline \hline
			AZ & 497  & 30 & $2 (-16.375) = -32.750$ \\ \hline 
			CO & 513  & 31 & $3 (-0.375) = -1.125 $ \\ \hline 
			ID & 502  & 25 & $-3 (-11.375) = 34.125$ \\  \hline 
			MT & 523  & 26 & $-2 (9.625) = -19.250$ \\ \hline 
			NV & 487  & 32 & $4(-26.375) = -105.500 $ \\ \hline 
			NM & 527  & 26 & $-2 (13.625) = -27.250$\\ \hline 
			UT & 539  & 25 & $-3 (25.625) =  -76.875$\\ \hline 
			WY & 519  & 29 & $1 (5.625) = 5.625$\\ \hline 
		\end{tabular} }
\begin{align*}
	r = \frac{\sum_{i=1}(x_i -\bar{x})(y_i - \bar{y})}{(n-1) s_x s_y} = \frac{-223}{(7)(\sqrt{8})(17.15424)} = -0.657
\end{align*}
\end{frame}

\begin{frame}{The Correlation coefficient}
	\begin{itemize}
		\item What does $r = -0.657$ tell you about the relationship between teacher pay and math SAT scores?
	\end{itemize}
\end{frame}
              
\end{document}