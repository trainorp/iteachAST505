\documentclass[xcolor=dvipsnames]{beamer} 
\usetheme{AnnArbor}
\usecolortheme{beaver}

\usepackage{amsmath,graphicx,booktabs,tikz,subfig,color,lmodern}
\definecolor{mycol}{rgb}{.4,.85,1}
\setbeamercolor{title}{bg=mycol,fg=black} 
\setbeamercolor{palette primary}{use=structure,fg=white,bg=red}
\setbeamercolor{block title}{fg=white,bg=red!50!black}
% \setbeamercolor{block title}{fg=white,bg=blue!75!black}

\title[Lecture 6]{Lecture 6: Random Variables and Probability Distributions}
\author[Patrick Trainor]{Patrick Trainor, PhD, MS, MA}
\institute[NMSU]{New Mexico State University}
\date{September 9, 2019}

\begin{document}

\begin{frame}
	\maketitle
\end{frame}

\begin{frame}{Outline}
	\tableofcontents[hideallsubsections]
\end{frame}

\section{Random Variables}
\begin{frame}{Outline}
	\tableofcontents[currentsection,subsectionstyle=show/shaded/hide]
\end{frame}

\begin{frame}{Random Variables}
	\begin{itemize}
		\item \textbf{\emph{Random Variable:}} A variable whose values depend on outcomes of a stochastic process or phenomenon that is ``random''
		\item[]
		\item \textbf{\emph{Qualitative random variable:}} A random variable in which the possible responses are categories that do not correspond to numerical amounts
		\begin{itemize}
			\item Example: In a study of the relationship between sugary beverages and diabetes, diabetes status: \{yes, no\} would be a qualitative RV
			\item Example 2: In a study of gene variants and the color of peas, \{yellow, green, white\} would be a qualitative RV
			\item There is no natural ordering of the values. It doesn't make sense to say no $<$ yes or that yellow $<$ green $<$ white
			\item May also be called a nominal random variable or sometimes a categorical variable*
		\end{itemize}
	\end{itemize}
\end{frame}

\begin{frame}{Random Variables}
	\begin{itemize}
		\item \textbf{\emph{Quantitative random variable:}} A random variable in which the possible responses vary in numerical magnitude
		\begin{itemize}
			\item Example: In a study of the relationship between sugary beverages and diabetes, blood glucose concentration would be a quantitative RV
			\item There exists a natural ordering between the values. It does make sense to say 10 mg/dL $>$ 5 mg/dL
			\item May also be called a numeric random variable
		\end{itemize}
	\item[]
	\item Types of quantitative random variables:
	\begin{itemize}
		\item \textbf{\emph{Discrete random variable:}} When values of a quantitative rv can assume only a countable number of values, the variable is called a discrete random variable
		\item \textbf{\emph{Continuous random variable:}} When the values of a quantitative rv can assume any of an uncountable number of values on a line interval, the variable is a continuous random variable
	\end{itemize}
	\end{itemize}
\end{frame}

\section{Probability Distribution Functions}
\begin{frame}{Outline}
	\tableofcontents[currentsection,subsectionstyle=show/shaded/hide]
\end{frame}

\begin{frame}{A Probability Distribution}
	\begin{itemize}
		\item The sum of a roll of two dice is a discrete random variable
		\item Here are all of the possible rolls--remember these are not the sum yet!
		\begin{center}
			\includegraphics[width=0.8\linewidth]{../lecture5_EventsProbabilityLaws/allDice2}
		\end{center}
		\item Here are the number of ways to get possible sums:
		{ \tiny
		\begin{center}
			\begin{tabular}{|c|c|} \hline
					\textbf{Sum} & \textbf{\# of Ways} \\ \hline \hline
				      2 &    1\\ \hline
				     3  &   2\\ \hline
				     4  &   3\\ \hline
				     5  &   4\\ \hline
				     6  &   5\\ \hline
				     7  &   6\\ \hline
				     8  &   7\\ \hline
				     9  &   6\\ \hline
				    10  &   5\\ \hline
				    11  &   4\\ \hline
				    12  &   3\\ \hline
				    13  &   2\\ \hline
				    14  &   1\\ \hline
			\end{tabular}
		\end{center}}
	\end{itemize}
\end{frame}

\begin{frame}{A Probability Distribution}
	\begin{itemize}
		\item The sum of a roll of two dice is a discrete random variable 
		\item[]
		\item Now we add the relative frequency of each sum:
		{ \tiny
			\begin{center}
				\begin{tabular}{|c|c|c|} \hline
					\textbf{Sum} & \textbf{\# of Ways} & Relative Frequency  \\ \hline \hline
					2 &    1 & 0.020 \\ \hline
					3  &   2 & 0.041\\ \hline
					4  &   3 &  0.061\\ \hline
					5  &   4 &   0.082\\ \hline
					6  &   5 & 0.102\\ \hline
					7  &   6 & 0.122\\ \hline
					8  &   7 & 0.143\\ \hline
					9  &   6 & 0.122\\ \hline
					10  &   5 & 0.102\\ \hline
					11  &   4 & 0.082\\ \hline
					12  &   3 & 0.061\\ \hline
					13  &   2 & 0.041\\ \hline
					14  &   1 &  0.020\\ \hline
				\end{tabular}
		\end{center}}
	\item If we define the random variable $y$, where $y = 2, 3, 4, \hdots, 14$, then the probability distribution of $y$ is $P(y)$, where $P(2) = 0.077, P(3) = 0.154, P(4) = 0.231, \hdots, P(14) = 0.077$ 
	\end{itemize}
\end{frame}

\begin{frame}{A Probability Distribution Function}
	\begin{itemize}
		\item The sum of a roll of two dice is a discrete random variable 
		\vspace{10pt}
		{ \tiny
			\begin{center}
				\begin{tabular}{|c|c|} \hline
					$y $ & $P(y)$  \\ \hline \hline
					2 &   0.020 \\ \hline
					3  &   0.041\\ \hline
					4  &   0.061\\ \hline
					5  &  0.082\\ \hline
					6  &  0.102\\ \hline
					7  &  0.122\\ \hline
					8  & 0.143 \\ \hline
					9  &   0.122\\ \hline
					10  & 0.102\\ \hline
					11  & 0.082\\ \hline
					12  & 0.061 \\ \hline
					13  &  0.041 \\ \hline
					14  & 0.020\\ \hline
				\end{tabular}
		\end{center}}
	\vspace{10pt}
		\item \emph{\textbf{Probability Distribution Function:}} The probability distribution function of a random variable describes for each possible value of the random variable, the probability (or likelihood) of that value
	\end{itemize}
\end{frame}

\begin{frame}{A Probability Distribution Function}
	\begin{center}
		\includegraphics[width=0.9\linewidth]{diceRollPDF}
	\end{center}
\end{frame}

\begin{frame}{Histograms \& Probability Distribution Functions}
	The Histogram of real data often looks more and more like a probability distribution function as the number of samples gets large:
	\begin{center}
		\includegraphics[width=0.9\linewidth]{diceRollHist1}
	\end{center}
\end{frame}

\begin{frame}{Histograms \& Probability Distribution Functions}
	The Histogram of real data often looks more and more like a probability distribution function as the number of samples gets large:
	\begin{center}
		\includegraphics[width=0.9\linewidth]{diceRollHist2}
	\end{center}
\end{frame}

\begin{frame}{Histograms \& Probability Distribution Functions}
	The Histogram of real data often looks more and more like a probability distribution function as the number of samples gets large:
	\begin{center}
		\includegraphics[width=0.9\linewidth]{diceRollHist3}
	\end{center}
\end{frame}

\begin{frame}{Histograms \& Probability Distribution Functions}
	The Histogram of real data often looks more and more like a probability distribution function as the number of samples gets large:
	\begin{center}
		\includegraphics[width=0.9\linewidth]{diceRollHist4}
	\end{center}
\end{frame}

\section{Discrete Random Variables}
\begin{frame}{Outline}
	\tableofcontents[currentsection,subsectionstyle=show/shaded/hide]
\end{frame}

\begin{frame}{Discrete Random Variable Facts}
	A few things that are true about discrete random variables:
	\begin{itemize}
		\item $0 < P(y) < 1$ for all $y$
		\item[]
		\item The sum of all the $P(y)$ is 1. That is $\sum_{i} P(y_i) = 1$
		\item[]
		\item The probabilities are additive. For example if we are rolling two dice and $y$ is the sum, $P(2 \text{ or } 7) = P(2) + P(7) = 0.020 + 0.122 = 0.142$
	\end{itemize}
\end{frame}

\subsection{Binomial Random Variables}
\begin{frame}{Outline}
	\tableofcontents[currentsection,subsectionstyle=show/shaded/hide]
\end{frame}

\begin{frame}{Binomial Random Variables}
	\begin{itemize}
		\item On the first day of class I mentioned that there are common types of data encountered in all scientific disciplines--here some are!
		\item[]
		\item Many experiments and/or studies generate ``binary'' data where there are two possible responses:
		\begin{itemize}
			\item Political polls: ``Do you approve of this proposed law?''
			\item Classic toxicology experiments: An experiment where a litter of rats are injected with a potential carcinogen. Random variable of whether or not a tumor was found after a specific time \{yes, no\}
		\end{itemize}
	\end{itemize}
\end{frame}

\begin{frame}{Binomial experiments}
	\begin{itemize}
		\item \textbf{\emph{Binomial experiment:}} An experiment is a binomial experiment if:
		\begin{enumerate}
			\item The experiment consists of $n$ identical trials
			\item Each trial results in one of two outcomes (``success'' and ``failure'')
			\item The probability of sucess on a single trial is equal to $\pi$, and $\pi$ remains the same from trial to trial
			\item The trials are independent--the outcome of one does not influence the outcome of another
			\item The random variable $y$ is the number of successes during the $n$ trials
		\end{enumerate}
	\item[]
	\item Example: we test a lot of pregnancy tests (1000 tests) to estimate the proportion that give negative readings (a failure) when exposed to synthetic hormones
	\begin{enumerate}
		\item The experiment does consist of 1,000 identical trials
		\item There are only two outcomes--``success'' and ``failure''
		\item The probability of success is constant
		\item The trials are independent 
		\item The random variable $y$ is the number of successes during the $n$ trials
	\end{enumerate}
	\end{itemize}
\end{frame}

\begin{frame}{Binomial Random Variables}
	\begin{itemize}
		\item We can compute $P(y)$ for a binomial random variable if we know $\pi$
		\item Given a fixed $n$, the probability distribution function for a binomial random variable is: 
		\begin{gather*}
		P(y) = \frac{n!}{y!(n-y)!} \pi^y (1-\pi)^{n-y}
		\end{gather*}
		where:
		\begin{align*}
		n &= \text{number of trials} \\
		\pi &= \text{probability of success on a single trial} \\ 
		1- \pi &= \text{probability of failure on a single trial} \\
		y &= \text{number of successes in $n$ trials} \\
		n! &= n \times (n-1) \times (n-2) \times 3 \times 2 \times 1
		\end{align*}
	\end{itemize}
\end{frame}

\begin{frame}{Binomial Random Variables}
	\begin{itemize}
		\item Ticks such as \emph{Dermacentor andersoni} (the Rocky Mountain wood tick) may carry the bacteria \emph{rickettsia} which causes Rocky Mountain spotted fever. We believe that the probability a tick is a host to the bacteria is 18\%. 
		\item Example question \#1: What is the probability that 2 out of 10 field-captured ticks carry the bacteria?
		\begin{itemize}
			\item $n = 10$
			\item $\pi = 0.18$
			\item $y = 2$
		\end{itemize}
		\begin{align*}
			P(y) &= \frac{n!}{y!(n-y)!} \pi^y (1-\pi)^{n-y} \\
			P(2) &= \frac{10!}{2! \times 8!} \times 0.18^{2} \times 0.82^{8} \\
			&= 45 \times 0.0324 \times 0.2044 \\
			&= 0.2980
 		\end{align*}
	\end{itemize}
\end{frame}

\begin{frame}{Binomial Random Variables}
	\begin{itemize}
		\item Ticks such as \emph{Dermacentor andersoni} (the Rocky Mountain wood tick) may carry the bacteria \emph{rickettsia} which causes Rocky Mountain spotted fever. We believe that the probability a tick is a host to the bacteria is 18\%. 
		\item Example question \#2: What is the probability that two or less (0, 1, 2) of 10 field-captured ticks carry the bacteria?
		\begin{itemize}
			\item $n = 10$
			\item $\pi = 0.18$
		\end{itemize}
		\begin{align*}
		P(0) &= \frac{10!}{0! \times 10!} \times 0.18^{0} \times 0.82^{10} = 0.1375 \\
		P(1) &= \frac{10!}{2! \times 8!} \times 0.18^{2} \times 0.82^{8} = 0.3017 \\
		P(2) &= \frac{10!}{2! \times 8!} \times 0.18^{2} \times 0.82^{8} = 0.2980\\
		P(0) + P(1) + P(2) &= 0.7372
		\end{align*}
	\end{itemize}
\end{frame}

\begin{frame}{Binomial Random Variables}
	\begin{itemize}
		\item Ticks such as \emph{Dermacentor andersoni} (the Rocky Mountain wood tick) may carry the bacteria \emph{rickettsia} which causes Rocky Mountain spotted fever. We believe that the probability a tick is a host to the bacteria is 18\%. 
		\item Example question \#2: What is the probability that two or less (0, 1, 2) of 10 field-captured ticks carry the bacteria?
		\begin{itemize}
			\item $n = 10$
			\item $\pi = 0.18$
		\end{itemize}
		\begin{align*}
		P(0) &= \frac{10!}{0! \times 10!} \times 0.18^{0} \times 0.82^{10} = 0.1375 \\
		P(1) &= \frac{10!}{2! \times 8!} \times 0.18^{2} \times 0.82^{8} = 0.3017 \\
		P(2) &= \frac{10!}{2! \times 8!} \times 0.18^{2} \times 0.82^{8} = 0.2980\\
		P(y \leq 2) &= P(0) + P(1) + P(2) = 0.7372
		\end{align*}
	\end{itemize}
\end{frame}

\begin{frame}{Binomial Random Variables}
	\begin{itemize}
		\item Ticks such as \emph{Dermacentor andersoni} (the Rocky Mountain wood tick) may carry the bacteria \emph{rickettsia} which causes Rocky Mountain spotted fever. We believe that the probability a tick is a host to the bacteria is 18\%. 
		\item[]
		\item Example question \#3: What is the probability that more than two (3, 4, 5, 6, 7, 8, 9, 10) of 10 field-captured ticks carry the bacteria?
		\begin{itemize}
			\item Recall that if $A$ is an event, then $P(A^c) = 1- P(A)$
		\end{itemize}
		\begin{align*}
			P(y > 2) &= 1 - P(y\leq 2) \\
			&= 1 - 0.7372 \\
			&= 0.2628
		\end{align*}
	\end{itemize}
\end{frame}

\begin{frame}{Binomial Random Variables}
	Binomial random variables (generated by binomial experiments) do produce histograms
	\begin{center}
		\includegraphics[width=.8\linewidth]{TickBinom}
	\end{center}
\end{frame}

\begin{frame}{Binomial Random Variables}{Mean and Standard Deviation}
	\begin{center}
		\includegraphics[width=.65\linewidth]{TickBinom}
	\end{center}
	\begin{itemize}
		\item We have formulas for both the mean and standard deviation:
		\begin{itemize}
			\item $\mu = n \pi$
			\item $\sigma = \sqrt{n \pi (1-\pi)}$
		\end{itemize}
	\end{itemize}
\end{frame}

\begin{frame}{Binomial Random Variables}{Repeated experiments}
	\begin{center}
		\includegraphics[width=.65\linewidth]{TickBinom2}
	\end{center}
	\begin{itemize}
		\item So if we repeat the experiment many times, we have:
		\begin{itemize}
			\item $\mu = 100 (0.18) = 18$
			\item $\sigma = \sqrt{100 (0.18) (1-0.18)} = 3.842$
		\end{itemize}
	\end{itemize}
\end{frame}

\begin{frame}{Binomial Random Variables}{Repeated experiments}
	\begin{columns}
		\begin{column}{0.5 \textwidth}
				\begin{center}
				\includegraphics[width=1\linewidth]{TickBinom2}
			\end{center}
		\begin{itemize}
			\item $\mu = 100 (0.18) = 18$
			\item $\sigma = \sqrt{100 (0.18) (1-0.18)} = 3.842$
		\end{itemize}
		\end{column}
		\begin{column}{0.5 \textwidth}
			\begin{itemize}
				\item This gives us an idea of whether an observed set of data is ``rare'' or ``typical'' assuming that our belief that 18\% of the ticks carry the bacteria is true
				\item[]
				\item If we go to a new state and in a field sample of 100 ticks 30 carry the bacteria, do we still believe that 18\% of the population carry the bacteria?
			\end{itemize}
		\end{column}
	\end{columns}
\end{frame}

\subsection{Poisson Random Variables}
\begin{frame}{Outline}
	\tableofcontents[currentsection,subsectionstyle=show/shaded/hide]
\end{frame}

\begin{frame}{Poisson Random Variables}
	\begin{itemize}
		\item Another common type of random variable observed in the sciences is Poisson random variables
		\item[]
		\item Describe processes in which events occur ``randomly'' in time and we are interested in the number of events over a time interval
		\item[]
		\item Examples: 
		\begin{itemize}
			\item Number of photons (or some type of particle) hitting a detector over a minute
			\item Number of new DNA variants arising in a population over a day
			\item Number of cars arriving at a toll booth between 10 and 11 PM
			\item Number of patients presenting to an Emergency room between 10 and 11 PM
			\item Number of calls arriving at a call center during an hour
		\end{itemize}
	\end{itemize}
\end{frame}

\begin{frame}{Poisson Random Variables}
	\textbf{\emph{Poisson Random Variables:}} Let $y$ be the number of events occurring during a fixed interval of time. $y$ is a Poisson random variable provided the following are true:
		\begin{enumerate}
			\item Events occur one at a time; cannot have two or more events at the same time*
			\item The occurrence of one event in a given period of time is independent of the timing of other events
			\item The expected number of events during one time interval is the same as the expected number of events in another time interval (of equal length)
		\end{enumerate}
		{\tiny
			*although two events could take place at very close times (e.g. 13.3456839358 seconds is not the same as 13.3456839359)}
\end{frame}

\begin{frame}{Poisson Random Variables}
	\begin{itemize}
		\item Let $y$ be the number of events occurring during a fixed interval of time, the probability distribution function for a Poisson random variable is:
		\begin{gather*}
			P(y) = \frac{\mu ^ y e^{-mu}}{y!}
		\end{gather*}
		where $\mu$ is the event rate per unit time
		\item[]
		\item Example: Assume that on average a call center receives 10 calls per hour. What is the probability that the call center receives exactly 5 calls in one hour?
		\begin{itemize}
			\item $\mu = 10$ (10 calls per hour)
			\item $y = 5$
		\end{itemize}
		\begin{align*}
			P(5) = \frac{10^5 e^{-10}}{5!} = 0.038
		\end{align*}
	\end{itemize}
\end{frame}

\begin{frame}{Poisson Random Variables}
	\begin{itemize}
		\item Example 2: Assume that on average a call center receives 10 calls per hour. What is the probability that the call center receives four or less calls in one hour?
		\begin{itemize}
			\item $\mu = 10$ (10 calls per hour)
			\item $y = 5$
		\end{itemize}
	{\scriptsize
		\begin{align*}
		P(y \leq 4) &= P(0) + P(1) + P(2) + P(3) + P(4) \\
		&= \frac{10^0 e^{-10}}{0!} + \frac{10^1 e^{-10}}{1!} + \frac{10^2 e^{-10}}{2!} + \frac{10^3 e^{-10}}{3!} + \frac{10^4 e^{-10}}{4!} \\
		&=  0.00004539993 + 0.0004539993 + 0.002269996 + 0.007566655 + 0.01891664 \\
		&= 0.02925269
		\end{align*}}
	\item Example 3: What is the probability that the call center receives more than four calls in one hour?
	\begin{align*}
		P(y > 4) &= 1 - P(y \leq 4) \\
		&= 1- 0.02925269 \\
		&=  0.9707473
	\end{align*}
	\end{itemize}
\end{frame}

\begin{frame}{Poisson Random Variables}
	\begin{itemize}
		\item While typically a Poisson RV describes events over time intervals, it can be over other types of intervals (time, volume, \& space)
		\item[]
		\item Space example: Assume you stain a tissue sample for a specific protein. Assume on average you count 10 cells in a field that express the protein. The number of cells that express the protein is a Poisson random variable 
		\item[]
		\item Volume example: A count of the number of particles in a volume of water
	\end{itemize}
\end{frame}

\section{Continuous Random Variables}
\begin{frame}{Outline}
	\tableofcontents[currentsection,subsectionstyle=show/shaded/hide]
\end{frame}

\begin{frame}{Continuous Random Variables}
	\begin{itemize}
	\item A continuous RV is one that can take any of infinitely many values on a line interval
	\item Example: if we had a tape measure with unlimited precision, every person in the class would have a unique height. If there were an infinite number of humans measured, there would be an infinite number of heights
	
	\item Complication: when we have a probability distribution function for a continous random variable, $P(y)$, there is no single $y$ value for which $P(y) > 0$,
	however intervals such as $P(a<y<b) > 0$. 
	\begin{itemize}
		\item The probability that a student in this class has height $y = 67.35719549248350910351...4427824$ on our unlimited precision tape measure is 0 (we won't find a single student with that exact height). 
		\item However, it is likely that we will find a student with a height between 67.0 and 67.5 inches. That is $P(67.0 < y < 67.5) > 0$. 
	\end{itemize}
\end{itemize}
\end{frame}

\begin{frame}{Continuous Random Variables}
	\begin{itemize}
		\item Now our probability statements will involve ranges: What is the probability a selected person is between 50 and 60 inches? $P(50 < y < 60)$
	\end{itemize}
\begin{center}
	\includegraphics[width = .7 \linewidth]{NormalHeights}
\end{center}
\end{frame}

\begin{frame}{Continuous Random Variables}
	\begin{itemize}
		\item Probabilities such as $P(50 < y < 60)$, are found by determining the area under the curve
		\item The total area under the curve is 1
	\end{itemize}
	\begin{center}
		\includegraphics[width = .7 \linewidth]{NormalHeights}
	\end{center}
\end{frame}

\begin{frame}{Continuous Random Variables}
	\begin{itemize}
		\item \textbf{Warning!} Unlike discrete random variables, for continuous RV's $f(y)$ is not the probability at a specific point. That is $f(y) \neq P(y)$ 
	\end{itemize}
	\begin{center}
		\includegraphics[width = .7 \linewidth]{NormalHeights}
	\end{center}
\end{frame}

\subsection{Normal Random Variables}
\begin{frame}{Outline}
	\tableofcontents[currentsection,subsectionstyle=show/shaded/hide]
\end{frame}

\begin{frame}{Normal Random Variables}
	\begin{itemize}
		\item Many variables we are interested in the sciences have a ``mound-shaped'' or ``bell-shaped'' distribution. These data are said to be ``normally'' distributed
		\item[]
		\item Sometimes variables do not have a perfect normal distribution, but are well approximated by a normal distribution curve
		\item[]
		\item The ``empirical rule'' that we learned earlier comes from approximating the distribution of data with a normal distribution
	\end{itemize}
\end{frame}


\begin{frame}{Normal Random Variables}{The Standard Normal Distribution}
	\begin{itemize}
		\item The standard normal distribution has $\mu = 0$ and $\sigma = 1$
		\item The ``empirical rule'' again:
	\end{itemize}
\begin{center}
	\includegraphics[width=.8\linewidth]{StandardNorm1}
\end{center}
\end{frame}

\begin{frame}{Normal Random Variables}{The Standard Normal Distribution}
	\begin{itemize}
		\item The standard normal distribution has $\mu = 0$ and $\sigma = 1$
		\item The ``empirical rule'' again:
	\end{itemize}
	\begin{center}
		\includegraphics[width=.8\linewidth]{StandardNorm2}
	\end{center}
\end{frame}

\begin{frame}{Normal Random Variables}{The Standard Normal Distribution}
	\begin{itemize}
		\item The standard normal distribution has $\mu = 0$ and $\sigma = 1$
		\item The ``empirical rule'' again:
	\end{itemize}
	\begin{center}
		\includegraphics[width=.8\linewidth]{StandardNorm3}
	\end{center}
\end{frame}

\begin{frame}{Normal Random Variables}
	\begin{itemize}
		\item We often need to compute the area under a normal curve
		\item[]
		\item For this we use tables such as the one in the textbook apendix
		\item[]
		\item Z scores:
		\begin{itemize}
			\item The first step in most probability calculations with normal distributions is to calculate \textbf{\emph{z-scores}}. A $z$-score is defined as:
			\begin{gather*}
			z = \frac{y-\mu}{\sigma}
			\end{gather*}
			\item A z-score relates our random variable of interest (example: the heights of students in this class) with the ``standard normal distribution''
		\end{itemize}
	\end{itemize}
\end{frame}

\begin{frame}{Normal Random Variables}
	\begin{center}
		\includegraphics[scale = .15]{Lahontan_cutthroat_trout_image_USFWS}
	\end{center}
	\begin{itemize}
		\item Assume the weights of Lahontan Cutthroat Trout are normally distributed with $\mu = 4.03$ pounds and $\sigma = 1.65$ pounds
		\item Question 1: What is the z-score for a trout weighting $5$ pounds?
		\begin{align*}
			z &= \frac{y-\mu}{\sigma} \\
			&= \frac{5.00-4.03}{1.65} \\
			&= 0.588
		\end{align*}
	\end{itemize}
\end{frame}

\begin{frame}{Normal Random Variables}{Using a Standard Normal Table}
	\begin{columns}
		\begin{column}{.6 \textwidth}
				\begin{center}
				\includegraphics[width=1 \linewidth]{StandardNorm4}
			\end{center}
		\end{column}
	\begin{column}{.4 \textwidth}
			\begin{itemize}
			\item Using a standard normal table is a bit tedious
			\item[]
			\item All area values in a standard normal table are to the left of the $z$-score
			\item[]
			\item The values are $P(Z\leq z)$
		\end{itemize}
	\end{column}
	\end{columns}
\end{frame}

\begin{frame}{Normal Random Variables}{Using a Standard Normal Table}
	\begin{columns}
		\begin{column}{.6 \textwidth}
				\vspace{-15pt}
				\includegraphics[width=1 \linewidth]{normalTable}
		\end{column}
		\begin{column}{.4 \textwidth}
			\begin{itemize}
				\item Using a standard normal table is a bit tedious
				\item[]
				\item All area values in a standard normal table are to the left of the $z$-score
				\item[]
				\item The values are $P(Z\leq z)$
			\end{itemize}
		\end{column}
	\end{columns}
\end{frame}


\begin{frame}{Normal Random Variables}
	\begin{center}
		\includegraphics[scale = .15]{Lahontan_cutthroat_trout_image_USFWS}
	\end{center}
	\begin{itemize}
		\item Assume the weights of Lahontan Cutthroat Trout are normally distributed with $\mu = 4.03$ pounds and $\sigma = 1.65$ pounds
		\item Question 2: What is the probability of catching a fish less than $5$ pounds? Remember $z = 0.588$.
		\begin{align*}
		P(Z\leq z) &= P(Z \leq 0.588) = 0.722
		\end{align*}
	\end{itemize}
\end{frame}

\begin{frame}{Normal Random Variables}{The Standard Normal Distribution}
	\begin{center}
		\includegraphics[width=.5\linewidth]{StandardNorm5}
	\end{center}
\end{frame}


\begin{frame}{Normal Random Variables}
	\begin{center}
		\includegraphics[scale = .15]{Lahontan_cutthroat_trout_image_USFWS}
	\end{center}
	\begin{itemize}
		\item Assume the weights of Lahontan Cutthroat Trout are normally distributed with $\mu = 4.03$ pounds and $\sigma = 1.65$ pounds
		\item Question 2: What is the probability of catching a fish that is greater than $5$ pounds? Remember $z = 0.588$
		\begin{align*}
		P(Z > z) &= 1 - P(Z \leq z) = 1 - P(Z \leq 0.588)\\
		&= 1 - 0.722 \\
		&= 0.278
		\end{align*}
	\end{itemize}
\end{frame}

\begin{frame}{Normal Random Variables}{The Standard Normal Distribution}
	\begin{center}
		\includegraphics[width=.8\linewidth]{StandardNorm6}
	\end{center}
\end{frame}

\begin{frame}{Normal Random Variables}
	\begin{center}
		\includegraphics[scale = .15]{Lahontan_cutthroat_trout_image_USFWS}
	\end{center}
	\begin{itemize}
		\item Assume the weights of Lahontan Cutthroat Trout are normally distributed with $\mu = 4.03$ pounds and $\sigma = 1.65$ pounds
		\item Question 3: What is the probability of catching a fish that is between 5 and 6 pounds?
		\begin{gather*}
		z_1 = \frac{5-4.03}{1.65} = 0.588 \quad \quad z_2 = \frac{6-4.03}{1.65} = 1.194\\
		P(Z \leq z_2) - P(Z \leq z_1) \\
		P(Z \leq 1.194) - P(Z \leq 0.588) \\
		0.884 - 0.722 =  0.162
		\end{gather*}
	\end{itemize}
\end{frame}

\begin{frame}{Normal Random Variables}{Percentiles}
	\begin{itemize}
		\item We may be interested to know the value of a percentile of a normal random variable
		\begin{itemize}
			\item The $p$\textsuperscript{th} percentile is the value at which $p$\% of the population values fall below and $(100-p)$\% of the population values are above
			\item For a standard normal distribution to find the 80th percentile, we we use the tables to find $z_{.80}$, that is the $z$-score that gives a probability closes to 0.80
		\end{itemize}
	\end{itemize}
	\begin{center}
		\includegraphics[width = .5\linewidth]{normalTable2}
	\end{center}
\end{frame}

\begin{frame}{Normal Random Variables}{Percentiles}
	\begin{itemize}
		\item If we have values that are from a normal random variable that isn't the standard normal, then we use the following formula to find percentiles: $y_p = \mu + z_p \sigma$
		\item[]
		\item Example: What value is the 80th percentile Lahontan Cutthroat Trout weights? In a sample of 100 random trout we would expect 80\% of the fish to be less than this number and 20\% to be greater
		\begin{itemize}
			\item Recall that for this population $\mu = 4.03$, and $\sigma = 1.65$
			\item First we find that $z_{.80} = 0.84$
			\item So then:
			\begin{align*}
				y_p &= \mu + z_p \sigma \\
				&= 4.03 + 0.84 \times 1.65 \\
				&= 5.416
			\end{align*}
		\end{itemize}
	\end{itemize}
\end{frame}


\end{document}