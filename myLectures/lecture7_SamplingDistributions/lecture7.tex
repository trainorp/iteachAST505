\documentclass[xcolor=dvipsnames]{beamer} 
\usetheme{AnnArbor}
\usecolortheme{beaver}

\usepackage{amsmath,graphicx,booktabs,tikz,subfig,color,lmodern}
\definecolor{mycol}{rgb}{.4,.85,1}
\setbeamercolor{title}{bg=mycol,fg=black} 
\setbeamercolor{palette primary}{use=structure,fg=white,bg=red}
\setbeamercolor{block title}{fg=white,bg=red!50!black}
% \setbeamercolor{block title}{fg=white,bg=blue!75!black}

\title[Lecture 7]{Lecture 7: Sampling distributions \& normal approximations}
\author[Patrick Trainor]{Patrick Trainor, PhD, MS, MA}
\institute[NMSU]{New Mexico State University}
\date{February 11, 2019}

\begin{document}
	
\begin{frame}
	\maketitle
\end{frame}

\begin{frame}{Outline}
	\tableofcontents[hideallsubsections]
\end{frame}

\section{Random Sampling}
\begin{frame}{Outline}
	\tableofcontents[currentsection,subsectionstyle=show/shaded/hide]
\end{frame}

\begin{frame}{Simple Random Sample}
	\begin{itemize}
		\item \textbf{\emph{Simple Random Sample:}} A sample of $n$ measurements selected from a population is a simple random sample if every different sample of size $n$ from the population has equal probability of being selected \pause 
		\item Book example: We make an inference about large cities in the US. Of 10 large cities we will randomly select ten and then conduct a survey \pause 
	\end{itemize}
	\begin{center}
		\includegraphics[width = .55 \linewidth]{srs}
	\end{center}
\end{frame}

\begin{frame}{Simple Random Sample}
	\begin{itemize}
		\item To make a simple random sample, each one of the 45 possible combinations must have an equal probability of being selected, specifically a $1/45$ chance 
	\end{itemize}
	\begin{center}
		\includegraphics[width = .65 \linewidth]{srs}
	\end{center}
\end{frame}

\begin{frame}{Simple Random Sample}
	\begin{itemize}
		\item Often with simple random sampling, the probability of selecting a specific set of measurements is nearly zero (hence random)  \pause 
		\item[]
		\item There are 376,437,600 ways to choose 5 measurements out of 100. So the probability of choosing a specific collection of measurements is 0.00000000265 \pause 
		\item[]
		\item We normally use computer algorithms to generate random numbers for sampling. Example: If I want a sample with $n=5$ with a population size of 100: \pause 
		\begin{enumerate}
			\item Go to https://www.random.org/sequences/ \pause 
			\item Enter 1 as smallest integer, 100 as largest \pause 
			\item Use sequence of first 5 integers
		\end{enumerate}
	\end{itemize}
\end{frame}

\section{Sampling Distributions}
\begin{frame}{Outline}
	\tableofcontents[currentsection,subsectionstyle=show/shaded/hide]
\end{frame}

\begin{frame}{Sampling Distributions}{Motivation}
	\begin{itemize}
		\item Suppose we knew the weight of every fish in the pond by the horseshoe. This group of fish would constitute a \textbf{population} \pause 
		\item The weights of the fish are: \\
		498.64 499.59 510.11 498.42 478.43 504.99 492.45 507.79 507.55 489.00 501.67 499.71 518.76 502.45 507.02 499.85
		498.57 503.21 501.22 494.05 495.58 502.91 507.24 504.60 501.85 502.34 505.93 520.01 481.63 491.38 515.83 501.55
		497.25 507.88 497.77 513.92 495.11 501.37 500.04 492.73 492.79 498.09 513.35 503.56 508.43 507.75 500.80 493.27
		518.36 497.93 494.60 484.99 502.68 500.34 490.05 492.30 494.22 490.67 478.23 494.86 500.89 502.94 506.92 483.88
		501.88 500.77 498.26 486.01 496.31 504.51 516.26 519.23 498.37 495.61 511.91 516.70 511.33 484.83 527.31 504.99
		485.08 509.22 494.83 521.05 491.13 498.32 509.09 486.05 508.14 490.43 496.30 514.79 493.96 497.09 482.16 522.58
		493.33 504.78 488.39 496.20
	\end{itemize}
\end{frame}

\begin{frame}{Sampling Distributions}{Motivation}
	\begin{itemize}
		\item The population mean of the fish data is $\mu = 500.595$ and the population standard deviation is $\sigma = 10.02815$ \pause 
		\item[]
		\item Here are 5 simple random samples (with sample mean at the bottom): \pause 
	\end{itemize}
	\begin{center}
		\includegraphics[width = .8 \linewidth]{randomFish}
	\end{center}
\end{frame}

\begin{frame}{Sampling Distributions}{Motivation}
	\begin{itemize}
		\item The population mean of the fish data is $\mu = 500.595$ and the population standard deviation is $\sigma = 10.02815$ \pause 
		\item And now a histogram of the sample means from 1,000 simple random samples:
	\end{itemize}
	\begin{center}
		\includegraphics[width = .65 \linewidth]{sampleMeanHist}
	\end{center}
\end{frame}

\begin{frame}{Sampling Distributions}
	\begin{columns}
		\begin{column}{.7 \textwidth}
			\begin{center}
				\includegraphics[width = 1 \linewidth]{sampleMeanHist2}
			\end{center}
		\end{column}
		\begin{column}{.3 \textwidth}
			\begin{itemize}
				\item The spread of a sampling distribution depends on the sample size
			\end{itemize}
		\end{column}
	\end{columns}
\end{frame}

\begin{frame}{Sampling Distributions}
	\begin{columns}
		\begin{column}{.5 \textwidth}
			\begin{center}
				\includegraphics[width = 1 \linewidth]{sampleMeanHist2}
			\end{center}
		\end{column}
		\begin{column}{.5 \textwidth}
			\begin{tabular}{|p{1.25cm}|p{1.7cm}|p{1.75cm}|} \hline
				\textbf{Sample Size}   &  \textbf{Mean (of $\bar{y}$)}  &  \textbf{Standard Deviation (of $\bar{y}$)} \\ \hline \hline
				5 &500.5901& 4.417413 \\ \hline
				10 &500.6141& 3.062839 \\ \hline
				50& 500.6128 &1.010857 \\ \hline
			\end{tabular}
		\end{column}
	\end{columns}
\end{frame}

\begin{frame}{Sampling Distributions}
	\begin{columns}
		\begin{column}{.5 \textwidth}
			\begin{center}
				\includegraphics[width = 1 \linewidth]{sampleMeanHist2}
			\end{center}
		\end{column}
		\begin{column}{.5 \textwidth}
			The sampling distribution is related to the population distribution by: \pause 
			\begin{itemize}
				\item $\mu_{\bar{y}} = \mu$ \pause 
				\item[]
				\item $\sigma_{\bar{y}} =\frac{\sigma}{\sqrt{n}} $\pause 
				\item[]
				\item $\sigma_{\bar{y}}$ is called the standard error of $\bar{y}$
			\end{itemize}
		\end{column}
	\end{columns}
\end{frame}

\section{The Central Limit Theorem}
\begin{frame}{Outline}
	\tableofcontents[currentsection,subsectionstyle=show/shaded/hide]
\end{frame}

\begin{frame}{The Central Limit Theorem}
	\textbf{\emph{The Central Limit Theorem:}} Let $\bar{y}$ denote the sample mean computed from a random sample of $n$ measurements from a population having mean $\mu$ and finite standard deviation $\sigma$. Let $\mu_{\bar{y}}$ and $\sigma_{\bar{y}}$ denote the mean and standard deviation of the sampling distribution of $\bar{y}$, respectively. Then: \pause 
	\begin{enumerate}
		\item $\mu_{\bar{y}} = \mu$ \pause 
		\item $\sigma_{\bar{y}} = \sigma / \sqrt{n}$ \pause 
		\item If $n$ is sufficiently large, the distribution of $\bar{y}$, will be approximately normal \pause 
		\item If the population distribution is normal, then $\bar{y}$ is normal for any sample size $n$ (as opposed to approximately normal if $n$ is large)
	\end{enumerate}
\end{frame}

\begin{frame}{The Central Limit Theorem}{Why is it important}
	One of the most significant results of the Central Limit Theorem is that even if the distribution of measurement in a population has a very ``non-normal'' distribution, the sampling distributions of many of the statistics we are interested in (mean, median, standard deviation), are normal distributions* \\
	
	\vspace*{20 pt}
	*If this was not the case your textbook would be 20,000 pages long instead of 1,000 pages
\end{frame}

\begin{frame}{The Central Limit Theorem}{Example}
	\begin{center}
		\includegraphics[width = .9\linewidth]{smHist1}
	\end{center}
\end{frame}

\begin{frame}{The Central Limit Theorem}{Example}
	\begin{center}
		\includegraphics[width = .9\linewidth]{smHist2}
	\end{center}
\end{frame}

\begin{frame}{The Central Limit Theorem}{Example}
	\begin{center}
		\includegraphics[width = .9\linewidth]{smHist3}
	\end{center}
\end{frame}

\begin{frame}{The Central Limit Theorem}{Example}
	\begin{center}
		\includegraphics[width = .9\linewidth]{smHist4}
	\end{center}
\end{frame}

\begin{frame}{The Central Limit Theorem}{Example}
	\begin{center}
		\includegraphics[width = .9\linewidth]{smHist5}
	\end{center}
\end{frame}

\begin{frame}{The Central Limit Theorem}{Example}
	\begin{center}
		\includegraphics[width = .9\linewidth]{smHist6}
	\end{center}
\end{frame}

\begin{frame}{The End of Lecture 7}
	\begin{center}
		\includegraphics[width=.8\linewidth]{DSC_0050_v1}
	\end{center}
\end{frame}

\end{document}