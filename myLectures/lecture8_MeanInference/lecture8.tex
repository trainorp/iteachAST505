\documentclass[xcolor=dvipsnames]{beamer} 
\usetheme{AnnArbor}
\usecolortheme{beaver}

\usepackage{amsmath,graphicx,booktabs,tikz,subfig,color,lmodern}
\definecolor{mycol}{rgb}{.4,.85,1}
\setbeamercolor{title}{bg=mycol,fg=black} 
\setbeamercolor{palette primary}{use=structure,fg=white,bg=red}
\setbeamercolor{block title}{fg=white,bg=red!50!black}
% \setbeamercolor{block title}{fg=white,bg=blue!75!black}

\title[Lecture 8]{Lecture 8: Inferences about the mean}
\author[Patrick Trainor]{Patrick Trainor, PhD, MS, MA}
\institute[NMSU]{New Mexico State University}
\date{September 16, 2019}

\begin{document}

\begin{frame}
	\maketitle
\end{frame}

\begin{frame}{Outline}
	\tableofcontents[hideallsubsections]
\end{frame}

\section{Introduction}
\begin{frame}{Outline}
	\tableofcontents[currentsection,subsectionstyle=show/shaded/hide]
\end{frame}

\begin{frame}{Introduction}
	\begin{columns}
		\begin{column}{.4 \textwidth}
			\begin{center}
				\includegraphics[width=1.2 \linewidth]{./../lecture1_Introduction/Sampling.png}
				\end{center}
		\end{column}
		\begin{column}{.5 \textwidth}
				\begin{itemize}
				\item All the way back in Lecture \#1 we discussed how we use samples to make inferences about larger populations
				\item[]
				\item In general, the two types of inference we want to make are:
				\begin{itemize}
					\item an estimate of the value of a population parameter 
					\item[]
					\item a test of a hypothesis about the value of a parameter(s)
				\end{itemize}
			\end{itemize}
		\end{column}
	\end{columns}
\end{frame}

\begin{frame}{Introduction}{Inference examples}
	\begin{columns}
		\begin{column}{.5 \textwidth}
			Population parameter estimations:
			\begin{itemize}
				\item What is the prevalence of Type 2 Diabetes in the United States (or in New Mexico, or in Las Cruces)?
				\item[]
				\item What percent of middle school-aged youth are bullied online?
				\item[]
				\item What percentage of \emph{Ixodes scapularis} (deer ticks) carry the bacterium \emph{Borrelia burgdorferi} in Kentucky?
			\end{itemize}
		\end{column}
	
		\begin{column}{.5 \textwidth}
			Test of hypotheses about the value of parameters:
			\begin{itemize}
				\item Is the prevalence of Type 2 Diabetes in the United States increasing? 
				\item[]
				\item Are more middle school-aged youth bullied online or at school?
				\item[]
				\item Does treatment of clothing with permethrin reduce incidence of lyme disease?
			\end{itemize}
		\end{column}
	\end{columns}
\end{frame}

\section{Estimation of Normal population mean}
\begin{frame}{Outline}
	\tableofcontents[currentsection,subsectionstyle=show/shaded/hide]
\end{frame}

\begin{frame}{The sample mean}
	\begin{itemize}
		\item There is an entire field of mathematical statistics devoted to determining the best estimator for different types of population parameters
		\item[]
		\item For Normal distributions, the sample mean is the ``best'' estimate (or estimator) of the population mean 
		\item[]
		\item That the sample mean is a good guess of the value of the population mean is intuitive, but an important question remains...\emph{How confident should we be in the reliability of the sample mean as an estimate of the population mean?}
	\end{itemize}
\end{frame}

\begin{frame}{The sample mean}
	\begin{itemize}
		\item Let's play a guessing game!
		\item[] 
		\item Sally says that the average high temperature in Louisville, Kentucky in September is between 84 and 84.5. Paul says that the average is between 0 and 150 degrees.
	\end{itemize}
\begin{center}
	\includegraphics[width=.7\linewidth]{Louisville}
\end{center}
\end{frame}

\begin{frame}{The sample mean}
	\begin{itemize}
		\item Sally says that the average high temperature in Louisville, Kentucky in September is between 84 and 84.5. Paul says that the average is between 0 and 150 degrees.
		\item[]
		\item Who is more ``right'' if the true (population) value is 82 degrees?
	\end{itemize}
	\begin{center}
		\includegraphics[width=.7\linewidth]{Louisville}
	\end{center}
\end{frame}

\begin{frame}{The sample mean}
	\begin{itemize}
		\item Sally says that the average high temperature in Louisville, Kentucky in September is between 84 and 84.5. Paul says that the average is between 0 and 200 degrees.
		\item[]
		\item Who is more ``right'' if the true (population) value is 82 degrees?
		\item[]
		\item Paul wins!
		\item[]
		\item But this highlights what we desire when we are using a sample mean to estimate a population mean
	\end{itemize}
\end{frame}

\begin{frame}{The sample mean}{Confidence intervals}
	\begin{center}
		\includegraphics[width=.8 \linewidth]{point1}
	\end{center}
\begin{itemize}
	\item For continuous random variables the probability that the sample mean is equal to the population mean is zero for a finite sample size $n$
	\item[]
	\item Unfortunately the picture above would never happen
\end{itemize}
\end{frame}

\begin{frame}{The sample mean}{Confidence intervals}
	\begin{center}
		\includegraphics[width=.8 \linewidth]{point2}
	\end{center}
	\begin{itemize}
		\item For continuous random variables the probability that the sample mean is equal to the population mean is zero for a finite sample size $n$
		\item[]
		\item Instead we have something like the above picture
	\end{itemize}
\end{frame}

\begin{frame}{The sample mean}{Confidence intervals}
	\begin{center}
		\includegraphics[width=.8 \linewidth]{point3}
	\end{center}
	\begin{itemize}
		\item With the above picture we want to propose an interval like the blue line between blue dots that is likely to contain the population mean
		\item The most straightforward interval would be to make an interval that is centered on the sample mean but extends a little to the left and a little to the right
		\item But how far to the left and right?!?!
	\end{itemize}
\end{frame}

\begin{frame}{The sample mean}{Confidence intervals}
	\begin{center}
		\includegraphics[width=.8 \linewidth]{point3}
	\end{center}
	\begin{itemize}
		\item But how far to the left and right?!?!
		\item We want an interval that is wide enough to contain the population mean \emph{most of the time}*, but not so wide that it has little information** \\
		\vspace{2.5pt}
		{\tiny *Sally's interval was bad because it doesn't contain the population mean
			**Paul's interval, while trivially containing the population mean, didn't really inform us where the population mean was likely to be}
	\end{itemize}
\end{frame}

\end{document}