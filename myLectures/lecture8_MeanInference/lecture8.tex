\documentclass[xcolor=dvipsnames]{beamer} 
\usetheme{AnnArbor}
\usecolortheme{beaver}

\usepackage{amsmath,graphicx,booktabs,tikz,subfig,color,lmodern}
\definecolor{mycol}{rgb}{.4,.85,1}
\setbeamercolor{title}{bg=mycol,fg=black} 
\setbeamercolor{palette primary}{use=structure,fg=white,bg=red}
\setbeamercolor{block title}{fg=white,bg=red!50!black}
% \setbeamercolor{block title}{fg=white,bg=blue!75!black}

\title[Lecture 8]{Lecture 8: Inferences about the mean}
\author[Patrick Trainor]{Patrick Trainor, PhD, MS, MA}
\institute[NMSU]{New Mexico State University}
\date{September 16, 2019}

\begin{document}

\begin{frame}
	\maketitle
\end{frame}

\begin{frame}{Outline}
	\tableofcontents[hideallsubsections]
\end{frame}

\section{Introduction}
\begin{frame}{Outline}
	\tableofcontents[currentsection,subsectionstyle=show/shaded/hide]
\end{frame}

\begin{frame}{Introduction}
	\begin{columns}
		\begin{column}{.4 \textwidth}
			\begin{center}
				\includegraphics[width=1.2 \linewidth]{./../lecture1_Introduction/Sampling.png}
				\end{center}
		\end{column}
		\begin{column}{.5 \textwidth}
				\begin{itemize}
				\item All the way back in Lecture \#1 we discussed how we use samples to make inferences about larger populations
				\item[]
				\item In general, the two types of inference we want to make are:
				\begin{itemize}
					\item an estimate of the value of a population parameter 
					\item[]
					\item a test of a hypothesis about the value of a parameter(s)
				\end{itemize}
			\end{itemize}
		\end{column}
	\end{columns}
\end{frame}

\begin{frame}{Introduction}{Inference examples}
	\begin{columns}
		\begin{column}{.5 \textwidth}
			Population parameter estimations:
			\begin{itemize}
				\item What is the prevalence of Type 2 Diabetes in the United States (or in New Mexico, or in Las Cruces)?
				\item[]
				\item What percent of middle school-aged youth are bullied online?
				\item[]
				\item What percentage of \emph{Ixodes scapularis} (deer ticks) carry the bacterium \emph{Borrelia burgdorferi} in Kentucky?
			\end{itemize}
		\end{column}
	
		\begin{column}{.5 \textwidth}
			Test of hypotheses about the value of parameters:
			\begin{itemize}
				\item Is the prevalence of Type 2 Diabetes in the United States increasing? 
				\item[]
				\item Are more middle school-aged youth bullied online or at school?
				\item[]
				\item Does treatment of clothing with permethrin reduce incidence of lyme disease?
			\end{itemize}
		\end{column}
	\end{columns}
\end{frame}

\section{Estimation of Normal population mean}
\begin{frame}{Outline}
	\tableofcontents[currentsection,subsectionstyle=show/shaded/hide]
\end{frame}

\begin{frame}{The sample mean}
	\begin{itemize}
		\item There is an entire field of mathematical statistics devoted to determining the best estimator for different types of population parameters
		\item[]
		\item For Normal distributions, the sample mean is the ``best'' estimate (or estimator) of the population mean 
		\item[]
		\item That the sample mean is a good guess of the value of the population mean is intuitive, but an important question remains...\emph{How confident should we be in the reliability of the sample mean as an estimate of the population mean?}
	\end{itemize}
\end{frame}

\begin{frame}{The sample mean}
	\begin{itemize}
		\item Let's play a guessing game!
		\item[] 
		\item Sally says that the average high temperature in Louisville, Kentucky in September is between 84 and 84.5. Paul says that the average is between 0 and 150 degrees.
	\end{itemize}
\begin{center}
	\includegraphics[width=.7\linewidth]{Louisville}
\end{center}
\end{frame}

\begin{frame}{The sample mean}
	\begin{itemize}
		\item Sally says that the average high temperature in Louisville, Kentucky in September is between 84 and 84.5. Paul says that the average is between 0 and 150 degrees.
		\item[]
		\item Who is more ``right'' if the true (population) value is 82 degrees?
	\end{itemize}
	\begin{center}
		\includegraphics[width=.7\linewidth]{Louisville}
	\end{center}
\end{frame}

\begin{frame}{The sample mean}
	\begin{itemize}
		\item Sally says that the average high temperature in Louisville, Kentucky in September is between 84 and 84.5. Paul says that the average is between 0 and 200 degrees.
		\item[]
		\item Who is more ``right'' if the true (population) value is 82 degrees?
		\item[]
		\item Paul wins!
		\item[]
		\item But this highlights what we desire when we are using a sample mean to estimate a population mean
	\end{itemize}
\end{frame}

\begin{frame}{The sample mean}{Confidence intervals}
	\begin{center}
		\includegraphics[width=.8 \linewidth]{point1}
	\end{center}
\begin{itemize}
	\item For continuous random variables the probability that the sample mean is equal to the population mean is zero (for a finite sample size $n$)
	\item[]
	\item Unfortunately the picture above would never happen
\end{itemize}
\end{frame}

\begin{frame}{The sample mean}{Confidence intervals}
	\begin{center}
		\includegraphics[width=.8 \linewidth]{point2}
	\end{center}
	\begin{itemize}
		\item For continuous random variables the probability that the sample mean is equal to the population mean is zero (for a finite sample size $n$)
		\item[]
		\item Instead we have something like the above picture
	\end{itemize}
\end{frame}

\begin{frame}{The sample mean}{Confidence intervals}
	\begin{center}
		\includegraphics[width=.8 \linewidth]{point3}
	\end{center}
	\begin{itemize}
		\item With the above picture we want to propose an interval like the blue line between blue dots that is likely to contain the population mean
		\item The most straightforward interval would be to make an interval that is centered on the sample mean but extends a little to the left and a little to the right
		\item But how far to the left and right?!?!
	\end{itemize}
\end{frame}

\begin{frame}{The sample mean}{Confidence intervals}
	\begin{center}
		\includegraphics[width=.8 \linewidth]{point3}
	\end{center}
	\begin{itemize}
		\item But how far to the left and right?!?!
		\item We want an interval that is wide enough to contain the population mean \emph{most of the time}*, but not so wide that it has little information** \\
		\vspace{2.5pt}
		{\tiny *Sally's interval was bad because it doesn't contain the population mean
			**Paul's interval, while trivially containing the population mean, didn't really inform us where the population mean was likely to be}
	\end{itemize}
\end{frame}

\begin{frame}{The sample mean}{Confidence intervals}
	\begin{itemize}
		\item We want to make \textbf{\emph{interval estimates}} for a population mean $\mu$ such that for a fixed \textbf{\emph{level of confidence}}, called $1-\alpha$,
		we expect that the population mean will be contained in the interval estimate $(1-\alpha) * 100 \%$ of the time
		\item[]
		\item For example, if we fix the level of confidence at .95, we expect that 95\% of the confidence intervals that we construct by sampling from a population will contain the population mean $\mu$. In this case $\alpha = .05$
	\end{itemize}
\end{frame}

\begin{frame}{The sample mean}{Confidence intervals}
	\begin{center}
		\includegraphics[width = .9\linewidth]{coverage1}
	\end{center}
\end{frame}

\begin{frame}{The sample mean}{Confidence intervals}
	Why we say that ``we expect that the population mean will be contained in the interval estimate 95\% of the time''
	\begin{center}
		\includegraphics[width = .85\linewidth]{coverage2}
	\end{center}
\end{frame}

\begin{frame}{The sample mean}{Confidence intervals}
	Why we say that ``we expect that the population mean will be contained in the interval estimate 95\% of the time''
	\begin{center}
		\includegraphics[width = .85\linewidth]{coverage3}
	\end{center}
\end{frame}

\subsection{Confidence intervals for $\mu$ given known $\sigma$}
\begin{frame}{Outline}
	\tableofcontents[currentsection,subsectionstyle=show/shaded/hide]
\end{frame}

\begin{frame}{The sample mean}{Confidence intervals}
	\begin{itemize}
		\item To construct confidence intervals for the population mean, $\mu$, for a fixed confidence level of $\alpha$ if $y$ has a normal distribution with known $\sigma$:
		\begin{enumerate}
			\item Compute the sample mean: $\bar{y}$
			\item[]
			\item Compute the lower confidence interval endpoint: $\bar{y} - z_{1-\alpha / 2} \times \sigma / \sqrt{n}$
			\item[]
			\item Compute the upper confidence interval endpoint: $\bar{y} + z_{1-\alpha / 2} \times \sigma / \sqrt{n}$
		\end{enumerate}
	\item[]
	\item \#2 and \#3 are often written together as $\bar{y} \pm z_{1-\alpha / 2} \times \sigma / \sqrt{n}$
	\end{itemize}
\end{frame}

\begin{frame}{Confidence intervals for $\mu$}
		\begin{itemize}
			\item Suppose we knew the weight of every fish in the pond by the horseshoe. This group of fish would constitute a \textbf{population}
			\item The weights of the fish are: \\
			498.64 499.59 510.11 498.42 478.43 504.99 492.45 507.79 507.55 489.00 501.67 499.71 518.76 502.45 507.02 499.85
			498.57 503.21 501.22 494.05 495.58 502.91 507.24 504.60 501.85 502.34 505.93 520.01 481.63 491.38 515.83 501.55
			497.25 507.88 497.77 513.92 495.11 501.37 500.04 492.73 492.79 498.09 513.35 503.56 508.43 507.75 500.80 493.27
			518.36 497.93 494.60 484.99 502.68 500.34 490.05 492.30 494.22 490.67 478.23 494.86 500.89 502.94 506.92 483.88
			501.88 500.77 498.26 486.01 496.31 504.51 516.26 519.23 498.37 495.61 511.91 516.70 511.33 484.83 527.31 504.99
			485.08 509.22 494.83 521.05 491.13 498.32 509.09 486.05 508.14 490.43 496.30 514.79 493.96 497.09 482.16 522.58
			493.33 504.78 488.39 496.20
		\end{itemize}
\end{frame}

\begin{frame}{Confidence intervals for $\mu$}
	\begin{itemize}
		\item Example problem 1: Suppose you have the following sample from the fish weights data: $\{501.85, 495.58, 502.94, 508.43, 511.33 \}$ and that the population standard deviation is known, $\sigma = 10.028$. Compute a 95\% confidence interval for the population mean.
		\begin{gather*}
			\bar{y} = 504.026 \\
			1 - \alpha = 0.95 \\
			\alpha = 0.05 \\
			1 - \alpha / 2 = 0.975 \\
			z_{1 - \alpha / 2} = z_{0.975} = 1.96 
		\end{gather*}
		
		So the lower endpoint is:
		\begin{align*}
			\bar{y} - z_{1 - \alpha / 2} \times \sigma / \sqrt{n} &= 504.026 - (1.96)(10.028)/\sqrt{5} \\
			&= 504.026 - 8.790 \\
			&= 495.236
		\end{align*}
	\end{itemize}
\end{frame}

\begin{frame}{Confidence interval for $\mu$}
	\begin{itemize}
		\item Example problem 1: Suppose you have the following sample from the fish weights data: $\{501.85, 495.58, 502.94, 508.43, 511.33 \}$ and that the population standard deviation is known, $\sigma = 10.028$. Compute a 95\% confidence interval for the population mean.\\ \vspace{10pt}
		
		And the upper endpoint is:
		\begin{align*}
		\bar{y} + z_{1 - \alpha / 2} \times \sigma / \sqrt{n} &= 504.026 + (1.96)(10.028)/\sqrt{5} \\
		&= 504.026 + 8.790 \\
		&= 512.816
		\end{align*}
		
		So the 95\% confidence interval for the population mean is $(495.236, 512.816)$
	\end{itemize}
\end{frame}

\begin{frame}{Confidence intervals for $\mu$}
	\begin{itemize}
		\item Example problem 2: Suppose you have the following sample from the fish weights data: $\{501.85, 495.58, 502.94, 508.43, 511.33 \}$ and that the population standard deviation is known, $\sigma = 10.028$. Compute a 90\% confidence interval and a 99\% confidence interval for the population mean.\\  \vspace{10pt}
		
		For the 90\% confidence interval, $\alpha = .10$ and 
		\begin{gather*}
			z_{1-\alpha / 2} = z_{0.95} = 1.645
		\end{gather*}
		
		For the 99\% confidence interval, $\alpha = .01$ and 
		\begin{gather*}
		z_{1-\alpha / 2} = z_{0.995} = 2.576
		\end{gather*}
		
		For both the 90\% and 99\% confidence intervals, the other values, such as $\bar{y}$, $\sigma$, and $n$ will remain as before
	\end{itemize}
\end{frame}

\begin{frame}{Confidence intervals for $\mu$}
	\begin{itemize}
		\item Example problem 2: Suppose you have the following sample from the fish weights data: $\{501.85, 495.58, 502.94, 508.43, 511.33 \}$ and that the population standard deviation is known, $\sigma = 10.028$. Compute a 90\% confidence interval and a 99\% confidence interval for the population mean.\\  \vspace{10pt}
		
		For the 90\% confidence interval, $\alpha = .10$ and 
		\begin{gather*}
		z_{1-\alpha / 2} = z_{0.95} = 1.645
		\end{gather*}
		
		For the 99\% confidence interval, $\alpha = .01$ and 
		\begin{gather*}
		z_{1-\alpha / 2} = z_{0.995} = 2.576
		\end{gather*}
		
		For both the 90\% and 99\% confidence intervals, the other values, such as $\bar{y}$, $\sigma$, and $n$ will remain as before
	\end{itemize}
\end{frame}

\begin{frame}{Confidence intervals for $\mu$}
	\begin{itemize}
		\item Example problem 2: Suppose you have the following sample from the fish weights data: $\{501.85, 495.58, 502.94, 508.43, 511.33 \}$ and that the population standard deviation is known, $\sigma = 10.028$. Compute a 90\% confidence interval and a 99\% confidence interval for the population mean.\\  \vspace{8pt}
		
		For the 90\% confidence interval: 
		\begin{align*}
			\bar{y} &\pm z_{1-\alpha / 2} \times \sigma / \sqrt{n} \\
			504.026 &\pm (1.645)(10.028)/\sqrt{5} \\
			(496.649&, 511.403)
		\end{align*}
		
		For the 99\% confidence interval: 
		\begin{align*}
		\bar{y} &\pm z_{1-\alpha / 2} \times \sigma / \sqrt{n} \\
		504.026 &\pm (2.576)(10.028)/\sqrt{5} \\
		(492.474&, 515.578)
		\end{align*}
		
	\end{itemize}
\end{frame}

\begin{frame}{Confidence intervals for $\mu$}{Width of the interval}
	\begin{itemize}
		\item Here are the three confidence intervals we have computed:
		\begin{itemize}
			\item 90\%: $(496.649, 511.403)$
			\item 95\%: $(495.236, 512.816)$
			\item 99\%: $(492.474, 515.578)$
		\end{itemize}
	\item[]
	\item Question \#1: How does the width of the confidence interval change as you increase the confidence level $1-\alpha$?
	\item[]
	\item Question \#2: How would the width of a confidence interval change (for a fixed $\sigma$ and $\alpha$) as $n$ is increased?
	\end{itemize}
\end{frame}

\begin{frame}{Confidence intervals for $\mu$}{Width of the interval}
	\begin{itemize}
		\item Question \#2: How would the width of a confidence interval change (for a fixed $\sigma$ and $\alpha$) as $n$ is increased?
		\item[]
		\item We have both intuition and a mathematical solution:
		\begin{itemize}
			\item Intuition: If we have more measurements from a distribution, we should have a more precise estimates of a population parameter $=$ smaller interval
			\item[]
			\item Mathematical solution: Let $n_2 > n_1$, then:
			\begin{gather*}
				\sqrt{n_2} > \sqrt{n_1} \text{ and } \frac{\sigma}{\sqrt{n_2}} < \frac{\sigma}{\sqrt{n_1}} \\
				\text{so...} \\
				z_{1-\alpha / 2} \frac{\sigma}{\sqrt{n_2}} < z_{1-\alpha / 2} \frac{\sigma}{\sqrt{n_1}} \\
				width_{n_2} < width_{n_1}
			\end{gather*}
		\end{itemize}
	\end{itemize}
\end{frame}

\subsection{Confidence intervals for $\mu$ when $\sigma$ is unknown}

\begin{frame}{Outline}
	\tableofcontents[currentsection,subsectionstyle=show/shaded/hide]
\end{frame}

\begin{frame}{Confidence intervals for $\mu$}{$\sigma$ unknown}
\begin{columns}
	\begin{column}{.4 \textwidth}
		\begin{center}
			\includegraphics[width=1.2 \linewidth]{./../lecture1_Introduction/Sampling.png}
		\end{center}
	\end{column}
	\begin{column}{.5 \textwidth}
		\begin{itemize}
			\item When we draw a sample from a population in order to estimate the population mean $\mu$ of a normal random variable, the vast majority of the time we don't know the population standard deviation $\sigma$
			\item[]
			\item It seem sensible that we can estimate $\sigma$ with $s$, the sample standard deviation, and use this value when we need it (e.g. to construct confidence intervals)
		\end{itemize}
	\end{column}
\end{columns}
\end{frame}

\begin{frame}{The Student's t-distribution}
	\begin{columns}
		\begin{column}{.5 \textwidth}
			\includegraphics[width=1 \linewidth]{guiness}
		\end{column}
		\begin{column}{.5 \textwidth}
			\includegraphics[width=.8 \linewidth]{William_Sealy_Gosset}
			William Sealy Gosset (Head Experimental Brewer at Guinness)
		\end{column}
	\end{columns}
\end{frame}

\begin{frame}{Confidence intervals for $\mu$}{$\sigma$ unknown}
	\begin{itemize}
		\item William Sealy Gosset knew that confidence intervals for estimating $\mu$ constructed using $s$ instead of $\sigma$ systematically do not contain $\mu$ as often as expected
		\item[]
		\item He proposed using a different type of distribution for computing the width of confidence intervals
		\begin{itemize}
			\item Instead of using $z_{1-\alpha / 2}$, where $z$ is the value from the standard normal distribution, use $t_{1-\alpha / 2, n - 1}$ where $t$ is a value from a ``Student's t-distribution'' with $n-1$ ``degrees of freedom''
			\item We will talk about the Student's t-distribution in more detail later in the lecture
				\item[]
		\end{itemize}
			\item Confidence intervals for $\mu$ if $\sigma$ is unknown: $\bar{y} \pm t_{1-\alpha / 2, n - 1} \times s / \sqrt{n}$
	\end{itemize}
\end{frame}

\begin{frame}{Student's $t$-distribution}
	Find the value $t_{1-\alpha / 2, df}$ if $\alpha = .05$ and $df = 4$. 
	\begin{center}
		\includegraphics[width = .8\linewidth]{tTable}
	\end{center}
\end{frame}

\begin{frame}{Confidence intervals for $\mu$}{$\sigma$ unknown}
	\begin{itemize}
		\item Example problem 1: Suppose you have the following sample from the fish weights data: $\{501.85, 495.58, 502.94, 508.43, 511.33\}$. Calculate a 95\% confidence interval for $\mu$ using the given sample (we don't know the population $\sigma$). \\
		\vspace{10 pt}
		First we note that $\alpha = 0.05$, so $1-\alpha / 2 = 0.975$; and that $n - 1 = 4$. So the $t$-value that we are looking for is:
		\begin{gather*}
			t_{1-\alpha / 2, n - 1} = t_{0.975, 4} = 2.776
		\end{gather*}
		From before we have that $\bar{y}=504.026$. Then we compute the sample standard deviation, $s = 6.124$. We are almost there:
		\begin{align*}
			\bar{y} &\pm t_{1-\alpha / 2, n - 1} \times s / \sqrt{n} \\
			504.026 &\pm 2.776 \times 6.124 / 2.236 \\
			(496.423&, 511.629)
		\end{align*}
	\end{itemize}
\end{frame}

\section{Tests of hypotheses regarding Normal population mean}

\subsection{Tests given a known $\sigma$}
\begin{frame}{Outline}
	\tableofcontents[currentsection,subsectionstyle=show/shaded/hide]
\end{frame}

\begin{frame}{Tests regarding Normal population means}
	\begin{itemize}
		\item Up until now we have been discussing how to use a random sample to estimate the value of $\mu$, the population mean of random variable that has a Normal distribution
		\begin{itemize}
			\item We used $\bar{y}$ as a point estimate for $\mu$, and $\bar{y} \pm z_{1-\alpha / 2} \times \sigma / \sqrt{n}$ or $\bar{y} \pm t_{1-\alpha / 2, n - 1} \times s / \sqrt{n}$ as an interval estimate
			\item[]
		\end{itemize}
	\item Now imagine that we want to test whether $\mu$ is less than, equal to, or greater than some pre-specified value $\mu_0$
	\item[]
	\item Example: I want to test whether mean the population mean of systolic blood pressure in Asian-Americans is less than 123.5 (the mean in Caucasian Americans in a large study)
	
	\end{itemize}
\end{frame}

\begin{frame}{Process}
	The process for testing hypotheses regarding a Normal population mean is: 
	\begin{enumerate}
		\item Define a ``research hypothesis'' / alternative hypothesis. We denote this $H_a$
		\item[]
		\item Determine the appropriate null hypothesis. We denote this $H_0$
		\item[]
		\item Using a simple random sample, we compute a test statistic
		\item[]
		\item We evaluate whether the test statistic supports the rejection of the null hypothesis
	\end{enumerate}
\end{frame}

\begin{frame}{Process}{Example}
Example: We want to test whether the population mean of systolic blood pressure (SBP) in Asian-Americans is less than 123.5\\
\vspace{10 pt}
	\begin{enumerate}
		\item Define a ``research hypothesis'' / alternative hypothesis. We denote this $H_a$
		\begin{itemize}
			\item $H_a$: The population mean of SBP is less than 123.5. $\mu < 123.5$
		\end{itemize}
		\item Determine the appropriate null hypothesis. We denote this $H_0$
		\begin{itemize}
			\item $H_0$: The population mean of SBP is greater than or equal to 123.5. $\mu \geq 123.5$
		\end{itemize}
		\item Using a simple random sample, we compute a test statistic
		\item[]
		\item We evaluate whether the test statistic supports the rejection of the null hypothesis
	\end{enumerate}
\end{frame}

\begin{frame}{Process}
	\begin{itemize}
		\item We now need a \textbf{\emph{test statistic}} that will inform us whether we have evidence from the sample to reject the null hypothesis
		\item[]
		\item For normally distributed random variables, we will use the sample mean $\bar{y}$ as the test statistic
		\item[] So how can our test statistic $\bar{y}$ inform us whether we have evidence from the sample to reject the null hypothesis
	\end{itemize}
\end{frame}

\begin{frame}{Distribution of test statistics}
	\begin{itemize}
		\item We know what the distribution of test statistics is if the null hypothesis is true, that is $\mu = \mu_0$
	\end{itemize}
\begin{center}
	\includegraphics[width = .95 \linewidth]{nullTests}
\end{center}
\end{frame}

\begin{frame}{Distribution of test statistics}
	\begin{itemize}
		\item So when a test statistic is observed that is inconsistent with the null hypothesis we will see it:
	\end{itemize}
	\begin{center}
		\includegraphics[width = .95 \linewidth]{nullTests2}
	\end{center}
\end{frame}

\end{document}