\documentclass[xcolor=dvipsnames]{beamer} 
\usetheme{AnnArbor}
\usecolortheme{beaver}

\usepackage{amsmath,graphicx,booktabs,tikz,subfig,color,lmodern}
\definecolor{mycol}{rgb}{.4,.85,1}
\setbeamercolor{title}{bg=mycol,fg=black} 
\setbeamercolor{palette primary}{use=structure,fg=white,bg=red}
\setbeamercolor{block title}{fg=white,bg=red!50!black}
% \setbeamercolor{block title}{fg=white,bg=blue!75!black}

\title[Lecture 9]{Lecture 9: Inferences about the median}
\author[Patrick Trainor]{Patrick Trainor, PhD, MS, MA}
\institute[NMSU]{New Mexico State University}
\date{September 23, 2019}

\begin{document}

\begin{frame}
	\maketitle
\end{frame}

\begin{frame}{Outline}
	\tableofcontents[hideallsubsections]
\end{frame}

\section{The median}
\begin{frame}{Outline}
\tableofcontents[currentsection,subsectionstyle=show/shaded/hide]
\end{frame}

\begin{frame}{Why the median?}
	\begin{itemize}
		\item You are conducting a study of the salaries of NMSU employees (faculty, staff, student employees). You select 6 individuals at random who have the following salaries: \$39,293, \$40,434, \$53,050, \$39,025, \$15,044, \$46,908
		\item[]
		\item What is the best measure of centrality (or central tendency)?
		\begin{itemize}
			\item Sample mean: \$38,959
			\item Sample median: \$39,863.5
			\item[]
		\end{itemize}
	\item You decide you want to collect one more measurement. You randomly select Doug Martin (football coach) who has a salary of \$419,640
	\item[]
			\item What is the best measure of centrality (or central tendency)?
	\begin{itemize}
		\item Sample mean: \$93,342
		\item Sample median: \$40,434
	\end{itemize}
	\end{itemize}
\end{frame}

\begin{frame}{Why the median?}
	\begin{itemize}
		\item If we are using a measure of central tendency to get a sense of what the ``typical'' salary is at NMSU, the mean may be a very poor measure
		\item[]
		\item The inclusion of one outlier (Coach Martin) brings the sample mean up to \$93,342 although $6/7$ employees in the sample make less than \$47,000
		\item[]
		\item When a random variable has a distribution that is either skewed or a random sample has outlier(s), the sample median is a better measure of ``typical'' than the sample mean 
	\end{itemize}
\end{frame}

\begin{frame}{Computation}
	\begin{itemize}
		\item To find the sample mean we first order our measurements making a list $\{y_{(1)}, y_{(2)}, \hdots, y_{(n)} \}$
		\begin{itemize}
			\item In our salary example: $\{15044, 39025, 39293, 40434, 46908, 53050, 419640\}$
			\item[]
		\end{itemize}
		\item Then to find the sample median $\hat{M}$:
		\begin{itemize}
			\item If $n$ is an odd number, then $\hat{M} = y_{(m)}$, where $m = (n+1)/2$
			\item If $n$ is an even number, then $\hat{M} = (y_{(m)} + y_{(m+1)})/2$ where $m = n/2$
			\item[]
		\end{itemize}
		\item In our example, $n = 7$, so we $m = (7+1)/2 = 4$, and $\hat{M} = y_{(4)} = 40434$
	\end{itemize}
\end{frame}

\section{Evaluating whether data is ``approximately normal''}
\begin{frame}{Outline}
\tableofcontents[currentsection,subsectionstyle=show/shaded/hide]
\end{frame}

\begin{frame}{Normal distributions versus not so Normal distributions}
	\begin{itemize}
		\item In the last lecture we dealt with samples that came from populations that follow a Normal distribution
		\begin{itemize}
			\item We used the sample mean as an estimate for the population mean
			\item We developed confidence intervals for the population mean
			\item We tested hypotheses about the population mean
			\item[]
		\end{itemize}
		\item Those approaches are still valid if $n$ is very large
		\item[]
		\item However, if our sample comes from a population that is highly skewed and not mound shaped: 
		\begin{itemize}
			\item We should use the median as the measure of centrality
			\item We will develop confidence intervals based on the median
			\item We will test hypotheses based on the location of the median
		\end{itemize}
	\end{itemize}
\end{frame}

\begin{frame}{How to determine if a distribution is not Normal}
\begin{center}
	\includegraphics[width=.775 \linewidth]{NormNotNorm}
\end{center}
\end{frame}

\begin{frame}{Q-Q Plots}
	\begin{itemize}
		\item To determine if a distribution is not normal, generate a Q-Q plot
		\item[]
		\item Preparation:
		\begin{enumerate}
			\item Order the measurements in the dataset from smallest to largest: $\{y_{(1)}, y_{(2)}, \hdots, y_{(n)} \}$
			\item Compute the quantile for each measurement in the dataset: $Q((i-0.5) /n)$
		\end{enumerate}
	\item Example. Our salary dataset: $\{15044, 39025, 39293, 40434, 46908, 53050, 419640\}$
	\begin{align*}
		15044 &= Q((1-0.5)/7) = 0.071 \\
		39025 &= Q((2-0.5)/7) = 0.214 \\
		39239 &= Q((3-0.5)/7) = 0.357 \\
		\vdots & \\
		419640 &= Q((7-0.5)/7) = 0.929
	\end{align*}
	\end{itemize}
\end{frame}

\begin{frame}{Q-Q Plots}
\begin{columns}
	\begin{column}{.45 \textwidth}
Process:
			\begin{enumerate}
				\item Order the measurements in the dataset from smallest to largest: $\{y_{(1)}, y_{(2)}, \hdots, y_{(n)} \}$
				\item Compute the quantile for each measurement in the dataset: $Q((i-0.5) /n)$
				\item Find normal quantiles: $z_{(i-0.5)/n}$
				\item Plot: $\{(z_{(0.5)/n}, y_{(1)}),(z_{(1.5)/n}, y_{(2)}),\hdots, (z_{(n-0.5)/n}, y_{(n)})\}$
			\end{enumerate}
	\end{column}
	\begin{column}{0.5 \textwidth}
		\vspace{-20pt}
		\begin{center}
			\begin{tabular}{|c|c|c|c|}
				\hline 
				$i$ & $y_{(i)}$ & $(i-0.5)/n$ & $z_{(i-0.5)/n}$ \\
				\hline \hline
				1 & 15044 & 0.071 & -1.468 \\ \hline
				2 & 39025 & 0.214 & -0.793 \\ \hline
				3 & 39293 & 0.356 & -0.366 \\ \hline
				4 & 40434 & 0.500 & 0 \\ \hline
				5 & 46908 & 0.643 & 0.366 \\ \hline
				6 & 53050 & 0.786 & 0.793 \\ \hline
				7 & 419640 & 0.929 & 1.468 \\ \hline
			\end{tabular}
		\end{center}
	\end{column}
\end{columns}

\end{frame}

\begin{frame}{Q-Q Plots}{Salary Data}
	\begin{center}
		\includegraphics[width = .8 \linewidth]{qq1}
	\end{center}
\end{frame}

\begin{frame}{Q-Q Plots}{Sample of Normally distributed data}
\begin{center}
	\includegraphics[width = .8 \linewidth]{qq2}
\end{center}
\end{frame}

\section{Confidence intervals}

\section{Hypothesis tests}

\end{document}