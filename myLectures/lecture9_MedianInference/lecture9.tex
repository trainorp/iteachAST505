\documentclass[xcolor=dvipsnames]{beamer} 
\usetheme{AnnArbor}
\usecolortheme{beaver}

\usepackage{amsmath,graphicx,booktabs,tikz,subfig,color,lmodern}
\definecolor{mycol}{rgb}{.4,.85,1}
\setbeamercolor{title}{bg=mycol,fg=black} 
\setbeamercolor{palette primary}{use=structure,fg=white,bg=red}
\setbeamercolor{block title}{fg=white,bg=red!50!black}
% \setbeamercolor{block title}{fg=white,bg=blue!75!black}

\title[Lecture 9]{Lecture 9: Inferences about the median}
\author[Patrick Trainor]{Patrick Trainor, PhD, MS, MA}
\institute[NMSU]{New Mexico State University}
\date{}

\begin{document}

\begin{frame}
	\maketitle
\end{frame}

\begin{frame}{Outline}
	\tableofcontents[hideallsubsections]
\end{frame}

\section{The median}
\begin{frame}{Outline}
\tableofcontents[currentsection,subsectionstyle=show/shaded/hide]
\end{frame}

\begin{frame}{Why the median?}
	\begin{itemize}
		\item You are conducting a study of the salaries of NMSU employees (faculty, staff, student employees). You select 6 individuals at random who have the following salaries: \$39,293, \$40,434, \$53,050, \$39,025, \$15,044, \$46,908 \pause
		\item[]
		\item What is the best measure of centrality (or central tendency)? \pause
		\begin{itemize}
			\item Sample mean: \$38,959 \pause
			\item Sample median: \$39,863.5 \pause
			\item[]
		\end{itemize}
	\item You decide you want to collect one more measurement. You randomly select Doug Martin (football coach) who has a salary of \$419,640 \pause
	\item[]
			\item What is the best measure of centrality (or central tendency)? \pause
	\begin{itemize}
		\item Sample mean: \$93,342 \pause
		\item Sample median: \$40,434
	\end{itemize}
	\end{itemize}
\end{frame}

\begin{frame}{Why the median?}
	\begin{itemize}
		\item If we are using a measure of central tendency to get a sense of what the ``typical'' salary is at NMSU, the mean may be a very poor measure \pause
		\item[]
		\item The inclusion of one outlier (Coach Martin) brings the sample mean up to \$93,342 although $6/7$ employees in the sample make less than \$47,000 \pause
		\item[]
		\item When a random variable has a distribution that is either skewed or a random sample has outlier(s), the sample median is a better measure of ``typical'' than the sample mean 
	\end{itemize}
\end{frame}

\begin{frame}{Computation}
	\begin{itemize}
		\item To find the sample mean we first order our measurements making a list $\{y_{(1)}, y_{(2)}, \hdots, y_{(n)} \}$ \pause
		\begin{itemize}
			\item In our salary example: $\{15044, 39025, 39293, 40434, 46908, 53050, 419640\}$ \pause
			\item[]
		\end{itemize}
		\item Then to find the sample median $\hat{M}$: \pause
		\begin{itemize}
			\item If $n$ is an odd number, then $\hat{M} = y_{(m)}$, where $m = (n+1)/2$ \pause
			\item If $n$ is an even number, then $\hat{M} = (y_{(m)} + y_{(m+1)})/2$ where $m = n/2$ \pause
			\item[]
		\end{itemize}
		\item In our example, $n = 7$, so we $m = (7+1)/2 = 4$, and $\hat{M} = y_{(4)} = 40434$
	\end{itemize}
\end{frame}

\section{Evaluating whether data is ``approximately normal''}
\begin{frame}{Outline}
\tableofcontents[currentsection,subsectionstyle=show/shaded/hide]
\end{frame}

\begin{frame}{Normal distributions versus not so Normal distributions}
	\begin{itemize}
		\item In the last lecture we dealt with samples that came from populations that follow a Normal distribution \pause
		\begin{itemize}
			\item We used the sample mean as an estimate for the population mean \pause
			\item We developed confidence intervals for the population mean \pause
			\item We tested hypotheses about the population mean \pause
			\item[]
		\end{itemize}
		\item Those approaches are still valid if $n$ is very large \pause
		\item[]
		\item However, if our sample comes from a population that is highly skewed and not mound shaped:  \pause
		\begin{itemize}
			\item We should use the median as the measure of centrality \pause
			\item We will develop confidence intervals based on the median \pause
			\item We will test hypotheses based on the location of the median
		\end{itemize}
	\end{itemize}
\end{frame}

\begin{frame}{How to determine if a distribution is not Normal}
\begin{center}
	\includegraphics[width=.775 \linewidth]{NormNotNorm}
\end{center}
\end{frame}

\begin{frame}{Q-Q Plots}
	\begin{itemize}
		\item To determine if a distribution is not normal, generate a Q-Q plot \pause
		\item[]
		\item Preparation: \pause
		\begin{enumerate}
			\item Order the measurements in the dataset from smallest to largest: $\{y_{(1)}, y_{(2)}, \hdots, y_{(n)} \}$ \pause
			\item Compute the quantile for each measurement in the dataset: $Q((i-0.5) /n)$ \pause
		\end{enumerate}
	\item Example. Our salary dataset: $\{15044, 39025, 39293, 40434, 46908, 53050, 419640\}$ \pause
	\begin{align*}
		15044 &= Q((1-0.5)/7) = 0.071 \\
		39025 &= Q((2-0.5)/7) = 0.214 \\
		39239 &= Q((3-0.5)/7) = 0.357 \\
		\vdots & \\
		419640 &= Q((7-0.5)/7) = 0.929
	\end{align*}
	\end{itemize}
\end{frame}

\begin{frame}{Q-Q Plots}
\begin{columns}
	\begin{column}{.45 \textwidth}
Process:
			\begin{enumerate}
				\item Order the measurements in the dataset from smallest to largest: $\{y_{(1)}, y_{(2)}, \hdots, y_{(n)} \}$ \pause
				\item Compute the quantile for each measurement in the dataset: $Q((i-0.5) /n)$ \pause
				\item Find normal quantiles: $z_{(i-0.5)/n}$ \pause
				\item Plot: $\{(z_{(0.5)/n}, y_{(1)}),(z_{(1.5)/n}, y_{(2)}),\hdots, (z_{(n-0.5)/n}, y_{(n)})\}$ \pause
			\end{enumerate}
	\end{column}
	\begin{column}{0.5 \textwidth}
		\vspace{-20pt}
		\begin{center}
			\begin{tabular}{|c|c|c|c|}
				\hline 
				$i$ & $y_{(i)}$ & $(i-0.5)/n$ & $z_{(i-0.5)/n}$ \\
				\hline \hline
				1 & 15044 & 0.071 & -1.468 \\ \hline
				2 & 39025 & 0.214 & -0.793 \\ \hline
				3 & 39293 & 0.356 & -0.366 \\ \hline
				4 & 40434 & 0.500 & 0 \\ \hline
				5 & 46908 & 0.643 & 0.366 \\ \hline
				6 & 53050 & 0.786 & 0.793 \\ \hline
				7 & 419640 & 0.929 & 1.468 \\ \hline
			\end{tabular}
		\end{center}
	\end{column}
\end{columns}

\end{frame}

\begin{frame}{Q-Q Plots}{Salary Data}
	\begin{center}
		\includegraphics[width = .8 \linewidth]{qq1}
	\end{center}
\end{frame}

\begin{frame}{Q-Q Plots}{Sample of Normally distributed data}
\begin{center}
	\includegraphics[width = .8 \linewidth]{qq2}
\end{center}
\end{frame}

\section{Confidence intervals}
\begin{frame}{Outline}
\tableofcontents[currentsection,subsectionstyle=show/shaded/hide]
\end{frame}

\subsection{Small sample method $n \leq 50$}
\begin{frame}{Outline}
\tableofcontents[currentsection,subsectionstyle=show/shaded/hide]
\end{frame}

\begin{frame}{Confidence intervals for the median}
	\begin{itemize}
		\item A confidence interval for $M$, the population median with level of confidence at least $100(1-\alpha)$\% is: \pause
		\begin{gather*}
		(M_L, M_U) = (y_{(L_{\alpha / 2})}, y_{(U_{\alpha/2})})
		\end{gather*} \pause
		where:
		\begin{gather*}
			L_{\alpha / 2} = C_{\alpha(2), n} + 1 \\
			U_{\alpha / 2} = n - C_{\alpha(2), n}
		\end{gather*} \pause
		The values $C_{\alpha(2),n}$ are from the Binomial distribution and are found in the Appendix of the textbook (Table 4).
	\end{itemize}
\end{frame}

\begin{frame}{Confidence intervals for the median}{Example}
	\begin{itemize}
		\item {\tiny The following is data on the amount of phosphorous (in ppm) in a sample of corn: \{64,  60, 71, 61, 54, 77, 81, 93, 93, 51, 76, 96, 77, 383, 95, 222, 168, 99\}. The Q-Q plot suggests that the data is not normally distributed. Let's compute the median phosphorous level and 90\% confidence interval }
	\end{itemize}
\vspace{-5pt}
\begin{center}
	\includegraphics[width = .6\linewidth]{phosQQ}
\end{center}
\end{frame}

\begin{frame}{Confidence intervals for the median}{Example}
\begin{itemize}
	\item First we order the data: \{51, 54, 60, 61, 64, 71, 76, 77, 77, 81, 93, 93, 95, 96, 99, 168 222, 383\} \pause
	\item[]
	\item Since there are 18 observations: \pause
	\begin{align*}
		\hat{M} &= (y_{(m)} + y_{(m+1)})/2 \quad \text{where } m = n/2 \\
		&=(y_{(9)} + y_{(10)})/2 \\
		&= (77 + 81)/2 = 79
	\end{align*}
\end{itemize}
\end{frame}

\begin{frame}{Confidence intervals for the median}{Example}
\begin{itemize}
	\item For our 90\% confidence intervals, we have $\alpha = 0.10$, and we need to calculate:
	\begin{align*}
		L_{\alpha /2} = C_{\alpha (2), n} +1 = C_{0.10(2), 18} +1 = 5+1=6 \\
	\end{align*} 
	and:
	\begin{align*}
		U_{\alpha /2} = n- C_{\alpha (2), n} = 18 - C_{0.10(2), 18} = 18-5 = 13
	\end{align*}\pause
	Finally, we are looking for:
	\begin{align*}
		(M_L, M_U) = (y_{(L_{\alpha / 2})}, y_{(U_{\alpha/2})}) = (y_{(6)}, y_{(13)}) = (71, 95)
	\end{align*}
\end{itemize}
\end{frame}

\subsection{Normal approximation}
\begin{frame}{Outline}
\tableofcontents[currentsection,subsectionstyle=show/shaded/hide]
\end{frame}

\begin{frame}{Confidence intervals for the median}{For large $n$}
	\begin{itemize}
		\item When the sample size, $n$ is large (for example $>50$), we use a different formula that uses a normal approximation to find $C_{\alpha(2),n}$:
		\begin{gather*}
		C_{\alpha(2),n} \approx \frac{n}{2} - z_{\alpha / 2} \sqrt{n/4}
		\end{gather*} \pause
		\item Example. Determine $C_{\alpha(2),n}$ for $n = 50$ and $\alpha = 0.05$
		\begin{gather*}
		C_{0.05(2),50} \approx \frac{50}{2} - z_{0.025} \sqrt{50/4} = 18.07 \approx 18
		\end{gather*}
	\end{itemize}
\end{frame}

\section{Hypothesis tests}
\begin{frame}{Outline}
\tableofcontents[currentsection,subsectionstyle=show/shaded/hide]
\end{frame}

\subsection{Small sample method $n \leq 50$}
\begin{frame}{Outline}
\tableofcontents[currentsection,subsectionstyle=show/shaded/hide]
\end{frame}

\begin{frame}{The Sign Test}{Background}
	\begin{itemize}
		\item To test one-sided hypotheses, we use the \textbf{\emph{sign test}} \pause
		\item[]
		\item Given a null hypothesis value of $M_0$, we define $W_i = y_i - M_0$ \pause
		\item[]
		\item The sign test statistic, $B$, is the number of $W_i$'s that are positive \pause
		\item[]
		\item If $M = M_0$, there is a 50\% chance that $y_i$ is greater than $M_0$, and 50\% chance that $y_i$ is less than $M_0$ \pause
		\item[]
		\item $W_i$ is then a binomial random variable, with $\pi = 0.5$
	\end{itemize}
\end{frame}

\begin{frame}{The Sign Test}{Process}
\begin{itemize}
\item Hypotheses:
\begin{itemize}
	\item Case 1. $H_0: M \leq M_0$ versus $H_a: M > M_0$ (right / upper-tailed test) \pause
	\item Case 2. $H_0: M \geq M_0$ versus $H_a: M < M_0$ (left / lower-tailed test) \pause
	\item Case 3. $H_0: M = M_0$ versus $H_a: M \neq M_0$ (two-tailed test) \pause
	\item[]
\end{itemize}
\item Test statistic: $B = \# \text{ of positive } W_i$ where $W_i = y_i - M_0$ \pause
\item[]
\item Rejection region / rules: \pause
\begin{itemize}
	\item Case 1. Reject $H_0$ if $B \geq n - C_{\alpha(1), n}$ \pause
	\item Case 2. Reject $H_0$ if $B \leq C_{\alpha(1), n}$ \pause
	\item Case 3. Reject $H_0$ if $B \geq n - C_{\alpha(2), n}$ or $B \leq C_{\alpha(2), n}$
\end{itemize}
\end{itemize}
\end{frame}

\begin{frame}{The Sign Test}{Example}
	\begin{itemize}
		\item You hypothesize that the median level of phosphorous in a population of corn is greater than 75. Given the following data \{51, 54, 60, 61, 64, 71, 76, 77, 77, 81, 93, 93, 95, 96, 99, 168 222, 383\}, what is your conclusion (assume $\alpha = 0.10$)? \pause
		\item[]
		\item Hypotheses: \pause
		\begin{itemize}
			\item $H_0: M \leq 75$ \pause
			\item $H_a: M > 75$ \pause
			\item Reject $H_0$ if $B \geq n - C_{\alpha(1), n}$ \pause
			\item[]
		\end{itemize}
	\item Test statistic: \pause
	\begin{itemize}
		\item $W$'s: \{-24, -21, -15, -14, -11,  -4, 1, 2, 2, 6, 18, 18, 20, 21, 24, 93, 147, 308 \} \pause
		\item $B = 12$
	\end{itemize}
	\end{itemize}
\end{frame}

\begin{frame}{The Sign Test}{Example}
\begin{itemize}
	\item You hypothesize that the median level of phosphorous in a population of corn is greater than 75. Given the following data \{51, 54, 60, 61, 64, 71, 76, 77, 77, 81, 93, 93, 95, 96, 99, 168 222, 383\}, what is your conclusion (assume $\alpha = 0.10$)?
	\item[]
	\item RR: Reject $H_0$ if $B \geq 18 - C_{0.10(1), 18} = 18 - 5 = 13$ \pause
	\item[]
	\item Since $B = 12$, we fail to reject the null hypothesis \pause
	\item[]
	\item This can happen. Although we have evidence that $M > 75$, since $\hat{M} = 79$, the evidence is not convincing enough for us to conclude definitively that $M > 75$
\end{itemize}
\end{frame}

\begin{frame}{The Sign Test}{An Alternative Alternative}
	\begin{itemize}
		\item Assume instead that we had hypothesized that the median level of phosphorous in a population of corn is less than 100 \pause
		\item[]
		\item Given the following data \{51, 54, 60, 61, 64, 71, 76, 77, 77, 81, 93, 93, 95, 96, 99, 168 222, 383\}, what is your conclusion (assume $\alpha = 0.10$)? \pause
		\item[]
		\item Hypotheses: \pause
		\begin{itemize}
			\item $H_0: M \geq 100$ \pause
			\item $H_a: M < 100$ \pause
			\item Reject $H_0$ if $B \leq C_{\alpha(1), n}$ \pause
			\item[]
		\end{itemize}
	\item Test statistic: \pause
	\begin{itemize}
		\item $W$'s: \{-49, -46, -40, -39, -36, -29, -24, -23, -23, -19, -7, -7, -5, -4, -1, 68, 122, 283\} \pause
		\item $B = 3$
	\end{itemize}
	\end{itemize}
\end{frame}

\begin{frame}{The Sign Test}{An Alternative Alternative}
\begin{itemize}
	\item Assume instead that we had hypothesized that the median level of phosphorous in a population of corn is less than 100. What is your conclusion (assume $\alpha = 0.10$)? \pause
	\item[]
	\item RR: Reject $H_0$ if $B \leq C_{0.10(1), 18} = 5 $ \pause
	\item[]
	\item Since $B = 3$, we reject the null hypothesis, we have evidence to conclude that the population median is less than 100
\end{itemize}
\end{frame}

\subsection{Normal approximation}
\begin{frame}{Outline}
\tableofcontents[currentsection,subsectionstyle=show/shaded/hide]
\end{frame}

\begin{frame}{The Sign Test}{Normal Approximation}
	\begin{itemize}
		\item When the sample size is large, we use the normal approximation to the binomial \pause
		\item[]
		\item We use the following ``standardized'' test statistic: \pause
		\begin{gather*}
		B^* = \frac{B-(n/2)}{\sqrt{n/4}}
		\end{gather*}
		
		\item Rejection regions / rules: 
		\begin{itemize}
			\item Case 1. Reject $H_0: M \leq M_0$ if $B^* \geq z_{\alpha}$ with $p$-value $=P(z \geq B^*)$ \pause
			\item Case 2. Reject $H_0: M \geq M_0$ if $B^* \leq -z_{\alpha}$ with $p$-value $=P(z \leq B^*)$
			\pause
			\item Case 3. Reject $H_0: M = M_0$ if $|B^*| \geq z_{\alpha/2}$ with $p$-value $=2\times P(z \geq |B^*|)$
		\end{itemize}
	\end{itemize}
\end{frame}

\begin{frame}{Normal Approximation Sign Test}{Example}
	\begin{itemize}
		\item You know that the median tumor thickness (mm) for melanomas seen in a skin clinic is 1.94. You also know that the tumor thickness does not follow a normal distribution (it is skewed to the right) \pause
		\item[]
		\item You want to determine if larger tumors are associated with a greater risk of death. You review 57 cases for which patient's died from cancer and observed the following initial thicknesses:
		\begin{center}
			\includegraphics[width = 1 \linewidth]{melDied}
		\end{center} \pause
	\item[]
		\item Are tumor thicknesses greater for cases in which the patient died? Test hypotheses at $\alpha = 0.05$.
	\end{itemize}
\end{frame}

\begin{frame}{Normal Approximation Sign Test}{Example}
	\begin{itemize}
		\item Hypotheses: \pause
		\begin{itemize}
			\item $H_0: M \leq 1.94$ \pause
			\item $H_a: M > 1.94$ \pause
			\item[]
		\end{itemize}
		\item Rejection region / rule: Reject $H_0: M \leq 1.94$ if $B^* \geq z_{\alpha}$ with $p$-value $=P(z \geq B^*)$ \pause
		\item[]
		\item We compute $W_i$ for each measurement: 
		\begin{center}
			\includegraphics[width = \linewidth]{melDied2}
		\end{center} \pause
		\item[]
		\item So $B = 44$ and $B^* = \frac{B-(n/2)}{\sqrt{n/4}} = \frac{44-(57/2)}{\sqrt{57/4}} = 4.106$
	\end{itemize}
\end{frame}

\begin{frame}{Normal Approximation Sign Test}{Example}
\begin{itemize}
	\item So we have $B^* = 4.106$ \pause
	\item[]
	\item Rejection region / rule: Reject $H_0: M \leq 1.94$ if $B^* \geq z_{\alpha}$ with $p$-value $=P(z \geq B^*)$ \pause
	\begin{itemize}
		\item $z_{\alpha} = z_{.05} = 1.645$ \pause
		\item[]
		\item Since $B^* = 4.106 > 1.645$, we reject the null hypothesis \pause
		\item[]
		\item $p$-value $=P(z \geq B^*) =  0.00002$ \pause
		\item[]
	\end{itemize}
	\item Conclusion: We have evidence that median tumor size in patients who died is greater than 1.94 mm. The $p$-value is 0.00002
\end{itemize}
\end{frame}

\begin{frame}{The end of Lecture \#9}
	\begin{center}
		\includegraphics[width=.9\linewidth]{DSC_2051_v1}
	\end{center}
\end{frame}

\end{document}